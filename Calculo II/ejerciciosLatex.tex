%%%%%%%%%%%%%%%%%%%%%%%%%%%%%%%%%%%%%%%%%%%%%%%%%%%%%%%%%%%%%%%%%%%%%%%%% 
% 
% Grado en Estadística.
% Universidad de Granada.
% Curso 2017/18.
% 
% 
% Colaboradores:
% Miguel Anguita Ruiz
% 
% Agradecimientos:
% Andrés Herrera (@andreshp) y Mario Román (@M42) por
% las plantillas base.
% 
% Sitio original:
% https://github.com/libreim/apuntesDGIIM/
% 
% Licencia:
% CC BY-NC-SA 4.0 (https://creativecommons.org/licenses/by-nc-sa/4.0/)
% 
%%%%%%%%%%%%%%%%%%%%%%%%%%%%%%%%%%%%%%%%%%%%%%%%%%%%%%%%%%%%%%%%%%%%%%%%% 


% ------------------------------------------------------------------------------
% ACKNOWLEDGMENTS
% ------------------------------------------------------------------------------

%%%%%%%%%%%%%%%%%%%%%%%%%%%%%%%%%%%%%%%%%%%%%%%%%%%%%%%%%%%%%%%%%%%%%%%% 
% Plantilla básica de Latex en Español.
% 
% Autor: Andrés Herrera Poyatos (https://github.com/andreshp) 
% 
% Es una plantilla básica para redactar documentos. Utiliza el paquete fancyhdr
% para darle un estilo moderno pero serio.
% 
% La plantilla se encuentra adaptada al español.
% 
%%%%%%%%%%%%%%%%%%%%%%%%%%%%%%%%%%%%%%%%%%%%%%%%%%%%%%%%%%%%%%%%%%%%%%%%% 

%%%%%%%%%%%%%%%%%%%%%%%%%%%%%%%%%%%%%%%%%%%%%%%%%%%%%%%%%%%%%%%%%%%%%%%%% 
% Plantilla de Trabajo
% Modificación de una plantilla de Latex de Frits Wenneker para adaptarla 
% al castellano y a las necesidades de escribir informática y matemáticas.
% 
% Editada por: Mario Román
% 
% License:
% CC BY-NC-SA 3.0 (http://creativecommons.org/licenses/by-nc-sa/3.0/)
%%%%%%%%%%%%%%%%%%%%%%%%%%%%%%%%%%%%%%%%%%%%%%%%%%%%%%%%%%%%%%%%%%%%%%%%% 

%%%%%%%%%%%%%%%%%%%%%%%%%%%%%%%%%%%%%%%%%%%%%%%%%%%%%%%%%%%%%%%%%%%%%%%%% 
% Short Sectioned Assignment
% LaTeX Template
% Version 1.0 (5/5/12)
% 
% This template has been downloaded from:
% http://www.LaTeXTemplates.com
% 
% Original author:
% Frits Wenneker (http://www.howtotex.com)
% 
% License:
% CC BY-NC-SA 3.0 (http://creativecommons.org/licenses/by-nc-sa/3.0/)
% 
%%%%%%%%%%%%%%%%%%%%%%%%%%%%%%%%%%%%%%%%%%%%%%%%%%%%%%%%%%%%%%%%%%%%%%%%% 


% Tipo de documento y opciones.
%\documentclass[11pt, a4paper, twoside]{article} % Usar para imprimir
\documentclass[11pt, a4paper]{article}

% ---------------------------------------------------------------------------
% PAQUETES
% ---------------------------------------------------------------------------

% Idioma y codificación para Español.
\usepackage[utf8]{inputenc}
\usepackage[spanish, es-tabla, es-lcroman, es-noquoting]{babel}
\selectlanguage{spanish} 
% \usepackage[T1]{fontenc}

% Fuente utilizada.
\usepackage{courier}    % Fuente Courier.
\usepackage{microtype}  % Mejora la letra final de cara al lector.

% Diseño de página.
\usepackage{fancyhdr}   % Utilizado para hacer cabeceras y pies de página.

\usepackage{titlesec} 	% Utilizado para hacer títulos propios.
\usepackage{lastpage}   % Referencia a la última página.
\usepackage{extramarks} % Marcas extras. Utilizado en pie de página y cabecera.
\usepackage[parfill]{parskip}    % Crea una nueva línea entre párrafos.
\usepackage{geometry}            % Geometría de las páginas.

% Símbolos y matemáticas.
\usepackage{amssymb, amsmath, amsthm, amsfonts, amscd}
\usepackage{upgreek}
\usepackage{mathrsfs}

\usepackage{mdframed}

% Gráficos

\usepackage{pgf,tikz}
\usepackage{tkz-euclide}
\usetkzobj{all}

% Otros.
\usepackage{enumitem}   % Listas mejoradas.
\usepackage{hyperref}
\usepackage{graphicx}   % Gráficos.
\usepackage[space]{grffile}  % Permitir espacios en rutas de gráficos
\usepackage{xcolor}     % Colores.8

% Fuentes personalizadas

\usepackage[scaled=.85]{newpxtext,newpxmath}
\usepackage[scaled=.85]{FiraSans}
\usepackage[T1]{fontenc}

% Código para ajustar las fuentes matemáticas al estilo del texto
% que le rodea

\DeclareMathVersion{sans}
\SetSymbolFont{operators}{sans}{OT1}{cmbr}{m}{n}
\SetSymbolFont{letters}{sans}{OML}{cmbr}{m}{it}
\SetSymbolFont{symbols}{sans}{OMS}{cmbrs}{m}{n}
\SetMathAlphabet{\mathit}{sans}{OT1}{cmbr}{m}{sl}
\SetMathAlphabet{\mathbf}{sans}{OT1}{cmbr}{bx}{n}
\SetMathAlphabet{\mathtt}{sans}{OT1}{cmtl}{m}{n}
\SetSymbolFont{largesymbols}{sans}{OMX}{iwona}{m}{n}

\DeclareMathVersion{boldsans}
\SetSymbolFont{operators}{boldsans}{OT1}{cmbr}{b}{n}
\SetSymbolFont{letters}{boldsans}{OML}{cmbrm}{b}{it}
\SetSymbolFont{symbols}{boldsans}{OMS}{cmbrs}{b}{n}
\SetMathAlphabet{\mathit}{boldsans}{OT1}{cmbr}{b}{sl}
\SetMathAlphabet{\mathbf}{boldsans}{OT1}{cmbr}{bx}{n}
\SetMathAlphabet{\mathtt}{boldsans}{OT1}{cmtl}{b}{n}
\SetSymbolFont{largesymbols}{boldsans}{OMX}{iwona}{bx}{n}

\newif\IfInSansMode
\let\oldsf\sffamily
\renewcommand*{\sffamily}{\oldsf\mathversion{sans}\InSansModetrue}
\let\oldmd\mdseries
\renewcommand*{\mdseries}{\oldmd\IfInSansMode\mathversion{sans}\fi\relax}
\let\oldbf\bfseries
\renewcommand*{\bfseries}{\oldbf\IfInSansMode\mathversion{boldsans}\else%
   \mathversion{bold}\fi\relax}
\let\oldnorm\normalfont
\renewcommand*{\normalfont}{\oldnorm\InSansModefalse\mathversion{normal}}
\let\oldrm\rmfamily
\renewcommand*{\rmfamily}{\oldrm\InSansModefalse\mathversion{normal}}

% Colores

\definecolor{50}{HTML}{E8F5E9}
\definecolor{300}{HTML}{81C784}
\definecolor{500}{HTML}{4CAF50}
\definecolor{700}{HTML}{388E3C}

% ---------------------------------------------------------------------------
% OPCIONES PERSONALIZADAS
% ---------------------------------------------------------------------------

% Formato de texto.
\linespread{1.3}            % Espaciado entre líneas.
\setlength\parindent{0pt}   % No indentar el texto por defecto.
\setlist{leftmargin=.5in}   % Indentación para las listas.

% Estilo de página.
\pagestyle{fancy}
\fancyhf{}
\geometry{left=3cm,right=3cm,top=3cm,bottom=3cm}   % Márgenes y cabecera.

% Estilo de las cabeceras

\titleformat{\section}
  {\Large\bfseries\sffamily}{\thesection}{1em}{}
\titleformat{\subsection}
  {\large\sffamily}{\thesubsection}{1em}{}
\titleformat{\subsubsection}
  {\sffamily}{\thesubsubsection}{1em}{}

% Estilo de enlaces
\hypersetup{
  % hidelinks = true,   % Oculta todos los enlaces.
  colorlinks = true,   % Muestra todos los enlaces, sin bordes alrededor.
  linkcolor=black,     % Color de enlaces genéricos
  citecolor={blue!50!black},   % Color de enlaces de referencias
  urlcolor={blue!80!black}     % Color de enlaces de URL
}

% Ruta donde buscar gráficos
\graphicspath{{../Recursos/Plantillas/} {Recursos/Plantillas/} {./img/} {Análisis Matemático II/img/}}

% Redefinir entorno de demostración (reducir espacio superior)
% \makeatletter
% \renewenvironment{proof}[1][\proofname] {\vspace{-15pt}\par\pushQED{\qed}\normalfont\topsep6\p@\@plus6\p@\relax\trivlist\item[\hskip\labelsep\it#1\@addpunct{.}]\ignorespaces}{\popQED\endtrivlist\@endpefalse}
% \makeatother

% Aumentar el tamaño del interlineado
\linespread{1.3}

% Permitir salto de página en ecuaciones
\allowdisplaybreaks


% ---------------------------------------------------------------------------
% COMANDOS PERSONALIZADOS
% ---------------------------------------------------------------------------

% Redefinir letra griega épsilon.
\let\epsilon\upvarepsilon

% Valor absoluto: \abs{}
\providecommand{\abs}[1]{\lvert#1\rvert}    

% Fracción grande: \ddfrac{}{}
\newcommand\ddfrac[2]{\frac{\displaystyle #1}{\displaystyle #2}}

% Texto en negrita en modo matemática: \bm{}
\newcommand{\bm}[1]{\boldsymbol{#1}}

% Línea horizontal.
\newcommand{\horrule}[1]{\rule{\linewidth}{#1}}

% Letras de conjuntos
\newcommand{\R}{\mathbb{R}} \newcommand{\N}{\mathbb{N}}

% Sucesiones
\newcommand{\xn}{\{x_n\}}
\newcommand{\fn}{\{f_n\}}

% Letra griega "chi" en línea con el texto
\DeclareRobustCommand{\rchi}{{\Large \mathpalette\irchi\relax}}
\newcommand{\irchi}[2]{\raisebox{0.4\depth}{$#1\chi$}} % inner command, used by \rchi 

% Letra 'omega'
\newcommand{\W}{\Omega}
\newcommand{\w}{\omega}


% ---------------------------------------------------------------------------
% CABECERA Y PIE DE PÁGINA
% ---------------------------------------------------------------------------

\renewcommand{\sectionmark}[1]{%
\markboth{#1}{}}

\renewcommand{\subsectionmark}[1]{%
\markright{#1}{}}

%\addtolength{\headheight}{4ex}
\renewcommand{\headrulewidth}{0pt}

\addtolength{\headwidth}{\marginparsep}
%\addtolength{\headwidth}{\marginparwidth}

\fancypagestyle{section}{%
  \fancyhead{}
  %\addtolength{\headheight}{-10ex}
  %\renewcommand{\headheight}{0pt}%
  %\setlength{\footskip}{-48pt}%
  \fancyfoot[LE,RO]{\Large\sffamily\thepage}
  \renewcommand{\headrulewidth}{0pt}%
  \renewcommand{\footrulewidth}{0pt}%
}

\let\originalsection\section
\RenewDocumentCommand{\section}{som}{%
  \IfBooleanTF{#1}
    {\originalsection*{#3}}
    {\IfNoValueTF{#2}
      {\originalsection{#3}}
      {\originalsection[#2]{#3}}%
    }%
  \thispagestyle{section}%
}

\fancyhead[LE,RO]{\rule[-4ex]{0pt}{2ex}\sffamily\textsl{\rightmark}}
\fancyhead[LO,RE]{\sffamily{\leftmark}}
\fancyfoot[LE,RO]{\Large\sffamily\thepage}



% ---------------------------------------------------------------------------
% ENTORNOS PARA MATEMÁTICAS
% ---------------------------------------------------------------------------

% Nuevo estilo para definiciones.
\newtheoremstyle{definition-style} % Nombre del estilo.
{}               % Espacio por encima.
{}               % Espacio por debajo.
{}                   % Fuente del cuerpo.
{}                   % Identación.
{\bf\sffamily}                % Fuente para la cabecera.
{.}                  % Puntuación tras la cabecera.
{.5em}               % Espacio tras la cabecera.
{\thmname{#1}\thmnumber{ #2}\thmnote{ (#3)}}     % Especificación de la cabecera (actual: nombre en negrita).

% Nuevo estilo para notas.
\newtheoremstyle{remark-style} 
{10pt}                
{10pt}                
{}                   
{}                   
{\itshape \sffamily}          
{.}                  
{.5em}               
{}                  

% Nuevo estilo para teoremas y proposiciones.
\newtheoremstyle{theorem-style}
{}                
{}                
{}           
{}                  
{\bfseries \sffamily}             
{.}                
{.5em}           
{\thmname{#1}\thmnumber{ #2}\thmnote{ (#3)}}

% Nuevo estilo para ejemplos.
\newtheoremstyle{example-style}
{10pt}                
{10pt}                
{}                  
{}                   
{\bf \sffamily}              
{}                 
{.5em}               
{\thmname{#1}\thmnumber{ #2.}\thmnote{ #3.}}  

% Nuevo estilo para la demostración
%\renewenvironment{proof}{{\itshape \sffamily Demostración \\}}{\hspace*{\fill}\qed}

\makeatletter
\renewenvironment{proof}[1][\proofname] {\par\pushQED{\qed}\normalfont\topsep6\p@\@plus6\p@\relax\trivlist\item[\hskip\labelsep\itshape\sffamily#1\@addpunct{.}]\ignorespaces}{\popQED\endtrivlist\@endpefalse}
\makeatother

% Configuración general de mdframe, los estilos de los teoremas, etc
\mdfsetup{skipabove=12pt,skipbelow=12pt,innertopmargin=12pt,innerbottommargin=4pt}

% Creamos los 'marcos' de los estilos

\mdfdefinestyle{nth-frame}{
	linewidth=2pt, %
	linecolor= 500, % 
	%linecolor=black,
	topline=false, %
	bottomline=false, %
	rightline=false,%
	leftmargin=0pt, %
	innerleftmargin=1em, % 
	rightmargin=0pt, %
	innerrightmargin=0pt, % % 
	%splittopskip=\topskip, %
	%splitbottomskip=\topskip, %
}% 

\mdfdefinestyle{nprop-frame}{
	linewidth=2pt, %
	linecolor= 300, %
	%linecolor= gray, 
	topline=false, %
	bottomline=false, %
	rightline=false,%
	leftmargin=0pt, %
	innerleftmargin=1em, %
	innerrightmargin=0em, 
	rightmargin=0pt, %
	%splittopskip=\topskip, %
	%splitbottomskip=\topskip, %
}%       

\mdfdefinestyle{ndef-frame}{
	linewidth=2pt, %
	linecolor= 300, % 
	%linecolor= gray!50,
	backgroundcolor= 50,
	%backgroundcolor= gray!5,
	topline=false, %
	bottomline=false, %
	rightline=false,%
	leftmargin=0pt, %
	innerleftmargin=1em, %
	innerrightmargin=1em, 
	rightmargin=0pt, % 
	innertopmargin=1.5em,%
	innerbottommargin=1em, % 
	splittopskip=\topskip, %
	%splitbottomskip=\topskip, %
}% 

\mdfdefinestyle{ejemplo-frame}{
	linewidth=0pt, %
	linecolor= 300, % 
	%backgroundcolor= 50,
	leftline=false, %
	rightline=false, %
	leftmargin=0pt, %
	innerleftmargin=1.5em, %
	innerrightmargin=1.5em, 
	rightmargin=0pt, % 
	innertopmargin=0em,%
	innerbottommargin=0em, % 
	splittopskip=\topskip, %
	%splitbottomskip=\topskip, %
}%                

% Asignamos los marcos a los estilos

\surroundwithmdframed[style=nth-frame]{nth}
\surroundwithmdframed[style=nprop-frame]{nprop}
\surroundwithmdframed[style=nprop-frame]{ncor}
\surroundwithmdframed[style=ndef-frame]{ndef}
\surroundwithmdframed[style=ejemplo-frame]{ejemplo}
                 
% Asignamos los estilos definidos anteriormente a los entornos correspondientes
% Teoremas, proposiciones y corolarios.
\theoremstyle{theorem-style}
\newtheorem{nth}{Teorema}[section]
\newtheorem{nprop}{Proposición}[section]
\newtheorem{ncor}{Corolario}[section]
\newtheorem{lema}{Lema}[section]
% Definiciones.
\theoremstyle{definition-style}
\newtheorem{ndef}{Definición}[section]
\newtheorem{ejer}{Ejercicio}[section]

% Notas.
\theoremstyle{remark-style}
\newtheorem*{nota}{Nota}
\newtheorem*{sol}{Solución}


% Ejemplos.
\theoremstyle{example-style}
\newtheorem{ejemplo}{Ejemplo}[section]



% Listas ordenadas con números romanos (i), (ii), etc.
\newenvironment{nlist}
{\begin{enumerate}
    \renewcommand\labelenumi{(\emph{\roman{enumi})}}}
  {\end{enumerate}}

% División por casos con llave a la derecha.
\newenvironment{rcases}
{\left.\begin{aligned}}
    {\end{aligned}\right\rbrace}


% ---------------------------------------------------------------------------
% PÁGINA DE TÍTULO
% ---------------------------------------------------------------------------

% Título del documento.
\newcommand{\subject}{Cálculo II: ejercicios}

% Autor del documento.
\newcommand{\docauthor}{Miguel Anguita Ruiz}

% Título
\title{
  \normalfont \normalsize 
  \textsc{Universidad de Granada} \\ [25pt]    % Texto por encima.
  \textsc{Grado en Estadística} \\ [25pt]    % Texto por encima.
  \horrule{0.5pt} \\[0.4cm] % Línea horizontal fina.
  \huge \sffamily\subject\\ % Título.
  \horrule{2pt} \\[0.5cm] % Línea horizontal gruesa.
}

% Autor.
\author{\Large\sffamily{\docauthor}}

% Fecha.
\date{\vspace{-1.5em} \normalsize \sffamily Curso 2017/18}



% ---------------------------------------------------------------------------
% COMIENZO DEL DOCUMENTO
% ---------------------------------------------------------------------------

\begin{document}

\maketitle  % Título.
\vfill
\begin{center}
  %\includegraphics{by-nc-sa.pdf}  % Licencia.
\end{center}
\newpage
\tableofcontents    % Índice
\newpage


% --------------------------------------------------------------------------------
% Introducción.
% --------------------------------------------------------------------------------

\section{Relación 1}

\begin{ejer}
Asociado al experimento aleatorio del lanzamiento de dos monedas y la observación de sus resultados, describir el vector aleatorio $X = (X_1,X_2)$ con: \\
$X_1$ = Número de rachas de resultados iguales. \\
$X_2$ = Diferencia en valor absoluto entre número de caras y cruces. \\
especificando los elementos del espacio probabilístico $(\Omega,A,P)$. \\
Calcular los conjuntos $X^{-1}((-\infty,x_1]x(-\infty,x_2]) \ \forall x_1,x_2 \in R$ \\
Calcular la función de distribución de dicho vector aleatorio.
\end{ejer}

\begin{sol}.
\\ El espacio de probabilidad $(\Omega,A,P)$ que definiremos es el siguiente: \\
$\Omega = \{CC,CX,XC,XX\}$ tal que C = sale cara, X = sale cruz.\\
$A = P(\Omega)$ y $P(\omega) = \frac{1}{4} \ \forall \omega \in \Omega$ \\

Definimos el vector aleatorio $X=(X_1,X_2)$ de esta forma:

\begin{equation}
X_1 = \left\lbrace
\begin{array}{ll}
0 \ : \omega \in \{CX,XC\} \\
1 \ : \omega \in \{CC,XX\}
\end{array}
\right.
\end{equation}

\begin{equation}
X_2 = \left\lbrace
\begin{array}{ll}
0 \ : \omega \in \{CX,XC\} \\
2 \ : \omega \in \{CC,XX\}
\end{array}
\right.
\end{equation}

Por tanto: $X = (X_1,X_2): \Omega \to \R^2$ \\
CC $ \to (1,2)$ \\
XX $ \to (1,2)$ \\
CX $ \to (0,0)$ \\
XC $ \to (0,0)$ \\

Por otra parte, los conjuntos $X^{-1}((-\infty,x_1]x(-\infty,x_2]) \ \forall x_1,x_2 \in R$ son los siguientes: \\
$$X^{-1}((-\infty,x]x(-\infty,y]) = \emptyset \in A \ \forall x,y < 0$$

$$X^{-1}((-\infty,0]x(-\infty,2]) = X^{-1}((-\infty,1]x(-\infty,0]) = \{CX,XC\} \in A $$

$$X^{-1}((-\infty,1]x(-\infty,2]) = \Omega \in A $$ \\

Por último, la función de distribución es la siguiente:

\begin{equation}
F(x,y) = \left\lbrace
\begin{array}{ll}
0 : x < 0 \lor y < 0\\
\frac{1}{2} : x \ge 1 \land 0 \le y < 2\\
\frac{1}{2} : 0 \le x < 1 \land y \ge 2 \\
1 : x \ge 1 \land y \ge 2 \\
\end{array}
\right.
\end{equation}

\end{sol}

\begin{ejer}
Definir un experimento aleatorio cualquiera, especificando los elementos del espacio probabilístico $(\Omega,A,P)$,
y describir adecuadamente sobre él un vector aleatorio discreto $X = (X_1,X_2,...,X_n)$ (con $n$ siendo una
cifra a tu elección) que mida una serie de características del experimento. Asimismo, calcular la función de distribución de dicho vector aleatorio.
\end{ejer}

\begin{sol}.
	
	El experimento consiste en lo siguiente. Tenemos dos urnas llenas de bolas, en una urna hay 2 bolas: una blanca y una roja no numeradas. En la otra urna hay 3 bolas numeradas del 1 al 3 (color no importa). Cojo 2 bolas con reemplazamiento, una de cada urna. \\
	
	 El espacio de probabilidad $(\Omega,A,P)$ que definiremos es el siguiente: \\
	$\Omega = \{1B,2B,3B,1R,2R,3R\}$ tal que iB = cojo bola blanca y con número i, iR = cojo bola roja y con número i para i $ = 1,2,3$\\
	$A = P(\Omega)$ y $P(\omega) = \frac{1}{6} \ \forall \omega \in \Omega$ \\
	
	Definimos el vector aleatorio $X=(X_1,X_2)$ de esta forma:
	
	\begin{equation}
	X_1 = \left\lbrace
	\begin{array}{ll}
	1 \ : \omega \in \{1B,1R\} \\
	2 \ : \omega \in \{2B,2R\} \\
	3 \ : \omega \in \{3B,3R\}
	\end{array}
	\right.
	\end{equation}
	
	\begin{equation}
	X_2 = \left\lbrace
	\begin{array}{ll}
	e \ : \omega \in \{1B,2B,3B\} \\
	\pi \ : \omega \in \{1R,2R,3R\}
	\end{array}
	\right.
	\end{equation}
	
	Por tanto: $X = (X_1,X_2): \Omega \to \R^2$ \\
	1B $ \to (1,e)$ \\
	1R $ \to (1,\pi)$ \\
	2B $ \to (2,e)$ \\
	2R $ \to (2,\pi)$ \\
	3B $ \to (3,e)$ \\
	3R $ \to (3,\pi)$ \\
	
	Por otra parte, los conjuntos
	
	\begin{equation}
	X^{-1}((-\infty,x_1]x(-\infty,x_2]) = \left\lbrace
	\begin{array}{ll}
	\varnothing : x_1 < 1 \lor x_2 < e \\
	\{1B\} : 1 \le x_1 < 2, e \le x_2 < \pi \\
	\{1B,1R\} : 1 \le x_1 < 2, x_2 \ge \pi \\
	\{1B,2B\} : 2 \le x_1 < 3, e \le x_2 < \pi \\
	\{1B,2B,3B\} : x_1 \ge 3, e \le x_2 < \pi \\
	\{1B,1R,2B,2R\} : 2 \le x_1 < 3, x_2 \ge \pi \\
	\Omega : x_1 \ge 3, x_2 \ge \pi
	\end{array}
	\right.
	\end{equation} 
	
	se definen de esta manera.
	
	Por último, la función de distribución es la siguiente:
	
	\begin{equation}
	P(X^{-1}((-\infty,x_1]x(-\infty,x_2])) = \left\lbrace
	\begin{array}{ll}
	P({\varnothing}) = 0 : x_1 < 1 \lor x_2 < e \\
	P(\{1B\}) = \frac{1}{6} : 1 \le x_1 < 2, e \le x_2 < \pi \\
	P(\{1B,1R\})= \frac{1}{3} : 1 \le x_1 < 2, x_2 \ge \pi \\
	P(\{1B,2B\})= \frac{1}{3} : 2 \le x_1 < 3, e \le x_2 < \pi \\
	P(\{1B,2B,3B\})= \frac{1}{2} : x_1 \ge 3, e \le x_2 < \pi \\
	P(\{1B,1R,2B,2R\})= \frac{2}{3} : 2 \le x_1 < 3, x_2 \ge \pi \\
	P(\Omega)=1 : x_1 \ge 3, x_2 \ge \pi
	\end{array}
	\right.
	\end{equation} 
	
\end{sol}

\begin{ejer}
Dar la expresión de las siguientes probabilidades en términos de la función de distribución para un vector aleatorio $X=(X_1,X_2)$.
\end{ejer}

\begin{sol}.
\\ $P(a < X_1 < b, c < X_2 < d) = F(b^{-},d^{-})-F(b^{-},c)-F(a,d^{-})+F(a,c)$ \\
$P(a \le X_1 < b, c < X_2 < d) = F(b^{-},d^{-})-F(b^{-},c)-F(a^{-},d^{-})+F(a^{-},c)$ \\
$P(a < X_1 \le b, c < X_2 < d) = F(b,d^{-})-F(b,c)-F(a,d^{-})+F(a,c)$ \\
$P(a \le X_1 \le b, c < X_2 < d) = F(b,d^{-})-F(b,c)-F(a^{-},d^{-})+F(a^{-},c)$ \\
$P(a < X_1 < b, c \le X_2 < d) = F(b^{-},d^{-})-F(b^{-},c^{-})-F(a,d^{-})+F(a,c^{-})$ \\
$P(a \le X_1 < b, c \le X_2 < d) = F(b^{-},d^{-})-F(b^{-},c^{-})-F(a^{-},d^{-})+F(a^{-},c^{-})$ \\
$P(a < X_1 \le b, c \le X_2 < d) = F(b,d^{-})-F(b,c^{-})-F(a,d^{-})+F(a,c^{-})$ \\
$P(a \le X_1 \le b, c \le X_2 < d) = F(b,d^{-})-F(b,c^{-})-F(a^{-},d^{-})+F(a^{-},c^{-})$ \\
$P(a < X_1 < b, c < X_2 \le d) = F(b^{-},d)-F(b^{-},c)-F(a,d)+F(a,c)$ \\
$P(a \le X_1 < b, c < X_2 \le d) = F(b^{-},d)-F(b^{-},c)-F(a^{-},d)+F(a^{-},c)$ \\
$P(a < X_1 \le b, c < X_2 \le d) = F(b,d)-F(b,c)-F(a,d)+F(a,c)$ \\
$P(a \le X_1 \le b, c < X_2 \le d) = F(b,d)-F(b,c)-F(a^{-},d)+F(a^{-},c)$ \\
$P(a < X_1 < b, c \le X_2 \le d) = F(b^{-},d)-F(b^{-},c^{-})-F(a,d)+F(a,c^{-})$ \\
$P(a \le X_1 < b, c \le X_2 \le d) = F(b^{-},d)-F(b^{-},c^{-})-F(a^{-},d)+F(a^{-},c^{-})$ \\
$P(a < X_1 \le b, c \le X_2 \le d) = F(b,d)-F(b,c^{-})-F(a,d)+F(a,c^{-})$ \\
$P(a \le X_1 \le b, c \le X_2 \le d) = F(b,d)-F(b,c^{-})-F(a^{-},d)+F(a^{-},c^{-})$ \\

\end{sol}

















% FIN DEL DOCUMENTO
% ---------------------------------------------------------------------------

\end{document}