%%%%%%%%%%%%%%%%%%%%%%%%%%%%%%%%%%%%%%%%%%%%%%%%%%%%%%%%%%%%%%%%%%%%%%%%% 
% 
% Apuntes de la asignatura Análisis Matemático II.
% Grado en Estadística.
% Universidad de Granada.
% Curso 2017/18.
% 
% 
% Colaboradores:
% Miguel Anguita Ruiz
% 
% Agradecimientos:
% Andrés Herrera (@andreshp) y Mario Román (@M42) por
% las plantillas base.
% 
% Sitio original:
% https://github.com/libreim/apuntesDGIIM/
% 
% Licencia:
% CC BY-NC-SA 4.0 (https://creativecommons.org/licenses/by-nc-sa/4.0/)
% 
%%%%%%%%%%%%%%%%%%%%%%%%%%%%%%%%%%%%%%%%%%%%%%%%%%%%%%%%%%%%%%%%%%%%%%%%% 


% ------------------------------------------------------------------------------
% ACKNOWLEDGMENTS
% ------------------------------------------------------------------------------

%%%%%%%%%%%%%%%%%%%%%%%%%%%%%%%%%%%%%%%%%%%%%%%%%%%%%%%%%%%%%%%%%%%%%%%% 
% Plantilla básica de Latex en Español.
% 
% Autor: Andrés Herrera Poyatos (https://github.com/andreshp) 
% 
% Es una plantilla básica para redactar documentos. Utiliza el paquete fancyhdr
% para darle un estilo moderno pero serio.
% 
% La plantilla se encuentra adaptada al español.
%
%%%%%%%%%%%%%%%%%%%%%%%%%%%%%%%%%%%%%%%%%%%%%%%%%%%%%%%%%%%%%%%%%%%%%%%%% 

%%%%%%%%%%%%%%%%%%%%%%%%%%%%%%%%%%%%%%%%%%%%%%%%%%%%%%%%%%%%%%%%%%%%%%%%% 
% Plantilla de Trabajo
% Modificación de una plantilla de Latex de Frits Wenneker para adaptarla 
% al castellano y a las necesidades de escribir informática y matemáticas.
% 
% Editada por: Mario Román
% 
% License:
% CC BY-NC-SA 3.0 (http://creativecommons.org/licenses/by-nc-sa/3.0/)
%%%%%%%%%%%%%%%%%%%%%%%%%%%%%%%%%%%%%%%%%%%%%%%%%%%%%%%%%%%%%%%%%%%%%%%%% 

%%%%%%%%%%%%%%%%%%%%%%%%%%%%%%%%%%%%%%%%%%%%%%%%%%%%%%%%%%%%%%%%%%%%%%%%% 
% Short Sectioned Assignment
% LaTeX Template
% Version 1.0 (5/5/12)
% 
% This template has been downloaded from:
% http://www.LaTeXTemplates.com
% 
% Original author:
% Frits Wenneker (http://www.howtotex.com)
% 
% License:
% CC BY-NC-SA 3.0 (http://creativecommons.org/licenses/by-nc-sa/3.0/)
% 
%%%%%%%%%%%%%%%%%%%%%%%%%%%%%%%%%%%%%%%%%%%%%%%%%%%%%%%%%%%%%%%%%%%%%%%%% 


% Tipo de documento y opciones.
%\documentclass[11pt, a4paper, twoside]{article} % Usar para imprimir
\documentclass[11pt, a4paper]{article}

% ---------------------------------------------------------------------------
% PAQUETES
% ---------------------------------------------------------------------------

% Idioma y codificación para Español.
\usepackage[utf8]{inputenc}
\usepackage[spanish, es-tabla, es-lcroman, es-noquoting]{babel}
\selectlanguage{spanish} 
% \usepackage[T1]{fontenc}

% Fuente utilizada.
\usepackage{courier}    % Fuente Courier.
\usepackage{microtype}  % Mejora la letra final de cara al lector.

% Diseño de página.
\usepackage{fancyhdr}   % Utilizado para hacer cabeceras y pies de página.

\usepackage{titlesec} 	% Utilizado para hacer títulos propios.
\usepackage{lastpage}   % Referencia a la última página.
\usepackage{extramarks} % Marcas extras. Utilizado en pie de página y cabecera.
\usepackage[parfill]{parskip}    % Crea una nueva línea entre párrafos.
\usepackage{geometry}            % Geometría de las páginas.

% Símbolos y matemáticas.
\usepackage{amssymb, amsmath, amsthm, amsfonts, amscd}
\usepackage{upgreek}
\usepackage{mathrsfs}

\usepackage{mdframed}

% Gráficos

\usepackage{pgf,tikz}
\usepackage{tkz-euclide}
\usetkzobj{all}

% Otros.
\usepackage{enumitem}   % Listas mejoradas.
\usepackage{hyperref}
\usepackage{graphicx}   % Gráficos.
\usepackage[space]{grffile}  % Permitir espacios en rutas de gráficos
\usepackage{xcolor}     % Colores.8

% Fuentes personalizadas

\usepackage[scaled=.85]{newpxtext,newpxmath}
\usepackage[scaled=.85]{FiraSans}
\usepackage[T1]{fontenc}

% Código para ajustar las fuentes matemáticas al estilo del texto
% que le rodea

\DeclareMathVersion{sans}
\SetSymbolFont{operators}{sans}{OT1}{cmbr}{m}{n}
\SetSymbolFont{letters}{sans}{OML}{cmbr}{m}{it}
\SetSymbolFont{symbols}{sans}{OMS}{cmbrs}{m}{n}
\SetMathAlphabet{\mathit}{sans}{OT1}{cmbr}{m}{sl}
\SetMathAlphabet{\mathbf}{sans}{OT1}{cmbr}{bx}{n}
\SetMathAlphabet{\mathtt}{sans}{OT1}{cmtl}{m}{n}
\SetSymbolFont{largesymbols}{sans}{OMX}{iwona}{m}{n}

\DeclareMathVersion{boldsans}
\SetSymbolFont{operators}{boldsans}{OT1}{cmbr}{b}{n}
\SetSymbolFont{letters}{boldsans}{OML}{cmbrm}{b}{it}
\SetSymbolFont{symbols}{boldsans}{OMS}{cmbrs}{b}{n}
\SetMathAlphabet{\mathit}{boldsans}{OT1}{cmbr}{b}{sl}
\SetMathAlphabet{\mathbf}{boldsans}{OT1}{cmbr}{bx}{n}
\SetMathAlphabet{\mathtt}{boldsans}{OT1}{cmtl}{b}{n}
\SetSymbolFont{largesymbols}{boldsans}{OMX}{iwona}{bx}{n}

\newif\IfInSansMode
\let\oldsf\sffamily
\renewcommand*{\sffamily}{\oldsf\mathversion{sans}\InSansModetrue}
\let\oldmd\mdseries
\renewcommand*{\mdseries}{\oldmd\IfInSansMode\mathversion{sans}\fi\relax}
\let\oldbf\bfseries
\renewcommand*{\bfseries}{\oldbf\IfInSansMode\mathversion{boldsans}\else%
   \mathversion{bold}\fi\relax}
\let\oldnorm\normalfont
\renewcommand*{\normalfont}{\oldnorm\InSansModefalse\mathversion{normal}}
\let\oldrm\rmfamily
\renewcommand*{\rmfamily}{\oldrm\InSansModefalse\mathversion{normal}}

% Colores

\definecolor{50}{HTML}{E8F5E9}
\definecolor{300}{HTML}{81C784}
\definecolor{500}{HTML}{4CAF50}
\definecolor{700}{HTML}{388E3C}

% ---------------------------------------------------------------------------
% OPCIONES PERSONALIZADAS
% ---------------------------------------------------------------------------

% Formato de texto.
\linespread{1.3}            % Espaciado entre líneas.
\setlength\parindent{0pt}   % No indentar el texto por defecto.
\setlist{leftmargin=.5in}   % Indentación para las listas.

% Estilo de página.
\pagestyle{fancy}
\fancyhf{}
\geometry{left=3cm,right=3cm,top=3cm,bottom=3cm}   % Márgenes y cabecera.

% Estilo de las cabeceras

\titleformat{\section}
  {\Large\bfseries\sffamily}{\thesection}{1em}{}
\titleformat{\subsection}
  {\large\sffamily}{\thesubsection}{1em}{}
\titleformat{\subsubsection}
  {\sffamily}{\thesubsubsection}{1em}{}

% Estilo de enlaces
\hypersetup{
  % hidelinks = true,   % Oculta todos los enlaces.
  colorlinks = true,   % Muestra todos los enlaces, sin bordes alrededor.
  linkcolor=black,     % Color de enlaces genéricos
  citecolor={blue!50!black},   % Color de enlaces de referencias
  urlcolor={blue!80!black}     % Color de enlaces de URL
}

% Ruta donde buscar gráficos
\graphicspath{{../Recursos/Plantillas/} {Recursos/Plantillas/} {./img/} {Análisis Matemático II/img/}}

% Redefinir entorno de demostración (reducir espacio superior)
% \makeatletter
% \renewenvironment{proof}[1][\proofname] {\vspace{-15pt}\par\pushQED{\qed}\normalfont\topsep6\p@\@plus6\p@\relax\trivlist\item[\hskip\labelsep\it#1\@addpunct{.}]\ignorespaces}{\popQED\endtrivlist\@endpefalse}
% \makeatother

% Aumentar el tamaño del interlineado
\linespread{1.3}

% Permitir salto de página en ecuaciones
\allowdisplaybreaks


% ---------------------------------------------------------------------------
% COMANDOS PERSONALIZADOS
% ---------------------------------------------------------------------------

% Redefinir letra griega épsilon.
\let\epsilon\upvarepsilon

% Valor absoluto: \abs{}
\providecommand{\abs}[1]{\lvert#1\rvert}    

% Fracción grande: \ddfrac{}{}
\newcommand\ddfrac[2]{\frac{\displaystyle #1}{\displaystyle #2}}

% Texto en negrita en modo matemática: \bm{}
\newcommand{\bm}[1]{\boldsymbol{#1}}

% Línea horizontal.
\newcommand{\horrule}[1]{\rule{\linewidth}{#1}}

% Letras de conjuntos
\newcommand{\R}{\mathbb{R}} \newcommand{\N}{\mathbb{N}}

% Sucesiones
\newcommand{\xn}{\{x_n\}}
\newcommand{\fn}{\{f_n\}}

% Letra griega "chi" en línea con el texto
\DeclareRobustCommand{\rchi}{{\Large \mathpalette\irchi\relax}}
\newcommand{\irchi}[2]{\raisebox{0.4\depth}{$#1\chi$}} % inner command, used by \rchi 

% Letra 'omega'
\newcommand{\W}{\Omega}
\newcommand{\w}{\omega}


% ---------------------------------------------------------------------------
% CABECERA Y PIE DE PÁGINA
% ---------------------------------------------------------------------------

\renewcommand{\sectionmark}[1]{%
\markboth{#1}{}}

\renewcommand{\subsectionmark}[1]{%
\markright{#1}{}}

%\addtolength{\headheight}{4ex}
\renewcommand{\headrulewidth}{0pt}

\addtolength{\headwidth}{\marginparsep}
%\addtolength{\headwidth}{\marginparwidth}

\fancypagestyle{section}{%
  \fancyhead{}
  %\addtolength{\headheight}{-10ex}
  %\renewcommand{\headheight}{0pt}%
  %\setlength{\footskip}{-48pt}%
  \fancyfoot[LE,RO]{\Large\sffamily\thepage}
  \renewcommand{\headrulewidth}{0pt}%
  \renewcommand{\footrulewidth}{0pt}%
}

\let\originalsection\section
\RenewDocumentCommand{\section}{som}{%
  \IfBooleanTF{#1}
    {\originalsection*{#3}}
    {\IfNoValueTF{#2}
      {\originalsection{#3}}
      {\originalsection[#2]{#3}}%
    }%
  \thispagestyle{section}%
}

\fancyhead[LE,RO]{\rule[-4ex]{0pt}{2ex}\sffamily\textsl{\rightmark}}
\fancyhead[LO,RE]{\sffamily{\leftmark}}
\fancyfoot[LE,RO]{\Large\sffamily\thepage}



% ---------------------------------------------------------------------------
% ENTORNOS PARA MATEMÁTICAS
% ---------------------------------------------------------------------------

% Nuevo estilo para definiciones.
\newtheoremstyle{definition-style} % Nombre del estilo.
{}               % Espacio por encima.
{}               % Espacio por debajo.
{}                   % Fuente del cuerpo.
{}                   % Identación.
{\bf\sffamily}                % Fuente para la cabecera.
{.}                  % Puntuación tras la cabecera.
{.5em}               % Espacio tras la cabecera.
{\thmname{#1}\thmnumber{ #2}\thmnote{ (#3)}}     % Especificación de la cabecera (actual: nombre en negrita).

% Nuevo estilo para notas.
\newtheoremstyle{remark-style} 
{10pt}                
{10pt}                
{}                   
{}                   
{\itshape \sffamily}          
{.}                  
{.5em}               
{}                  

% Nuevo estilo para teoremas y proposiciones.
\newtheoremstyle{theorem-style}
{}                
{}                
{}           
{}                  
{\bfseries \sffamily}             
{.}                
{.5em}           
{\thmname{#1}\thmnumber{ #2}\thmnote{ (#3)}}

% Nuevo estilo para ejemplos.
\newtheoremstyle{example-style}
{10pt}                
{10pt}                
{}                  
{}                   
{\bf \sffamily}              
{}                 
{.5em}               
{\thmname{#1}\thmnumber{ #2.}\thmnote{ #3.}}  

% Nuevo estilo para la demostración
%\renewenvironment{proof}{{\itshape \sffamily Demostración \\}}{\hspace*{\fill}\qed}

\makeatletter
\renewenvironment{proof}[1][\proofname] {\par\pushQED{\qed}\normalfont\topsep6\p@\@plus6\p@\relax\trivlist\item[\hskip\labelsep\itshape\sffamily#1\@addpunct{.}]\ignorespaces}{\popQED\endtrivlist\@endpefalse}
\makeatother

% Configuración general de mdframe, los estilos de los teoremas, etc
\mdfsetup{skipabove=12pt,skipbelow=12pt,innertopmargin=12pt,innerbottommargin=4pt}

% Creamos los 'marcos' de los estilos

\mdfdefinestyle{nth-frame}{
	linewidth=2pt, %
	linecolor= 500, % 
	%linecolor=black,
	topline=false, %
	bottomline=false, %
	rightline=false,%
	leftmargin=0pt, %
	innerleftmargin=1em, % 
	rightmargin=0pt, %
	innerrightmargin=0pt, % % 
	%splittopskip=\topskip, %
	%splitbottomskip=\topskip, %
}% 

\mdfdefinestyle{nprop-frame}{
	linewidth=2pt, %
	linecolor= 300, %
	%linecolor= gray, 
	topline=false, %
	bottomline=false, %
	rightline=false,%
	leftmargin=0pt, %
	innerleftmargin=1em, %
	innerrightmargin=0em, 
	rightmargin=0pt, %
	%splittopskip=\topskip, %
	%splitbottomskip=\topskip, %
}%       

\mdfdefinestyle{ndef-frame}{
	linewidth=2pt, %
	linecolor= 300, % 
	%linecolor= gray!50,
	backgroundcolor= 50,
	%backgroundcolor= gray!5,
	topline=false, %
	bottomline=false, %
	rightline=false,%
	leftmargin=0pt, %
	innerleftmargin=1em, %
	innerrightmargin=1em, 
	rightmargin=0pt, % 
	innertopmargin=1.5em,%
	innerbottommargin=1em, % 
	splittopskip=\topskip, %
	%splitbottomskip=\topskip, %
}% 

\mdfdefinestyle{ejemplo-frame}{
	linewidth=0pt, %
	linecolor= 300, % 
	%backgroundcolor= 50,
	leftline=false, %
	rightline=false, %
	leftmargin=0pt, %
	innerleftmargin=1.5em, %
	innerrightmargin=1.5em, 
	rightmargin=0pt, % 
	innertopmargin=0em,%
	innerbottommargin=0em, % 
	splittopskip=\topskip, %
	%splitbottomskip=\topskip, %
}%                

% Asignamos los marcos a los estilos

\surroundwithmdframed[style=nth-frame]{nth}
\surroundwithmdframed[style=nprop-frame]{nprop}
\surroundwithmdframed[style=nprop-frame]{ncor}
\surroundwithmdframed[style=ndef-frame]{ndef}
\surroundwithmdframed[style=ejemplo-frame]{ejemplo}
                 
% Asignamos los estilos definidos anteriormente a los entornos correspondientes
% Teoremas, proposiciones y corolarios.
\theoremstyle{theorem-style}
\newtheorem{nth}{Teorema}[section]
\newtheorem{nprop}{Proposición}[section]
\newtheorem{ncor}{Corolario}[section]
\newtheorem{lema}{Lema}[section]
% Definiciones.
\theoremstyle{definition-style}
\newtheorem{ndef}{Definición}[section]
\newtheorem{ejer}{Ejercicio}[section]

% Notas.
\theoremstyle{remark-style}
\newtheorem*{nota}{Nota}
\newtheorem*{sol}{Solución}

% Ejemplos.
\theoremstyle{example-style}
\newtheorem{ejemplo}{Ejemplo}[section]



% Listas ordenadas con números romanos (i), (ii), etc.
\newenvironment{nlist}
{\begin{enumerate}
    \renewcommand\labelenumi{(\emph{\roman{enumi})}}}
  {\end{enumerate}}

% División por casos con llave a la derecha.
\newenvironment{rcases}
{\left.\begin{aligned}}
    {\end{aligned}\right\rbrace}


% ---------------------------------------------------------------------------
% PÁGINA DE TÍTULO
% ---------------------------------------------------------------------------

% Título del documento.
\newcommand{\subject}{Análisis Matemático II}

% Autor del documento.
\newcommand{\docauthor}{Grado en Estadística}

% Título
\title{
  \normalfont \normalsize 
  \textsc{Universidad de Granada} \\ [25pt]    % Texto por encima.
  \horrule{0.5pt} \\[0.4cm] % Línea horizontal fina.
  \huge \sffamily\subject\\ % Título.
  \horrule{2pt} \\[0.5cm] % Línea horizontal gruesa.
}

% Autor.
\author{\Large\sffamily{\docauthor}}

% Fecha.
\date{\vspace{-1.5em} \normalsize \sffamily Curso 2017/18}



% ---------------------------------------------------------------------------
% COMIENZO DEL DOCUMENTO
% ---------------------------------------------------------------------------

\begin{document}

\maketitle  % Título.
\vfill
\begin{center}
  %\includegraphics{by-nc-sa.pdf}  % Licencia.
\end{center}
\newpage
\tableofcontents    % Índice
\newpage


% --------------------------------------------------------------------------------
% Introducción.
% --------------------------------------------------------------------------------

\section{El espacio euclídeo}
Como punto de partida para el estudio de las funciones de varias variables reales, debemos
familiarizarnos con la estructura y propiedades del espacio en el que dichas funciones tendrán
su conjunto de definición, el espacio euclídeo n-dimensional, donde $n$ es un número natural.
Al tiempo que estudiamos algunas propiedades de dicho espacio, las iremos abstrayendo, para
entender ciertos conceptos generales que son importantes en Análisis Matemático.
Partimos de la definición de $\R^n$ y su estructura algebraica básica, la de espacio vectorial.
Al estudiar el producto escalar en $\R^n$ , completamos la definición del espacio euclídeo, así
llamado porque formaliza analíticamente los axiomas y resultados de la geometría de Euclides.

\begin{ndef}[Espacio euclídeo]
	Definimos el espacio euclídeo n-dimensional como el producto cartesiano de n copias de $\R$, es decir, el conjunto de
	todas las posibles n-uplas de números reales: $$\R^n = \R  \ \ x \ \ \R \ \ x \ \ {...}^{(n)} \ \ x \ \ \R = \{(x_1,...,x_n): x_i \in R, \forall i \in {1...n}\}$$
\end{ndef}

Sin embargo, no siempre es conveniente usar subíndices para denotar las componentes de los
elementos de $\R^n$, pues podemos necesitar los subíndices para otra finalidad. Para valores concretos de n, podemos denotar las componentes con
letras diferentes, siendo habitual escribir:

$$\R^2 = \{(x,y): x,y \in R\}$$
$$\R^3 = \{(x,y,z): x,y,z \in R\}$$

En $\R^n$ disponemos de las operaciones de suma y producto por escalares, definidas, para $x = (x_1,...,x_n) \in \R^n, y = (y_1,...,y_n) \in \R^n$ y $\lambda \in \R$, por $$x + y = (x_1+y_1,...,x_n+y_n)$$ $$\lambda x = (\lambda x_1,...,\lambda x_n)$$

\begin{nprop}
Sea $n \in \N, x,y,z \in \R^n, \alpha, \beta \in \R$. Entonces:

$a) \ (x+y)+z = x + (y+z)$ (Propiedad asociativa) \\
$b) \ 0 = (0,...,0) \in \R^n \implies x+0 = 0+x = x$ \\
$c)$ \ Dado $x = (x_1,...,x_n) \implies \exists ! v \in \R^n: x+v=v+x=0 \implies v = (-x_1,...,-x_n) = -x$ \\
$d) \ x+y = y+x$ \\
$e) \ 1*x = x$ \\
$f) \ (\lambda + \beta)x = \lambda x + \beta x$ \\
$g) \ \lambda (x+y) = \lambda x + \lambda y$ \\
$h) \ (\lambda \beta)y = \lambda (\beta y)$
	
\end{nprop}

\newpage

\begin{ndef}[Producto escalar]
Sea $n \in \N$. Definimos el producto escalar de $x = (x_1,...,x_n) \in \R^n, y = (y_1,...,y_n) \in \R^n$ como $$ <x,y> = \sum_{i=1}^{n}x_i y_i$$
\end{ndef}

\begin{nprop}
Sea $n \in \N$:

$a) \ x,y,z \in \R^n \implies <x+y,z> = <x,z> + <y,z> \\
b) \ x,y \in \R^n, \lambda \in \R \implies <\lambda x,y> = \lambda <x,y> \\
c) \ <x,y> = <y,x> \ \forall x,y \in \R^n $
\end{nprop}

\begin{nprop}[Desigualdad de Cauchy-Schwartz]
Sean $x = (x_1,...,x_n) \in \R^n, y = (y_1,...,y_n) \in \R^n$, entonces: $$(<x,y>)^2 \le (\sum_{i=1}^{n}x_i^2)(\sum_{i=1}^n y_i^2)$$ 
\end{nprop}

\begin{proof} \hfill \\
 Para $x = (x_1,...,x_n) \in \R^n, y = (y_1,...,y_n)$, es cierto que $0 \le \sum_{i=1}^n (ax_i+y_i)^2 = \sum_{i=1}^n (a^2x_i^2+y_i^2+2ax_iy_i) = a^2(\sum_{i=1}^n x_i^2)+(\sum_{i=1}^n y_i^2)+2a (\sum_{i=1}^n x_iy_i)$ para todo número real y es igualdad si, y sólo si, cada término de la suma es cero. Esta desigualdad puede escribirse en la forma: $Ax^2 + Bx + C$ donde $A = \sum_{i=1}^n x_i^2, B = \sum_{i=1}^n x_iy_i, C = \sum_{i=1}^n y_i^2$. En particular, la desigualdad se cumple para $\frac{-B}{2A}$: \ $A(\frac{-B}{A})^2 + 2B(\frac{-B}{A}) + C \ge 0 \implies C \ge \frac{B^2}{2} \implies B^2 \le AC$.
\end{proof}

\begin{ndef}[Norma]
Dado $n \in \N, x = (x_1,...,x_n) \in \R^n$, se define la norma de $x$ como: $$||x|| = \sqrt{\sum_{i=1}^{n}x_i^2} = \sqrt{<x,x>}$$
\end{ndef}

\begin{nprop}
Sea $n \in \N, x,y \in \R^n, \lambda \in \R. \\
a) \ ||x|| \ge 0 \land ||x|| = 0 \Leftrightarrow x = 0 \in \R^n \\
b) \ ||\lambda x|| = |\lambda| ||x|| \\
c) \ ||x+y|| \le ||x|| + ||y||$ (Desigualdad triangular)
\end{nprop}

\begin{proof}
(Desigualdad triangular) \\ $||x+y||^2 = <x+y,x+y> = <x,x> + <x,y> + <y,x> + <y,y> = ||x||^2 + ||y||^2 + 2<x,y> \\ \le ||x||^2 + ||y||^2 + 2||x|| ||y|| = (||x||+||y||)^2$
\end{proof}

\begin{ndef}[Distancia]
Sea $n \in \N, x,y \in \R^n$, se define la distancia de $x$ a $y$ como $$d(x,y)= ||x-y||$$
\end{ndef}

\begin{nprop}
Sea $n \in \N, x,y \in \R^n. \\
a) \ d(x,y) \ge 0 \land d(x,y)=0 \Leftrightarrow x=y \\
b) \ d(x,y)=d(y,x) \\
c) \ d(x,z) \le d(x,y)+d(y,z)$
\end{nprop}

\section{Topología en $\R^n$}

\begin{ndef}[Bola de centro a y radio r]
Sea $n \in \N, a \in \R^n, r > 0.$ Se definen las bolas abiertas, cerradas y esferas, respectivamente, de centro a y radio r como: $$B(a,r) = \{x \in \R^n: d(x,y) < r\} $$ $$\overline{B}(a,r) = \{x \in \R^n: d(x,y) \leq r\} $$ $$S(a,r) = \{x \in \R^n: d(x,y) = r\}$$
\end{ndef}

\begin{ndef}[Sucesión convergente en $\R^n$]
Sea $n \in \N$, y $\{x_n\}$ una sucesión convergente en $\R^n$. Diremos que $\{x_n\}$ es convergente hacia un límite $a \in \R^n$ si $$ \forall \epsilon > 0 \ \exists m \in \N: n \geq m \implies d(x_n,a) < \epsilon$$
\end{ndef}

\begin{nprop}
Sea $n \in \N, x,y \in \R^n, \{x_n\},\{y_n\}$ dos sucesiones  en $\R^n$. \\
$a) \ \{x_n\} \rightarrow x, \{x_n\} \rightarrow y \implies x = y \ $(Unicidad de límites) \\
$b) \ \{x_n\} \rightarrow x, \{y_n\} \rightarrow y \implies \{x_n+y_n\} \rightarrow x+y \\
c) \ \{x_n\} \rightarrow x, \lambda \in \R \implies \{\lambda x_n\} \rightarrow \lambda x \\
d) \ \{x_n\} $ converge $ \implies \{x_n\} $ acotada. $(\exists M \in \N: ||x_n|| \le M \ \forall n \in N$) (Acotación)\\
$e) \ \{x_n\} \rightarrow x, \{y_n\} \rightarrow y \implies \{<x_n,y_n>\} \rightarrow <x,y> \\
f) \ \{x_n\} \rightarrow x \implies \{||x_n||\} \rightarrow ||x||$
\end{nprop}

\begin{proof}
(Unicidad de límites) \\ Supongamos $\{x_n\} \rightarrow x, \{x_n\} \rightarrow y$. Fijo $\epsilon > 0$ y demuestro $||x-y|| < \epsilon \implies ||x-y|| = 0 \implies x=y$. \\
Dado $\epsilon > 0, \{x_n\} \rightarrow x \implies \exists m_1 \in \N: ||x_n-x|| < \frac{\epsilon}{2} \ \forall n \ge m_1$ y $\exists m_2 \in \N: ||x_n-y|| < \frac{\epsilon}{2} \ \forall n \ge m_2$. Tomo $n \ge max \{m_1,m_2\} \implies ||x-y|| = ||(x-x_n)+(x_n-y)|| \le ||x_n-x||+||x_n-y|| < \frac{\epsilon}{2}+\frac{\epsilon}{2} = \epsilon \implies x=y$
\end{proof}

\begin{proof}
(Acotación) \\ Supongamos $\{x_n\} \rightarrow x \implies \exists m \in \N: ||x_n-x|| < 1 \ \forall n \ge m \implies ||x_n|| \le 1+||x|| \implies ||x_n|| \in max\{||x_1||,||x_2||,...,||x_m||, 1+||x||=M\}$
\end{proof}

\begin{nprop}
Sea $N \in \N$. Tomo $\{x_n\} \rightarrow x \in \R^N $con $x = (x^1,...,x^N), x_n = (x_n^1,...,x_n^N)$. Entonces $$\{(x_n^1,...,x_n^N)\} \rightarrow (x^1,...,x^N) \Leftrightarrow \{x_n^i\} \rightarrow x^i \ \forall i \in \{1...N\}$$
\end{nprop}

\begin{proof}
 POR HACER
\end{proof}

\begin{ndef}[Interior, adherente y frontera]
Sea $n \in \N, A \subseteq \R^n$. Entonces: \\
a) Un punto $x \in \R^n$ es adherente a $A$ si $ \forall \epsilon > 0, B(x,\epsilon) \cap A \ne \varnothing \implies \overline{A} = \{x_n \in \R^n: x $ adherente a $A$\} y $A$ cerrado si $A = \overline{A}$. \\
$b)$ Un punto $a \in A$ es interior a $A$ si $ \exists r > 0: B(a,r) \subseteq A$. Denotaremos $\mathring{A} = \{a \in A: a $ interior a $A$\} y $A$ abierto si $A = \mathring{A}. \\
c)$ Un punto $x \in \R^n$ es frontera de $A$ si $x \in \overline{A}$ pero $x \not\in \mathring{A}$. Denotaremos $Fr(A) = \{x \in \R^n: x$ es frontera de $A\}$
\end{ndef}

\begin{ejemplo}.
\\ $ a) \ A = [0,1[ \cup \{4\} \implies \overline{A} = [0,1] \cup \{4\}, \ \mathring{A} = ]0,1[, \ Fr(A) = \{0,1,4\} \\
b) \ A = B(a,r) \implies \overline{A} = \overline{B}(a,r), \ \mathring{A} = A, \ Fr(A) = \overline{B}(a,r)- B(a,r)$ DEMOSTRAR
\end{ejemplo}

\begin{nprop}[Caracterización secuencial de la topología]
Sea $n \in \N, A \subseteq \R^n$. Entonces: \\
$a)$ Un punto $x \in \R^n$ es adherente a $A$ si $ \Leftrightarrow \exists \{x_n\} \rightarrow x: \{x_n\} \in A \ \forall n \in \N \\
b) $ Un punto $a \in A$ es interior $ \Leftrightarrow \ \forall \{x_n\} \rightarrow a, x_n \in \R^n \implies \exists m \in \N: \ \forall n \ge m, \ x_n \in A \\
c) $ Un punto $x \in \R^n$ es frontera de $A$ si $ \Leftrightarrow \exists \{x_n\} \rightarrow x: \{x_n\} \in A \ \forall n \in \N \land \{y_n\} \rightarrow x: \{y_n\} \not\in A \ \forall n \in \N $
\end{nprop}

\begin{proof}.
(Apartado a) \\ $ \Rightarrow \\ \forall n \in \N \ \exists x_n \in B(x,\frac{1}{n}) \cap A$ por ser $x$ adherente $ \implies \{x_n\} \subseteq A$. Además, $d(x_n,x) \le \frac{1}{n} \land \{\frac{1}{n}\} \rightarrow 0 \implies \{x_n\} \rightarrow x. $ 

$\Leftarrow \\ $ Sea $ \epsilon > 0, \{x_n\} \rightarrow x \implies \exists n \in \N: d(x_n,x) < \epsilon \implies x_n \in B(x,\epsilon) \ $y$ \ x_n \in A $ \ por hipótesis $ \implies x_n \in B(x,\epsilon) \cap A \ne \varnothing \implies x \in \overline{A}$. \\
\end{proof}

\begin{nprop}
Sea $n \in \N$: \\

$a)$ \ Los conjuntos abiertos cumplen las siguientes propiedades: \\
 \begin{itemize}
 	\item $\varnothing, \R^n$ son abiertos \\
	\item Si $A_1,...,A_n$ abiertos $\implies  \displaystyle\bigcap_{k=1}^{n}{A_k}$ abierto. \\
	\item Si $\{A_{\alpha}: \alpha \in \land\} $ es una familia de abiertos $ \implies \displaystyle\bigcup_{\alpha \in \land}{A_\alpha}$ abierto. \\
 \end{itemize}
$b) $ \ Un conjunto $A \subseteq \R^n $ es abierto $ \Leftrightarrow \R^n/A$ es cerrado. \\
$c) $ \ Los cerrados cumplen: \\
 \begin{itemize} 
\item $\varnothing, \R^n$ son cerrados \\
\item Si $F_1,...,F_n$ cerrados $\implies  \displaystyle\bigcup_{k=1}^{n}{F_k}$ cerrado. \\
\item Si $\{F_{\alpha}: \alpha \in \land\} $ es una familia de cerrados $ \implies \displaystyle\bigcap_{\alpha \in \land}{F_\alpha}$ cerrado.
 \end{itemize}
\end{nprop}

\begin{proof}.
(Apartado b) \\ $\Rightarrow$ \\ Tomo $x \in \overline{\R^n/A} \implies \ \forall \epsilon > 0, \ B(x,\epsilon) \cap (\R^n/A) \ne \varnothing \implies B(x,\epsilon) \not\subseteq A \implies x \not\in \mathring{A} = A \implies x \in \R^n/A \implies \R^n$ es cerrado. \\
$\Leftarrow \\ $Suponemos $\R^n/A$ cerrado y tomo $a \in A \implies a \not\in \R^n/A = \overline{\R^n/A} \implies \exists r > 0: B(a,r) \cap (\R^n/A) = \varnothing \implies B(a,r) \subseteq A \implies a \in \mathring{A} \implies A $ abierto.
\end{proof}

\begin{ndef}[Sucesión parcial]
Sea $n \in \N, \ \{x_n\} \subseteq \R^n.$ Una sucesión parcial de $\{x_n\} $ es otra sucesión $\{x_{\sigma(n)}\}$ donde $\sigma: N \rightarrow N$ es creciente.
\end{ndef}

\begin{ndef}[Conjunto compacto]
Sea $n \in \N, \ A \subseteq \R^n.$ Diremos que $A$ es compacto si $ \ \forall \{x_n\} \subseteq A, \ \exists \{x_{\sigma(n)}\} \rightarrow x \in A$.
\end{ndef}

\begin{nprop}
Sea $n \in \N, \ A \subseteq \R^n$ compacto. Entonces A es cerrado y acotado.
\end{nprop}

\begin{proof}.
\\ Demostremos que $A$ es cerrado. Tomo $x \in \overline{A}, $ hay que ver si $x \in A$. Como $x \in \overline{A} \implies \exists \{x_n\} \rightarrow x, \ \{x_n\} \in A.$ Como $A$ es compacto $ \implies \exists \{x_{\sigma(n)}\} \rightarrow a \in A.$ Por las propiedades de las sucesiones parciales, $\{x_{\sigma(n)}\} \rightarrow x \implies x = a \in A$ por unicidad de límites. Es decir, A es cerrado. \\ \\
Demostremos que A es acotado. Supongamos, por reducción al absurdo, que $A$ no es acotado. Entonces $ \ \forall n \in \N \ \exists x_n \in A: ||x_n|| > M, \ M \in \N.$ Tenemos que $\{x_n\}$ no tiene parciales acotadas, luego no tiene parciales convergentes, en contradicción con la definición de compacidad. Es decir, A es acotado.
\end{proof}

\begin{nth}[Teorema de Bolzano-Weirstrass n-dimensional]
Sea $n \in \N.$ Entonces, toda sucesión acotada admite una parcial convergente.
\end{nth}

\begin{proof}
COMPLETAR MÁS ADELANTE
\end{proof}

\begin{nth}
	Sea $n \in \N, \ A \subseteq \R^n$ cerrado y acotado. Entonces, A es compacto.
\end{nth}

\begin{proof}
Tomo $ \{x_n\} \subseteq A.$ Como $A$ es acotado $ \implies \{x_n\}$ acotada.$ \implies \{x_n\}$ tiene una parcial convergente a $x \in \R^n.$ Como $x$ es límite de una sucesión de puntos de $A \implies x \in \overline{A} = A$ por ser $A$ cerrado $ \implies A$ es compacto.
\end{proof}

\begin{ejemplo}
Dado $x \in \R^n, r > 0 \implies A = B(x,r)$ es compacto.
\end{ejemplo}

\begin{proof}
	Veamos que la bola es cerrada y acotada. \\
	$A$ es trivialmente acotado, pues $A \subseteq \overline{B}(0, ||x||+r).$ Demostremos que es cerrado. Sabemos que $A = \{y \in R^n: ||x-y|| \le r\}$, veamos que $ \R^n/A = \{y \in \R^n: ||x-y||>r\}$ es abierto. Para ver que es abierto, tomo $y \in \R^n/A$ y $ \{y_n\} \rightarrow y, \ y_n \in A$. Como $\{y_n\} \rightarrow y \implies \{y_n-x\} \rightarrow y - x \implies \{||y_n-x||\} \rightarrow ||y - x|| > r \implies \exists m: \forall n \ge m \implies ||y_n-x||>r \implies y_n \in \R^n/A \ \forall n \ge m \implies \R^n/A$ es abierto, luego $A$ es cerrado.
\end{proof}


\section{Continuidad}

Generalizando lo que conocemos para funciones reales de variable real, vamos a estudiar
las nociones de límite y continuidad para funciones entre dos espacios métricos cualesquiera.
Las definimos de forma que quede claro que se trata de nociones topológicas. Analizamos con
detalle el carácter local de ambas nociones, aclaramos la relación entre ellas y comprobamos
que la composición de aplicaciones preserva la continuidad. Al considerar el límite de una
composición de funciones, obtenemos una regla de cambio de variable, útil en la práctica para
el cálculo de límites. Prestamos especial atención al caso particular de funciones definidas en un
subconjunto de $\R^n$ y con valores en $\R^m$ donde $m \in \N$ , que se denominan campos escalares
cuando $m = 1$ , o campos vectoriales cuando $m > 1$ .

\begin{ndef}[Continuidad en un punto]
Sea $f: A \subseteq \R^n \to \R, \ a \in \R^n. \ f $ es continua en $a$ si $$ \forall \epsilon > 0 \ \exists \delta > 0: \|x-a\| < \delta \implies \|f(x)-f(a)\| < \epsilon$$
\end{ndef}

Por lo tanto, $f$ es continua en $A$ si en continua en todos sus puntos. Tambien puede demostrarse que la suma, producto, composiciones y restricciones de funciones, entre ellas, son funciones continuas (Se deja como ejercicio al lector.)

También son continuas las funciones racionales en todo su dominio, es decir, donde el denominador sea distinto de 0.

\begin{nprop}[Carácter local de la continuidad]
Sea $f: A \subseteq \R^n \to \R, \ a \in B, a \in \mathring{B}$. Si $f_{|B}$ es continua en $a \implies f$ es continua en $a$.
\end{nprop}

\begin{proof}.
\\ Por la continuidad, $ \forall \epsilon > 0 \ \exists \delta_1 > 0: \|x-a\| < \delta_1 \implies \|f(x)-f(a)\| < \epsilon \ \forall x \in B$. Tengo que ver que $ \forall \epsilon > 0 \ \exists \delta_2 > 0: \|x-a\| < \delta_2 \implies \|f(x)-f(a)\| < \epsilon \ \forall x \in A$ \\

Como $a \in \mathring{B} \implies \exists r > 0: B(a,r) \subseteq B \implies $Si $x \in A, \|x-a\| < r \implies x \in B.$ Sea $d_2 = min\{d_1,r\} \implies $Si $x \in A, \|x-a\| < d_2 \implies \|x-a\| < r \implies x \in B $ y $ \|x-a\| < d_2 \implies \|f(x)-f(a)\| < \epsilon$
\end{proof}

\begin{nprop}[Caracterización secuencial de la continuidad]
Sean $m,n \in \N, A \subseteq \R^n$ no vacío y $f:A \to \R^m $ una función. Son equivalentes:

\begin{itemize}
	\item $f$ es continua en $a$.
	\item $\forall \{x_n\} \rightarrow a, x_n \in A \implies \{f(x_n)\} \to f(a)$
\end{itemize}
\end{nprop}

\begin{proof}.
\\ $2 \implies 1.$ \\
Por contrarrecíproco, probamos $\lnot 1 \implies \lnot 2$. Supongamos que $f$ no es continua, entonces $ \exists \epsilon_0 > 0 \ \forall \delta > 0: \exists x \in A \ \|x-a\| < \delta \implies \|f(x)-f(a)\| \ge \epsilon_0$. \\

Para $\delta = \frac{1}{n} \implies \exists \{x_n\}: \|x_n-a\| < \frac{1}{n} \implies \|f(x_n)-f(a)\| \ge \epsilon_0$. Tenemos que $\{x_n\} \rightarrow a$ pero $\{f(x_n)\} \not\rightarrow f(a)$, como queríamos demostrar. 
\end{proof}

\begin{nprop}[Continuidad y compacidad]
Sean $m,n \in \N, K \subseteq \R^n$ compacto y $f:K \to \R^m $ continua. Entonces $f(K) = \{f(x):x \in K)\}$ es compacto.
\end{nprop}

\begin{proof}.
\\ Sea $\{y_n\} \subseteq K.$ Por definición, $ \ \forall n \in \N \ \exists x_n \in K: y_n = f(x_n)$. Como $x_n \in K$ compacto $ \implies \exists \{x_{\sigma(n)}\} \rightarrow x \in K$.

Por la caracterización secuencial de la continuidad $ \implies \{f(x_{\sigma(n)})\} \rightarrow f(x)$. Si $y = f(x) \in f(K) \implies \{y_{\sigma(n)}\} \rightarrow y \implies f(K) $ es compacto.
\end{proof}

\begin{nth}[Teorema de Weirstrass]
Sean $n \in \N, K \subseteq \R^n$ compacto y $f:K \to \R $ continua. Entonces $f$ tiene máximo y mínimo.
\end{nth}

\begin{proof}.
	\\ Como $f(K) \subseteq \R$ y $f(K)$ es compacto por la proposición anterior, $f$ alcanza su máximo y mínimo.
\end{proof}

\begin{ejemplo}
Sea $K = \{(\frac{x^2+8}{e^{x+y^2}},x^2+y^3): (x,y) \in B(0,1)\}$ y $f:K \to \R: f(x,y)=y-sen(x,y)$.
\end{ejemplo}

\begin{proof}.
\\ Sabemos que f es continua por ser suma de continuas. Por otra parte, $K$ es compacto pues $K = G(B(0,1))$ con $G:\overline{B} \rightarrow \R^2 $ tal que $ G(x,y)=(\frac{x^2+8}{e^{x+y^2}},x^2+y^3) $ continua. Por ello, $f(K)$ es compacto.
\end{proof}

\begin{nprop}
	Sean $m,n \in \N, A \subseteq \R^n$ no vacío y $f:A \to \R^m $ una función. Son equivalentes:
	
	\begin{itemize}
		\item $f$ es continua en $A$.
		\item Dado $O \subseteq \R^m$ abierto $ \implies f^{-1}(O) $ es abierto relativo de $A$.
	\end{itemize}
\end{nprop}

\begin{proof}.
\\ $1 \implies 2$ \\ Sea $O \subseteq \R^m$ abierto, veamos $  f^{-1}(O) $ es abierto. Tomamos $a \in f^{-1}(O)$. Como $f(a) \in O$ y $O$ abierto $ \implies f(a) \in \mathring{O} \implies \exists \epsilon > 0: B(f(a),\epsilon) \subseteq O$. \\

Como $f$ es continua $ \implies \exists \delta > 0: f(B(a,\delta) \cap A) \subseteq B(f(a),\epsilon)$. Tomando imagen inversa, entonces: $ B(a,\delta) \subseteq f^{-1}(f(B(a,\delta) \cap A) \subseteq f^{-1}(B(f(a),\epsilon)) \subseteq f^{-1}(O)$, luego $f^{-1}(O)$ es abierto relativo de $A$.
\end{proof}

\begin{ndef}[Conjunto conexo]
Sea $n \in \N, A \subseteq \R^n$ no vacío. Diremos que $A$ es conexo si dados dos abiertos $O_1,O_2$ relativos de $A$ tal que $O_1 \cup O_2 = A, O_1 \cap O_2 = \varnothing \implies O_1 = \varnothing $ o $O_2 = \varnothing$.
\end{ndef}

\begin{ejemplo}
$\R / \{0\}$ no es conexo, pues $\R = ]-\infty,0[ \cup ]0,+\infty[$
\end{ejemplo}

\begin{ejemplo}
	$\R^n / \{0\}$ es conexo para $n \ge 2$.
\end{ejemplo}

\begin{nprop}
	Sea $A \in \R$ no vacío. Entonces $A$ es conexo $\Longleftrightarrow A$ es un intervalo.
\end{nprop}

\begin{proof}.
	\\ $\boxed{\Leftarrow}$ Trivial.
	\\ $\boxed{\Rightarrow}$ Supongamos que $A$ no es un intervalo. Entonces $ \exists x,y \in A: x<z<y, z \not\in A \implies A = (]-\infty,z[\cap A)\cup(]z,+\infty[\cap A) \implies A$ no es conexo.
\end{proof}

\begin{ndef}[Conjunto arcoconexo]
Un conjunto $A \subseteq \R^n$ es arcoconexo si $ \forall x,y \in A $ con $ x \ne y, \exists \gamma:[0,1] \to \R^n$ continua tal que $ \gamma(0)=x, \gamma(1)=y $ con $ \gamma(t) \in A \ \forall t$.
\end{ndef}

\begin{nprop}
Los conjuntos arcoconexos son conexos.
\end{nprop}

\begin{proof}.
\\ Probémoslo por el contrarrecíproco. Supongamos que $A$ no es conexo. Si $A = O_1 \cup O_2$ es una partición no trivial, tomamos $x \in O_1, y \in O_2 $ uniendo los abiertos. Tomamos $ \gamma:[0,1] \to \R^n$ tal que $ I_1 = \gamma^{-1}(O_1)$ y $ I_2 = \gamma^{-1}(O_2)$. Por ser $\gamma$ continua, $I_1$ Y $I_2$ son abiertos relativos de $[0,1]$. Además, $[0,1] = I_1 \cup I_2$ y $I_1 \cap I_2 = \varnothing$. \\

Si $s = sup\{t: \gamma(t) \in O_1\} \implies s \in I_1$ pero $s \not\in I_2$ luego $A$ no es arcoconexo. 
\end{proof}

\begin{ndef}[Conjunto convexo]
	Un conjunto $A \subseteq \R^n$ es convexo si dados $ x,y \in A \implies \\ \lambda x + (1-\lambda)y \in A \ \forall \lambda \in [0,1]$.
\end{ndef}

\begin{nprop}
Si $A$ es convexo, es arcoconexo, luego también es conexo.
\end{nprop}

\begin{proof}
Dados $ x,y \in A \ \exists \gamma:[0,1] \rightarrow \R^n \ \gamma(t)= tx + (1-t)y$
\end{proof}

\begin{ejemplo}
Las bolas en $\R^n$ son convexas.
\end{ejemplo}

\begin{proof}.
\\ Sea $A = B(z,r)$. Veamos que $A$ es convexo. Tomemos $x,y \in A \implies \|[\lambda x + (1-\lambda)y]-z\| = \|\lambda x+(1-\lambda)y-(\lambda+1-\lambda)z\| = \|\lambda (x-z)+(1-\lambda)(y-z\|) \le |\lambda|\|x-z\|+|1-\lambda|\|y-z\| < |\lambda|r + |1-\lambda|r = r(\lambda+1-\lambda)=r \implies \lambda x+(1-\lambda)y \in A$.
\end{proof}

\begin{nth}
Sean $n,m \in \N, A \subseteq \R^n$ conexo, $f:A \to \R^m$ continua. Entonces, $f(A)$ es conexo.
\end{nth}

\begin{proof}.
\\ Tomamos $O_1,O_2$ abiertos tal que $f(A)=(O_1 \cap f(A)) \cup (O_2 \cap f(A))$ y $(O_1 \cap f(A)) \cap (O_2 \cap f(A)) = \varnothing$. Entonces $A= f^{-1}(O_1) \cup f^{-1}(O_2)$ con intersección vacía y $f^{-1}(O_1) $ y $ f^{-1}(O_2)$ son abiertos por ser $f$ continua. \\

Como $A$ es conexo $ \implies f^{-1}(O_1)=A$ y $f^{-1}(O_2)=\varnothing$ (o viceversa). Como $ f^{-1}(O_2)=\varnothing \implies O_2 \cap f(A) = \varnothing \implies f(A)$ es conexo.
\end{proof}

\begin{nth}[Teorema del Valor Intermedio]
Sean $n,m \in \N, A \subseteq \R^n$ conexo, $f:A \to \R^m$ continua. Si $\alpha,\beta \in f(A), \alpha < \gamma < \beta \implies \gamma \in f(A)$.
\end{nth}

\begin{proof}
Como $f(A)$ es conexo y $f(A) \subseteq \R \implies A$ es un intervalo.
\end{proof}

\section{Límites en $\R^n$}

\begin{ndef}
Sean $n \in \N, A \subseteq \R^n$ y $x \in \R^n$. Diremos que $x$ es un punto de acumulación de $A$ si $ \forall \epsilon > 0, (B(x,\epsilon)/\{x\}) \cap A \ne \varnothing$. Denotaremos por $A^{'} = \{x \in \R^n: x$ es punto de acumulación$\}$.
\end{ndef}

\begin{ndef}
	Sean $n \in \N, A \subseteq \R^n$ y $x \in \R^n$. Diremos que $x$ es un punto aislado de $A$ si $ \exists \epsilon_0 > 0, (B(a,\epsilon)/\{a\}) \cap A == \varnothing$.  
\end{ndef}



\section{Ejercicios}

\subsection{Relación $1$}

\begin{ejer}
Sea $n \in \N$. Demostrar que dados $x,y \in \R^n$ se cumple que:
$$ | \ \|x\| - \|y\| \ |  \le \|x-y\|$$
\end{ejer}

\begin{sol}.
\\ $ ||x|| = ||x-y+y|| \le ||x-y||+||y||$ por la propiedad triangular $ \implies  ||x||-||y|| \le ||x-y||$. Esto es cierto para cualquier $x,y \in \R^n \implies ||y||-||x|| \le ||y-x|| = ||x-y|| $ \ (Fijarse que $||y||-||x||$ es el opuesto de $||x||-||y||$)$ \implies | \ \|x\| - \|y\| \ |  \le \|x-y\|$
\end{sol}

\begin{ejer}
Sea $n \in \N$ y $x,y,z \in \R^n$. Demostrar que se cumple que:
 \\ a) $d(x,y) \ge 0 \land d(x,y) = 0 \Leftrightarrow x = y.$ \\
b) $d(x,y)=d(y,x) $
 \\ c) $d(x,y) \le d(x,z)+d(z,y)$
\end{ejer}

\begin{sol}.
\\ a) $ d(x,y) = \displaystyle{\sqrt{\sum_{i=1}^{n}(x_i-y_i)^2}} = \sqrt{<x-y,x-y>} \ge 0$ y la igualdad a 0 es trivial. \\
b) $ d(x,y) = \displaystyle{\sqrt{\sum_{i=1}^{n}(x_i-y_i)^2}} = \displaystyle{\sqrt{\sum_{i=1}^{n}(y_i-x_i)^2}} = d(y,x)$ \\
c) $ d(x,y)^2 = <x-y,x-y> = <x-z+z-y,x-z+z-y> = <x-z,x-z> + <z-y,z-y> + 2<x-z,z-y> \ \le \ <x-z,x-z> + <z-y,z-y> + 2\|x-z\| \|z-y\|$ por la desigualdad de Cauchy-Schwarz. $ \implies <x-z,x-z> + <z-y,z-y> + 2\|x-z\| \|z-y\| = d(x,z)^2+d(z,y)^2+2d(x,z)d(z,y) = (d(x,z)+d(z,y))^2.$ En definitiva, tenemos que $ d(x,y)^2 \le (d(x,z)+d(z,y))^2 \implies |d(x,y)| \le |d(x,z)+d(z,y)|$ y como las distancias son positivas $ \implies d(x,y) \le d(x,z)+d(z,y)$
\end{sol}

\begin{ejer}
Sea $n \in \N$ y $x,y \in \R^n$. Demostrar que $ \|x+y\|^2 = \|x\|^2+\|y\|^2 \Leftrightarrow <x,y> = 0$
\end{ejer}

\begin{sol}.
\\ $ \|x+y\|^2 = <x+y,x+y> = <x,x> + <y,y> + 2<x,y> \implies \|x\|^2+\|y\|^2 + 2<x,y> = \|x\|^2+\|y\|^2 \Leftrightarrow 2<x,y> = 0 \Leftrightarrow <x,y> = 0$
\end{sol}

\begin{ejer}
Sea $n \in \N$ y $x,y \in \R^n$. \\
a) Demostrar que existe un único $ \theta \in [0, \pi]$ de manera que: $$ cos(\theta) = \frac{<x,y>}{\|x\|\|y\|}$$ \ \ \ \ A este único $\theta$ lo llamaremos el ángulo entre $x$ e $y$ y lo denotaremos por $\theta(x, y)$. \\
b) Demostrar la siguiente identidad: $$ \|x+y\|^2 = \|x\|^2+\|y\|^2 + 2 \|x\|\|y\|cos(\theta(x,y))$$
\end{ejer}

\begin{sol}.
\\ a) Por la desigualdad de Cauchy-Schwarz, sabemos que $ |<x,y>| \le \|x\|\|y\| \implies \frac{|<x,y>|}{\|x\|\|y\|} \le 1 \implies -1 \le \frac{<x,y>}{\|x\|\|y\|} \le 1$, luego puede establecerse una aplicación biyectiva $ cos: [0,\pi] \rightarrow [-1,1]$ tal que $cos(\theta) = \frac{<x,y>}{\|x\|\|y\|}$ para $ \theta \in [0,\pi]$. \\
b) $ \|x+y\|^2 = <x+y,x+y> = <x,x> + <y,y> + 2<x,y> = <x,x> + <y,y> + 2\|x\|\|y\|cos(\theta(x,y))$ \\
\end{sol}

\begin{ejer}
Sea $n \in \N$ y $x \in \R^n, \{x_n\} \in \R^n$. Demostrar que $\{x_n\} \rightarrow x$ si, y sólamente si, todas las sucesiones parciales de $\{x_n\}$ convergen hacia $x$. 
\end{ejer}

\begin{sol}.
\\ $\boxed{\Leftarrow}$ \\ Supongamos $ \sigma(n)=n \implies \{x_n\} $ es parcial de $ \{x_n\} \implies \{x_n\} \rightarrow x. \\
 \boxed{\Rightarrow} \\ \forall \epsilon > 0, \ \exists m \in \N: n \ge m: ||x_n-x|| < \epsilon$. Si $n \ge m \implies \sigma(n) \ge m \implies \forall \epsilon > 0, \ \exists m \in \N: \sigma(n) \ge m: ||x_{\sigma(n)}-x|| < \epsilon$.
\end{sol}

\begin{ejer}
Sea $A \subseteq R$ compacto. Entonces, $A$ tiene máximo y mínimo.
\end{ejer}

\begin{sol}.
\\ Como $A$ compacto, es cerrado y acotado, luego tiene supremo por ser acotado, es decir, $ \exists \alpha = sup A \in R \implies \ \forall \epsilon > 0 \ \exists a \in A: a > \alpha - \epsilon$. Sea $ n \in N, \epsilon = \frac{1}{n} \implies \exists a_n \in A: a_n > \alpha - \frac{1}{n} \implies \alpha - a_n < \frac{1}{n} \rightarrow 0 \implies lim a_n = \alpha. \\
$Como $A$ es compacto, $ \exists a_{\sigma(n)} \rightarrow a \in A$, pero $a_{\sigma(n)}	$ converge a $\alpha$ por ser parcial de una convergente a $\alpha$, luego $\alpha = a$ por unicidad de límites $\implies \alpha \in A \implies \alpha = max A$.

Probariamos lo mismo para el mínimo, con un cambio adecuado de signo.
\end{sol}

\begin{ejer}
Sea $n \in \N$ y $\{x_n\} \in \R^n$. Diremos que la
sucesión $\{x_n\}$ es de Cauchy si dado $\epsilon > 0 \ \exists m \in \N$ de manera que: $$ p,q \ge m \implies \|x_p-x_q\| < \epsilon$$
Demostrar: \\
 \begin{nlist}
\item a) Las sucesiones de Cauchy están acotadas. \\
\item b) Si $\{x_n\}$ es una sucesión de Cauchy y tiene una parcial $\{x_{\sigma(n)}\}$ convergente hacia un punto $x$, entonces $\{x_n\} \rightarrow x$. \\
\item c) Una sucesión $\{x_n\} \in \R^n $ es de Cauchy si, y sólamente si, $\{x_n\}$ es convergente (Complitud de $R^n$).
\end{nlist}
\end{ejer}

\begin{sol}.
\\a) \ Sea $\epsilon = 1 \implies \exists n_0 \in \N: \|x_p-x_q\| < \epsilon \ \forall p,q \ge n_0.$ \\ Sea $ n, n_0 \ge n_0 \implies \|x_n-x_{n_0}\| < 1 \implies \{x_n: n \ge n_0\} \subseteq B(x_{n_0},1)$. \\ Sean $\alpha = sup \{\|x_n\|: n \ge n_0 \}, \beta = inf \{\|x_n\|: n \ge n_0 \}, M = max \{\|x_n\|: n < n_0 \}, m = inf \{\|x_n\|: n < n_0 \} \implies min \{\beta,m\} \le \|x_n\| \le max \{\alpha,M\} \implies \{x_n\} $ acotada. \\

b) $x_n$ es de Cauchy $ \implies \ \forall \epsilon > 0 \ \exists n_0 \in \N: \|x_p-x_q\| < \frac{\epsilon}{2} \ \forall p,q \ge n_0$  \\
$\{x_{\sigma(n)}\} \rightarrow x \implies \ \forall \epsilon > 0 \ \exists n_1 \in \N: \|x_{\sigma(n)}-x\| < \frac{\epsilon}{2} \ \forall n \ge n_1$. \\
Sea $n \ge max\{n_0,n_1\} \implies \|x_n-x\| = \|x_n-x_{\sigma(n)}+x_{\sigma(n)}-x\| \le \|x_n-x_{\sigma(n)}\|+\|x_{\sigma(n)}-x\| < \frac{\epsilon}{2}+\frac{\epsilon}{2} = \epsilon \implies \{x_n\} \rightarrow x$ \\

c) $\boxed{\Leftarrow} \\ \{x_n\}$ converge $ \implies \exists x: lim x_n = x \implies \forall \epsilon > 0 \ \exists n_0 \in \N: \|x_n-x\| < \frac{\epsilon}{2} \ \forall n \ge n_0$.\\ Por otra parte: $ \forall p,q \ge n_0 : ||x_p-x_q|| \le ||x_p-x||+||x-x_q|| < \frac{\epsilon}{2}+\frac{\epsilon}{2} = \epsilon \implies \{x_n\}$ es de Cauchy. \\
$\boxed{\Rightarrow} \\
\{x_n\}$ es de Cauchy. Entonces es acotada (por el apartado a), luego admite una parcial convergente (por Bolzano-Weirstrass). Por el apartado b, la parcial converge al mismo punto que la principal.
\end{sol}



% --------------------------------------------------------------------------------
% Bibliografía.
% --------------------------------------------------------------------------------
\newpage

\begin{thebibliography}{9}
  \addcontentsline{toc}{section}{Referencias}

.

\end{thebibliography}



% ---------------------------------------------------------------------------
% FIN DEL DOCUMENTO
% ---------------------------------------------------------------------------

\end{document}

uvtxdfff
