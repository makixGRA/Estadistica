%%%%%%%%%%%%%%%%%%%%%%%%%%%%%%%%%%%%%%%%%%%%%%%%%%%%%%%%%%%%%%%%%%%%%%%%% 
% 
% Apuntes de la asignatura Análisis Matemático II.
% Grado en Estadística.
% Universidad de Granada.
% Curso 2017/18.
% 
% 
% Colaboradores:
% Miguel Anguita Ruiz
% 
% Agradecimientos:
% Andrés Herrera (@andreshp) y Mario Román (@M42) por
% las plantillas base.
% 
% Sitio original:
% https://github.com/libreim/apuntesDGIIM/
% 
% Licencia:
% CC BY-NC-SA 4.0 (https://creativecommons.org/licenses/by-nc-sa/4.0/)
% 
%%%%%%%%%%%%%%%%%%%%%%%%%%%%%%%%%%%%%%%%%%%%%%%%%%%%%%%%%%%%%%%%%%%%%%%%% 


% ------------------------------------------------------------------------------
% ACKNOWLEDGMENTS
% ------------------------------------------------------------------------------

%%%%%%%%%%%%%%%%%%%%%%%%%%%%%%%%%%%%%%%%%%%%%%%%%%%%%%%%%%%%%%%%%%%%%%%% 
% Plantilla básica de Latex en Español.
% 
% Autor: Andrés Herrera Poyatos (https://github.com/andreshp) 
% 
% Es una plantilla básica para redactar documentos. Utiliza el paquete fancyhdr
% para darle un estilo moderno pero serio.
% 
% La plantilla se encuentra adaptada al español.
% 
%%%%%%%%%%%%%%%%%%%%%%%%%%%%%%%%%%%%%%%%%%%%%%%%%%%%%%%%%%%%%%%%%%%%%%%%% 

%%%%%%%%%%%%%%%%%%%%%%%%%%%%%%%%%%%%%%%%%%%%%%%%%%%%%%%%%%%%%%%%%%%%%%%%% 
% Plantilla de Trabajo
% Modificación de una plantilla de Latex de Frits Wenneker para adaptarla 
% al castellano y a las necesidades de escribir informática y matemáticas.
% 
% Editada por: Mario Román
% 
% License:
% CC BY-NC-SA 3.0 (http://creativecommons.org/licenses/by-nc-sa/3.0/)
%%%%%%%%%%%%%%%%%%%%%%%%%%%%%%%%%%%%%%%%%%%%%%%%%%%%%%%%%%%%%%%%%%%%%%%%% 

%%%%%%%%%%%%%%%%%%%%%%%%%%%%%%%%%%%%%%%%%%%%%%%%%%%%%%%%%%%%%%%%%%%%%%%%% 
% Short Sectioned Assignment
% LaTeX Template
% Version 1.0 (5/5/12)
% 
% This template has been downloaded from:
% http://www.LaTeXTemplates.com
% 
% Original author:
% Frits Wenneker (http://www.howtotex.com)
% 
% License:
% CC BY-NC-SA 3.0 (http://creativecommons.org/licenses/by-nc-sa/3.0/)
% 
%%%%%%%%%%%%%%%%%%%%%%%%%%%%%%%%%%%%%%%%%%%%%%%%%%%%%%%%%%%%%%%%%%%%%%%%% 


% Tipo de documento y opciones.
%\documentclass[11pt, a4paper, twoside]{article} % Usar para imprimir
\documentclass[11pt, a4paper]{article}

% ---------------------------------------------------------------------------
% PAQUETES
% ---------------------------------------------------------------------------

% Idioma y codificación para Español.
\usepackage[utf8]{inputenc}
\usepackage[spanish, es-tabla, es-lcroman, es-noquoting]{babel}
\selectlanguage{spanish} 
% \usepackage[T1]{fontenc}

% Fuente utilizada.
\usepackage{courier}    % Fuente Courier.
\usepackage{microtype}  % Mejora la letra final de cara al lector.

% Diseño de página.
\usepackage{fancyhdr}   % Utilizado para hacer cabeceras y pies de página.

\usepackage{titlesec} 	% Utilizado para hacer títulos propios.
\usepackage{lastpage}   % Referencia a la última página.
\usepackage{extramarks} % Marcas extras. Utilizado en pie de página y cabecera.
\usepackage[parfill]{parskip}    % Crea una nueva línea entre párrafos.
\usepackage{geometry}            % Geometría de las páginas.

% Símbolos y matemáticas.
\usepackage{amssymb, amsmath, amsthm, amsfonts, amscd}
\usepackage{upgreek}
\usepackage{mathrsfs}

\usepackage{mdframed}

% Gráficos

\usepackage{pgf,tikz}
\usepackage{tkz-euclide}
\usetkzobj{all}

% Otros.
\usepackage{enumitem}   % Listas mejoradas.
\usepackage{hyperref}
\usepackage{graphicx}   % Gráficos.
\usepackage[space]{grffile}  % Permitir espacios en rutas de gráficos
\usepackage{xcolor}     % Colores.8

% Fuentes personalizadas

\usepackage[scaled=.85]{newpxtext,newpxmath}
\usepackage[scaled=.85]{FiraSans}
\usepackage[T1]{fontenc}

% Código para ajustar las fuentes matemáticas al estilo del texto
% que le rodea

\DeclareMathVersion{sans}
\SetSymbolFont{operators}{sans}{OT1}{cmbr}{m}{n}
\SetSymbolFont{letters}{sans}{OML}{cmbr}{m}{it}
\SetSymbolFont{symbols}{sans}{OMS}{cmbrs}{m}{n}
\SetMathAlphabet{\mathit}{sans}{OT1}{cmbr}{m}{sl}
\SetMathAlphabet{\mathbf}{sans}{OT1}{cmbr}{bx}{n}
\SetMathAlphabet{\mathtt}{sans}{OT1}{cmtl}{m}{n}
\SetSymbolFont{largesymbols}{sans}{OMX}{iwona}{m}{n}

\DeclareMathVersion{boldsans}
\SetSymbolFont{operators}{boldsans}{OT1}{cmbr}{b}{n}
\SetSymbolFont{letters}{boldsans}{OML}{cmbrm}{b}{it}
\SetSymbolFont{symbols}{boldsans}{OMS}{cmbrs}{b}{n}
\SetMathAlphabet{\mathit}{boldsans}{OT1}{cmbr}{b}{sl}
\SetMathAlphabet{\mathbf}{boldsans}{OT1}{cmbr}{bx}{n}
\SetMathAlphabet{\mathtt}{boldsans}{OT1}{cmtl}{b}{n}
\SetSymbolFont{largesymbols}{boldsans}{OMX}{iwona}{bx}{n}

\newif\IfInSansMode
\let\oldsf\sffamily
\renewcommand*{\sffamily}{\oldsf\mathversion{sans}\InSansModetrue}
\let\oldmd\mdseries
\renewcommand*{\mdseries}{\oldmd\IfInSansMode\mathversion{sans}\fi\relax}
\let\oldbf\bfseries
\renewcommand*{\bfseries}{\oldbf\IfInSansMode\mathversion{boldsans}\else%
   \mathversion{bold}\fi\relax}
\let\oldnorm\normalfont
\renewcommand*{\normalfont}{\oldnorm\InSansModefalse\mathversion{normal}}
\let\oldrm\rmfamily
\renewcommand*{\rmfamily}{\oldrm\InSansModefalse\mathversion{normal}}

% Colores

\definecolor{50}{HTML}{E8F5E9}
\definecolor{300}{HTML}{81C784}
\definecolor{500}{HTML}{4CAF50}
\definecolor{700}{HTML}{388E3C}

% ---------------------------------------------------------------------------
% OPCIONES PERSONALIZADAS
% ---------------------------------------------------------------------------

% Formato de texto.
\linespread{1.3}            % Espaciado entre líneas.
\setlength\parindent{0pt}   % No indentar el texto por defecto.
\setlist{leftmargin=.5in}   % Indentación para las listas.

% Estilo de página.
\pagestyle{fancy}
\fancyhf{}
\geometry{left=3cm,right=3cm,top=3cm,bottom=3cm}   % Márgenes y cabecera.

% Estilo de las cabeceras

\titleformat{\section}
  {\Large\bfseries\sffamily}{\thesection}{1em}{}
\titleformat{\subsection}
  {\large\sffamily}{\thesubsection}{1em}{}
\titleformat{\subsubsection}
  {\sffamily}{\thesubsubsection}{1em}{}

% Estilo de enlaces
\hypersetup{
  % hidelinks = true,   % Oculta todos los enlaces.
  colorlinks = true,   % Muestra todos los enlaces, sin bordes alrededor.
  linkcolor=black,     % Color de enlaces genéricos
  citecolor={blue!50!black},   % Color de enlaces de referencias
  urlcolor={blue!80!black}     % Color de enlaces de URL
}

% Ruta donde buscar gráficos
\graphicspath{{../Recursos/Plantillas/} {Recursos/Plantillas/} {./img/} {Análisis Matemático II/img/}}

% Redefinir entorno de demostración (reducir espacio superior)
% \makeatletter
% \renewenvironment{proof}[1][\proofname] {\vspace{-15pt}\par\pushQED{\qed}\normalfont\topsep6\p@\@plus6\p@\relax\trivlist\item[\hskip\labelsep\it#1\@addpunct{.}]\ignorespaces}{\popQED\endtrivlist\@endpefalse}
% \makeatother

% Aumentar el tamaño del interlineado
\linespread{1.3}

% Permitir salto de página en ecuaciones
\allowdisplaybreaks


% ---------------------------------------------------------------------------
% COMANDOS PERSONALIZADOS
% ---------------------------------------------------------------------------

% Redefinir letra griega épsilon.
\let\epsilon\upvarepsilon

% Valor absoluto: \abs{}
\providecommand{\abs}[1]{\lvert#1\rvert}    

% Fracción grande: \ddfrac{}{}
\newcommand\ddfrac[2]{\frac{\displaystyle #1}{\displaystyle #2}}

% Texto en negrita en modo matemática: \bm{}
\newcommand{\bm}[1]{\boldsymbol{#1}}

% Línea horizontal.
\newcommand{\horrule}[1]{\rule{\linewidth}{#1}}

% Letras de conjuntos
\newcommand{\R}{\mathbb{R}} \newcommand{\N}{\mathbb{N}}

% Sucesiones
\newcommand{\xn}{\{x_n\}}
\newcommand{\fn}{\{f_n\}}

% Letra griega "chi" en línea con el texto
\DeclareRobustCommand{\rchi}{{\Large \mathpalette\irchi\relax}}
\newcommand{\irchi}[2]{\raisebox{0.4\depth}{$#1\chi$}} % inner command, used by \rchi 

% Letra 'omega'
\newcommand{\W}{\Omega}
\newcommand{\w}{\omega}


% ---------------------------------------------------------------------------
% CABECERA Y PIE DE PÁGINA
% ---------------------------------------------------------------------------

\renewcommand{\sectionmark}[1]{%
\markboth{#1}{}}

\renewcommand{\subsectionmark}[1]{%
\markright{#1}{}}

%\addtolength{\headheight}{4ex}
\renewcommand{\headrulewidth}{0pt}

\addtolength{\headwidth}{\marginparsep}
%\addtolength{\headwidth}{\marginparwidth}

\fancypagestyle{section}{%
  \fancyhead{}
  %\addtolength{\headheight}{-10ex}
  %\renewcommand{\headheight}{0pt}%
  %\setlength{\footskip}{-48pt}%
  \fancyfoot[LE,RO]{\Large\sffamily\thepage}
  \renewcommand{\headrulewidth}{0pt}%
  \renewcommand{\footrulewidth}{0pt}%
}

\let\originalsection\section
\RenewDocumentCommand{\section}{som}{%
  \IfBooleanTF{#1}
    {\originalsection*{#3}}
    {\IfNoValueTF{#2}
      {\originalsection{#3}}
      {\originalsection[#2]{#3}}%
    }%
  \thispagestyle{section}%
}

\fancyhead[LE,RO]{\rule[-4ex]{0pt}{2ex}\sffamily\textsl{\rightmark}}
\fancyhead[LO,RE]{\sffamily{\leftmark}}
\fancyfoot[LE,RO]{\Large\sffamily\thepage}



% ---------------------------------------------------------------------------
% ENTORNOS PARA MATEMÁTICAS
% ---------------------------------------------------------------------------

% Nuevo estilo para definiciones.
\newtheoremstyle{definition-style} % Nombre del estilo.
{}               % Espacio por encima.
{}               % Espacio por debajo.
{}                   % Fuente del cuerpo.
{}                   % Identación.
{\bf\sffamily}                % Fuente para la cabecera.
{.}                  % Puntuación tras la cabecera.
{.5em}               % Espacio tras la cabecera.
{\thmname{#1}\thmnumber{ #2}\thmnote{ (#3)}}     % Especificación de la cabecera (actual: nombre en negrita).

% Nuevo estilo para notas.
\newtheoremstyle{remark-style} 
{10pt}                
{10pt}                
{}                   
{}                   
{\itshape \sffamily}          
{.}                  
{.5em}               
{}                  

% Nuevo estilo para teoremas y proposiciones.
\newtheoremstyle{theorem-style}
{}                
{}                
{}           
{}                  
{\bfseries \sffamily}             
{.}                
{.5em}           
{\thmname{#1}\thmnumber{ #2}\thmnote{ (#3)}}

% Nuevo estilo para ejemplos.
\newtheoremstyle{example-style}
{10pt}                
{10pt}                
{}                  
{}                   
{\bf \sffamily}              
{}                 
{.5em}               
{\thmname{#1}\thmnumber{ #2.}\thmnote{ #3.}}  

% Nuevo estilo para la demostración
%\renewenvironment{proof}{{\itshape \sffamily Demostración \\}}{\hspace*{\fill}\qed}

\makeatletter
\renewenvironment{proof}[1][\proofname] {\par\pushQED{\qed}\normalfont\topsep6\p@\@plus6\p@\relax\trivlist\item[\hskip\labelsep\itshape\sffamily#1\@addpunct{.}]\ignorespaces}{\popQED\endtrivlist\@endpefalse}
\makeatother

% Configuración general de mdframe, los estilos de los teoremas, etc
\mdfsetup{skipabove=12pt,skipbelow=12pt,innertopmargin=12pt,innerbottommargin=4pt}

% Creamos los 'marcos' de los estilos

\mdfdefinestyle{nth-frame}{
	linewidth=2pt, %
	linecolor= 500, % 
	%linecolor=black,
	topline=false, %
	bottomline=false, %
	rightline=false,%
	leftmargin=0pt, %
	innerleftmargin=1em, % 
	rightmargin=0pt, %
	innerrightmargin=0pt, % % 
	%splittopskip=\topskip, %
	%splitbottomskip=\topskip, %
}% 

\mdfdefinestyle{nprop-frame}{
	linewidth=2pt, %
	linecolor= 300, %
	%linecolor= gray, 
	topline=false, %
	bottomline=false, %
	rightline=false,%
	leftmargin=0pt, %
	innerleftmargin=1em, %
	innerrightmargin=0em, 
	rightmargin=0pt, %
	%splittopskip=\topskip, %
	%splitbottomskip=\topskip, %
}%       

\mdfdefinestyle{ndef-frame}{
	linewidth=2pt, %
	linecolor= 300, % 
	%linecolor= gray!50,
	backgroundcolor= 50,
	%backgroundcolor= gray!5,
	topline=false, %
	bottomline=false, %
	rightline=false,%
	leftmargin=0pt, %
	innerleftmargin=1em, %
	innerrightmargin=1em, 
	rightmargin=0pt, % 
	innertopmargin=1.5em,%
	innerbottommargin=1em, % 
	splittopskip=\topskip, %
	%splitbottomskip=\topskip, %
}% 

\mdfdefinestyle{ejemplo-frame}{
	linewidth=0pt, %
	linecolor= 300, % 
	%backgroundcolor= 50,
	leftline=false, %
	rightline=false, %
	leftmargin=0pt, %
	innerleftmargin=1.5em, %
	innerrightmargin=1.5em, 
	rightmargin=0pt, % 
	innertopmargin=0em,%
	innerbottommargin=0em, % 
	splittopskip=\topskip, %
	%splitbottomskip=\topskip, %
}%                

% Asignamos los marcos a los estilos

\surroundwithmdframed[style=nth-frame]{nth}
\surroundwithmdframed[style=nprop-frame]{nprop}
\surroundwithmdframed[style=nprop-frame]{ncor}
\surroundwithmdframed[style=ndef-frame]{ndef}
\surroundwithmdframed[style=ejemplo-frame]{ejemplo}
                 
% Asignamos los estilos definidos anteriormente a los entornos correspondientes
% Teoremas, proposiciones y corolarios.
\theoremstyle{theorem-style}
\newtheorem{nth}{Teorema}[section]
\newtheorem{nprop}{Proposición}[section]
\newtheorem{ncor}{Corolario}[section]
\newtheorem{lema}{Lema}[section]
% Definiciones.
\theoremstyle{definition-style}
\newtheorem{ndef}{Definición}[section]

% Notas.
\theoremstyle{remark-style}
\newtheorem*{nota}{Nota}

% Ejemplos.
\theoremstyle{example-style}
\newtheorem{ejemplo}{Ejemplo}[section]



% Listas ordenadas con números romanos (i), (ii), etc.
\newenvironment{nlist}
{\begin{enumerate}
    \renewcommand\labelenumi{(\emph{\roman{enumi})}}}
  {\end{enumerate}}

% División por casos con llave a la derecha.
\newenvironment{rcases}
{\left.\begin{aligned}}
    {\end{aligned}\right\rbrace}


% ---------------------------------------------------------------------------
% PÁGINA DE TÍTULO
% ---------------------------------------------------------------------------

% Título del documento.
\newcommand{\subject}{Álgebra: ejercicio 11}

% Autor del documento.
\newcommand{\docauthor}{Miguel Anguita Ruiz}

% Título
\title{
  \normalfont \normalsize 
  \textsc{Universidad de Granada} \\ [25pt]    % Texto por encima.
  \horrule{0.5pt} \\[0.4cm] % Línea horizontal fina.
  \huge \sffamily\subject\\ % Título.
  \horrule{2pt} \\[0.5cm] % Línea horizontal gruesa.
}

% Autor.
\author{\Large\sffamily{\docauthor}}

% Fecha.
\date{\vspace{-1.5em} \normalsize \sffamily Curso 2017/18}



% ---------------------------------------------------------------------------
% COMIENZO DEL DOCUMENTO
% ---------------------------------------------------------------------------

\begin{document}

\maketitle  % Título.
\vfill
\begin{center}
  %\includegraphics{by-nc-sa.pdf}  % Licencia.
\end{center}
\newpage
\tableofcontents    % Índice
\newpage


% --------------------------------------------------------------------------------
% Introducción.
% --------------------------------------------------------------------------------

\section{Enunciado}
DNI: 77149477W \\
Dada la matriz:
$$A = \begin{pmatrix}
	63 & -28 & 49 & 49\\
	63 & -28 & 49 & 49\\
	9 & -4 & 7 & 7\\
	-36 & 16 & -28 & -28 \\
\end{pmatrix}$$
	
i) Halla su polinomio característico p(x) usando transformaciones elementales. Razona que tiene todas sus raíces enteras y que una de ellas es d1*d5 - d2*d6 +
d3*d7 - d4*d8.\\
ii) Halla el determinante y clasifica el endomorfismo que define.\\
iii) Halla sus multiplicidades algebraicas y geométricas. ¿Verifica A el teorema espectral?\\
iv) Diagonaliza A por una semejanza racional. ¿Es diagonalizable A por una congruencia-semejanza?\\

\section{Solución}

Una matriz cuadrada real A, n×n, se puede interpretar como la matriz de una a.l. de $\R^n$ en sí mismo, $f: \R^n \to \R^n$, definida por la igualdad matricial $f(u)=A*X$, donde $X=u^{t}$ es u escrito como columna.

En este caso, $f: \R^4 \to \R^4$ viene definido por la siguiente matriz:

$$f(u) = \begin{pmatrix}
63 & -28 & 49 & 49\\
63 & -28 & 49 & 49\\
9 & -4 & 7 & 7\\
-36 & 16 & -28 & -28 \\
\end{pmatrix}\begin{pmatrix}
x_1\\
x_2\\
x_3\\
x_4
\end{pmatrix}=
\begin{pmatrix}
y_1\\
y_2\\
y_3\\
y_4
\end{pmatrix}$$

O sea, el endomorfismo f viene definido por la igualdad matricial $Y=A*X$, donde $Y=v^{t}$ es $f(u)=v$ escrito como columna.

La matriz A tiene dos filas linealmente dependientes, luego el determinante de A es 0 y $f$ no es monomorfismo, epimorfismo ni isomorfismo.

Su polinomio característico es el siguiente:

$$|A-\lambda I| = 0 \rightarrow 
\begin{vmatrix}
63-\lambda & -28 & 49 & 49\\
63 & -28-\lambda & 49 & 49\\
9 & -4 & 7-\lambda & 7\\
-36 & 16 & -28 & -28-\lambda \\
\end{vmatrix} = 0 $$

Haciendo transformaciones $E_{21}(-1)$, $E_{13}(-7)$ y $E_{43}(4)$ por filas, me queda:

$$\begin{vmatrix}
63-\lambda & -28 & 49 & 49\\
\lambda & -\lambda & 0 & 0\\
9 & -4 & 7-\lambda & 7\\
-36 & 16 & -28 & -28-\lambda \\
\end{vmatrix} \rightarrow 
\begin{vmatrix}
-\lambda & 0 & 7\lambda & 0\\
\lambda & -\lambda & 0 & 0\\
9 & -4 & 7-\lambda & 7\\
-36 & 16 & -28 & -28-\lambda \\ 
\end{vmatrix} \rightarrow 
\begin{vmatrix}
-\lambda & 0 & 7\lambda & 0\\
\lambda & -\lambda & 0 & 0\\
9 & -4 & 7-\lambda & 7\\
0 & 0 & -4\lambda & -\lambda \\ 
\end{vmatrix}$$

A continuación, haciendo transformaciones $E_{34}(-4)$ y $E_{12}(1)$ por columnas, me queda:

$$\begin{vmatrix}
-\lambda & 0 & 7\lambda & 0\\
\lambda & -\lambda & 0 & 0\\
9 & -4 & -21-\lambda & 7\\
0 & 0 & 0 & -\lambda \\
\end{vmatrix} \rightarrow 
\begin{vmatrix}
-\lambda & 0 & 7\lambda & 0\\
0 & -\lambda & 0 & 0\\
5 & -4 & -21-\lambda & 7\\
0 & 0 & 0 & -\lambda \\ 
\end{vmatrix}$$

Como trabajamos con transformaciones elementales usando numeros enteros, nos quedarán valores propios enteros. Esto es igual a:

$$\begin{vmatrix}
-\lambda & 0 & 7\lambda \\
0 & -\lambda & 0 \\
5 & -4 & -21-\lambda \\ 
 \end{vmatrix}(0+\lambda)= (-\lambda^3+14\lambda^2)(0-\lambda) = 0 \rightarrow (0-\lambda)(0-\lambda)(0-\lambda)(14-\lambda)=0$$

Por tanto, el espectro de los valores propios es: $$\{0,14\}$$

Vemos que uno de sus autovectores es: d1*d5-d2*d6+d3*d7-d4*d8 = $63-28+7-28=14$.

El autovalor 14 tienen multiplicidad algebraica 1 luego tendrá un único autovector asociado, mientras que el autovalor 0 tiene multiplicidad algebraica 3 y tenemos que calcular r = rango $(A-0)$ ya que $dim(V_{0})=4-r$.

$$\begin{pmatrix}
63-\lambda & -28 & 49 & 49\\
63 & -28-\lambda & 49 & 49\\
9 & -4 & 7-\lambda & 7\\
-36 & 16 & -28 & -28-\lambda \\ 
\end{pmatrix}\begin{pmatrix}
x_1\\
x_2\\
x_3\\
x_4
\end{pmatrix}=
\begin{pmatrix}
0\\
0\\
0\\
0\\
\end{pmatrix}$$

Si sustituimos el valor propio 14 en $\lambda$, nos queda un sistema de 4 ecuaciones con 4 incógnitas, luego es compatible determinado y tiene solución única. Por lo tanto, el vector propios asociado al autovalor 14 es:

$$v_1=(-7,-7,-1,4)$$  

Por otra parte, sustituimos 0 en $\lambda$ para hallar el rango de $(A-0I) = A$, entonces:

$$\begin{pmatrix}
63 & -28 & 49 & 49\\
63 & -28 & 49 & 49\\
9 & -4 & 7 & 7\\
-36 & 16 & -28 & -28 \\ 
\end{pmatrix}\begin{pmatrix}
x_1\\
x_2\\
x_3\\
x_4
\end{pmatrix}=
\begin{pmatrix}
0\\
0\\
0\\
0\\
\end{pmatrix}$$

Esta matriz tiene rango 1 pues sus submatrices de órdenes 2, 3 y 4 tienen determinante 0, y tiene una submatriz de orden 1 con determinante distinto de 0. Luego $dim(V_{0})=4-1=3$, luego su multiplicidad algebraica es igual a la geométrica. Los autovectores asociados a $\lambda=0$ son:

$$v_2=(4,9,0,0)$$ $$v_3=(-7,0,9,0)$$ $$v_4=(-7,0,0,9)$$ 

Por tanto, la multiplicidad geométrica del autovalor 14 es 1 pues tiene un único autovector asociado y la multiplicidad geométrica de 0 es 3, pues tiene tres autovectores asociados, luego A cumple el teorema espectral, se puede diagonalizar por semejanza y puede hallarse una matriz P de cambio.

Con los 4 vectores propios escritos por columnas se puede formar una matriz de cambio P que diagonaliza A. O sea:

$$P^{-1}AP = \begin{pmatrix}
-\frac{1}{2} & \frac{1}{3} & -\frac{7}{18} & -\frac{7}{18}\\
-\frac{1}{14} & \frac{2}{63} & \frac{1}{18} & -\frac{1}{18}\\
\frac{2}{7} & -\frac{8}{63} & \frac{2}{9} & \frac{1}{3}\\
-\frac{9}{14} & \frac{2}{7} & -\frac{1}{2} & -\frac{1}{2} \\ 
\end{pmatrix}\begin{pmatrix}
63 & -28 & 49 & 49\\
63 & -28 & 49 & 49\\
9 & -4 & 7 & 7\\
-36 & 16 & -28 & -28 \\
\end{pmatrix}\begin{pmatrix}
4 & -7 & -7 & -7\\
9 & 0 & 0 & -7\\
0 & 9 & 0 & -1\\
0 & 0 & 9 & 4 \\ 
\end{pmatrix}=\begin{pmatrix}
0 & 0 & 0 & 0\\
0 & 0 & 0 & 0\\
0 & 0 & 0 & 0\\
0 & 0 & 0 & 14 \\
\end{pmatrix}
$$

A no es diagonalizable por una congruencia-semejanza pues:

$$P^t = \begin{pmatrix}
4 & 9 & 0 & 0\\
-7 & 0 & 9 & 0\\
-7 & 0 & 0 & 9\\
-7 & -7 & -1 & 4 \\ 
\end{pmatrix} \neq 
\begin{pmatrix}
-\frac{1}{2} & \frac{1}{3} & -\frac{7}{18} & -\frac{7}{18}\\
-\frac{1}{14} & \frac{2}{63} & \frac{1}{18} & -\frac{1}{18}\\
\frac{2}{7} & -\frac{8}{63} & \frac{2}{9} & \frac{1}{3}\\
-\frac{9}{14} & \frac{2}{7} & -\frac{1}{2} & -\frac{1}{2} \\ 
\end{pmatrix} = P^{-1}$$


% ---------------------------------------------------------------------------
% FIN DEL DOCUMENTO
% ---------------------------------------------------------------------------

\end{document}