%%%%%%%%%%%%%%%%%%%%%%%%%%%%%%%%%%%%%%%%%%%%%%%%%%%%%%%%%%%%%%%%%%%%%%%%% 
% 
% Apuntes de la asignatura Análisis Matemático II.
% Grado en Estadística.
% Universidad de Granada.
% Curso 2017/18.
% 
% 
% Colaboradores:
% Miguel Anguita Ruiz
% 
% Agradecimientos:
% Andrés Herrera (@andreshp) y Mario Román (@M42) por
% las plantillas base.
% 
% Sitio original:
% https://github.com/libreim/apuntesDGIIM/
% 
% Licencia:
% CC BY-NC-SA 4.0 (https://creativecommons.org/licenses/by-nc-sa/4.0/)
% 
%%%%%%%%%%%%%%%%%%%%%%%%%%%%%%%%%%%%%%%%%%%%%%%%%%%%%%%%%%%%%%%%%%%%%%%%% 


% ------------------------------------------------------------------------------
% ACKNOWLEDGMENTS
% ------------------------------------------------------------------------------

%%%%%%%%%%%%%%%%%%%%%%%%%%%%%%%%%%%%%%%%%%%%%%%%%%%%%%%%%%%%%%%%%%%%%%%% 
% Plantilla básica de Latex en Español.
% 
% Autor: Andrés Herrera Poyatos (https://github.com/andreshp) 
% 
% Es una plantilla básica para redactar documentos. Utiliza el paquete fancyhdr
% para darle un estilo moderno pero serio.
% 
% La plantilla se encuentra adaptada al español.
% 
%%%%%%%%%%%%%%%%%%%%%%%%%%%%%%%%%%%%%%%%%%%%%%%%%%%%%%%%%%%%%%%%%%%%%%%%% 

%%%%%%%%%%%%%%%%%%%%%%%%%%%%%%%%%%%%%%%%%%%%%%%%%%%%%%%%%%%%%%%%%%%%%%%%% 
% Plantilla de Trabajo
% Modificación de una plantilla de Latex de Frits Wenneker para adaptarla 
% al castellano y a las necesidades de escribir informática y matemáticas.
% 
% Editada por: Mario Román
% 
% License:
% CC BY-NC-SA 3.0 (http://creativecommons.org/licenses/by-nc-sa/3.0/)
%%%%%%%%%%%%%%%%%%%%%%%%%%%%%%%%%%%%%%%%%%%%%%%%%%%%%%%%%%%%%%%%%%%%%%%%% 

%%%%%%%%%%%%%%%%%%%%%%%%%%%%%%%%%%%%%%%%%%%%%%%%%%%%%%%%%%%%%%%%%%%%%%%%% 
% Short Sectioned Assignment
% LaTeX Template
% Version 1.0 (5/5/12)
% 
% This template has been downloaded from:
% http://www.LaTeXTemplates.com
% 
% Original author:
% Frits Wenneker (http://www.howtotex.com)
% 
% License:
% CC BY-NC-SA 3.0 (http://creativecommons.org/licenses/by-nc-sa/3.0/)
% 
%%%%%%%%%%%%%%%%%%%%%%%%%%%%%%%%%%%%%%%%%%%%%%%%%%%%%%%%%%%%%%%%%%%%%%%%% 


% Tipo de documento y opciones.
%\documentclass[11pt, a4paper, twoside]{article} % Usar para imprimir
\documentclass[11pt, a4paper]{article}

% ---------------------------------------------------------------------------
% PAQUETES
% ---------------------------------------------------------------------------

% Idioma y codificación para Español.
\usepackage[utf8]{inputenc}
\usepackage[spanish, es-tabla, es-lcroman, es-noquoting]{babel}
\selectlanguage{spanish} 
% \usepackage[T1]{fontenc}

% Fuente utilizada.
\usepackage{courier}    % Fuente Courier.
\usepackage{microtype}  % Mejora la letra final de cara al lector.

% Diseño de página.
\usepackage{fancyhdr}   % Utilizado para hacer cabeceras y pies de página.

\usepackage{titlesec} 	% Utilizado para hacer títulos propios.
\usepackage{lastpage}   % Referencia a la última página.
\usepackage{extramarks} % Marcas extras. Utilizado en pie de página y cabecera.
\usepackage[parfill]{parskip}    % Crea una nueva línea entre párrafos.
\usepackage{geometry}            % Geometría de las páginas.

% Símbolos y matemáticas.
\usepackage{amssymb, amsmath, amsthm, amsfonts, amscd}
\usepackage{upgreek}
\usepackage{mathrsfs}

\usepackage{mdframed}

% Gráficos

\usepackage{pgf,tikz}
\usepackage{tkz-euclide}
\usetkzobj{all}

% Otros.
\usepackage{enumitem}   % Listas mejoradas.
\usepackage{hyperref}
\usepackage{graphicx}   % Gráficos.
\usepackage[space]{grffile}  % Permitir espacios en rutas de gráficos
\usepackage{xcolor}     % Colores.8

% Fuentes personalizadas

\usepackage[scaled=.85]{newpxtext,newpxmath}
\usepackage[scaled=.85]{FiraSans}
\usepackage[T1]{fontenc}

% Código para ajustar las fuentes matemáticas al estilo del texto
% que le rodea

\DeclareMathVersion{sans}
\SetSymbolFont{operators}{sans}{OT1}{cmbr}{m}{n}
\SetSymbolFont{letters}{sans}{OML}{cmbr}{m}{it}
\SetSymbolFont{symbols}{sans}{OMS}{cmbrs}{m}{n}
\SetMathAlphabet{\mathit}{sans}{OT1}{cmbr}{m}{sl}
\SetMathAlphabet{\mathbf}{sans}{OT1}{cmbr}{bx}{n}
\SetMathAlphabet{\mathtt}{sans}{OT1}{cmtl}{m}{n}
\SetSymbolFont{largesymbols}{sans}{OMX}{iwona}{m}{n}

\DeclareMathVersion{boldsans}
\SetSymbolFont{operators}{boldsans}{OT1}{cmbr}{b}{n}
\SetSymbolFont{letters}{boldsans}{OML}{cmbrm}{b}{it}
\SetSymbolFont{symbols}{boldsans}{OMS}{cmbrs}{b}{n}
\SetMathAlphabet{\mathit}{boldsans}{OT1}{cmbr}{b}{sl}
\SetMathAlphabet{\mathbf}{boldsans}{OT1}{cmbr}{bx}{n}
\SetMathAlphabet{\mathtt}{boldsans}{OT1}{cmtl}{b}{n}
\SetSymbolFont{largesymbols}{boldsans}{OMX}{iwona}{bx}{n}

\newif\IfInSansMode
\let\oldsf\sffamily
\renewcommand*{\sffamily}{\oldsf\mathversion{sans}\InSansModetrue}
\let\oldmd\mdseries
\renewcommand*{\mdseries}{\oldmd\IfInSansMode\mathversion{sans}\fi\relax}
\let\oldbf\bfseries
\renewcommand*{\bfseries}{\oldbf\IfInSansMode\mathversion{boldsans}\else%
   \mathversion{bold}\fi\relax}
\let\oldnorm\normalfont
\renewcommand*{\normalfont}{\oldnorm\InSansModefalse\mathversion{normal}}
\let\oldrm\rmfamily
\renewcommand*{\rmfamily}{\oldrm\InSansModefalse\mathversion{normal}}

% Colores

\definecolor{50}{HTML}{E8F5E9}
\definecolor{300}{HTML}{81C784}
\definecolor{500}{HTML}{4CAF50}
\definecolor{700}{HTML}{388E3C}

% ---------------------------------------------------------------------------
% OPCIONES PERSONALIZADAS
% ---------------------------------------------------------------------------

% Formato de texto.
\linespread{1.3}            % Espaciado entre líneas.
\setlength\parindent{0pt}   % No indentar el texto por defecto.
\setlist{leftmargin=.5in}   % Indentación para las listas.

% Estilo de página.
\pagestyle{fancy}
\fancyhf{}
\geometry{left=3cm,right=3cm,top=3cm,bottom=3cm}   % Márgenes y cabecera.

% Estilo de las cabeceras

\titleformat{\section}
  {\Large\bfseries\sffamily}{\thesection}{1em}{}
\titleformat{\subsection}
  {\large\sffamily}{\thesubsection}{1em}{}
\titleformat{\subsubsection}
  {\sffamily}{\thesubsubsection}{1em}{}

% Estilo de enlaces
\hypersetup{
  % hidelinks = true,   % Oculta todos los enlaces.
  colorlinks = true,   % Muestra todos los enlaces, sin bordes alrededor.
  linkcolor=black,     % Color de enlaces genéricos
  citecolor={blue!50!black},   % Color de enlaces de referencias
  urlcolor={blue!80!black}     % Color de enlaces de URL
}

% Ruta donde buscar gráficos
\graphicspath{{../Recursos/Plantillas/} {Recursos/Plantillas/} {./img/} {Análisis Matemático II/img/}}

% Redefinir entorno de demostración (reducir espacio superior)
% \makeatletter
% \renewenvironment{proof}[1][\proofname] {\vspace{-15pt}\par\pushQED{\qed}\normalfont\topsep6\p@\@plus6\p@\relax\trivlist\item[\hskip\labelsep\it#1\@addpunct{.}]\ignorespaces}{\popQED\endtrivlist\@endpefalse}
% \makeatother

% Aumentar el tamaño del interlineado
\linespread{1.3}

% Permitir salto de página en ecuaciones
\allowdisplaybreaks


% ---------------------------------------------------------------------------
% COMANDOS PERSONALIZADOS
% ---------------------------------------------------------------------------

% Redefinir letra griega épsilon.
\let\epsilon\upvarepsilon

% Valor absoluto: \abs{}
\providecommand{\abs}[1]{\lvert#1\rvert}    

% Fracción grande: \ddfrac{}{}
\newcommand\ddfrac[2]{\frac{\displaystyle #1}{\displaystyle #2}}

% Texto en negrita en modo matemática: \bm{}
\newcommand{\bm}[1]{\boldsymbol{#1}}

% Línea horizontal.
\newcommand{\horrule}[1]{\rule{\linewidth}{#1}}

% Letras de conjuntos
\newcommand{\R}{\mathbb{R}} \newcommand{\N}{\mathbb{N}}

% Sucesiones
\newcommand{\xn}{\{x_n\}}
\newcommand{\fn}{\{f_n\}}

% Letra griega "chi" en línea con el texto
\DeclareRobustCommand{\rchi}{{\Large \mathpalette\irchi\relax}}
\newcommand{\irchi}[2]{\raisebox{0.4\depth}{$#1\chi$}} % inner command, used by \rchi 

% Letra 'omega'
\newcommand{\W}{\Omega}
\newcommand{\w}{\omega}


% ---------------------------------------------------------------------------
% CABECERA Y PIE DE PÁGINA
% ---------------------------------------------------------------------------

\renewcommand{\sectionmark}[1]{%
\markboth{#1}{}}

\renewcommand{\subsectionmark}[1]{%
\markright{#1}{}}

%\addtolength{\headheight}{4ex}
\renewcommand{\headrulewidth}{0pt}

\addtolength{\headwidth}{\marginparsep}
%\addtolength{\headwidth}{\marginparwidth}

\fancypagestyle{section}{%
  \fancyhead{}
  %\addtolength{\headheight}{-10ex}
  %\renewcommand{\headheight}{0pt}%
  %\setlength{\footskip}{-48pt}%
  \fancyfoot[LE,RO]{\Large\sffamily\thepage}
  \renewcommand{\headrulewidth}{0pt}%
  \renewcommand{\footrulewidth}{0pt}%
}

\let\originalsection\section
\RenewDocumentCommand{\section}{som}{%
  \IfBooleanTF{#1}
    {\originalsection*{#3}}
    {\IfNoValueTF{#2}
      {\originalsection{#3}}
      {\originalsection[#2]{#3}}%
    }%
  \thispagestyle{section}%
}

\fancyhead[LE,RO]{\rule[-4ex]{0pt}{2ex}\sffamily\textsl{\rightmark}}
\fancyhead[LO,RE]{\sffamily{\leftmark}}
\fancyfoot[LE,RO]{\Large\sffamily\thepage}



% ---------------------------------------------------------------------------
% ENTORNOS PARA MATEMÁTICAS
% ---------------------------------------------------------------------------

% Nuevo estilo para definiciones.
\newtheoremstyle{definition-style} % Nombre del estilo.
{}               % Espacio por encima.
{}               % Espacio por debajo.
{}                   % Fuente del cuerpo.
{}                   % Identación.
{\bf\sffamily}                % Fuente para la cabecera.
{.}                  % Puntuación tras la cabecera.
{.5em}               % Espacio tras la cabecera.
{\thmname{#1}\thmnumber{ #2}\thmnote{ (#3)}}     % Especificación de la cabecera (actual: nombre en negrita).

% Nuevo estilo para notas.
\newtheoremstyle{remark-style} 
{10pt}                
{10pt}                
{}                   
{}                   
{\itshape \sffamily}          
{.}                  
{.5em}               
{}                  

% Nuevo estilo para teoremas y proposiciones.
\newtheoremstyle{theorem-style}
{}                
{}                
{}           
{}                  
{\bfseries \sffamily}             
{.}                
{.5em}           
{\thmname{#1}\thmnumber{ #2}\thmnote{ (#3)}}

% Nuevo estilo para ejemplos.
\newtheoremstyle{example-style}
{10pt}                
{10pt}                
{}                  
{}                   
{\bf \sffamily}              
{}                 
{.5em}               
{\thmname{#1}\thmnumber{ #2.}\thmnote{ #3.}}  

% Nuevo estilo para la demostración
%\renewenvironment{proof}{{\itshape \sffamily Demostración \\}}{\hspace*{\fill}\qed}

\makeatletter
\renewenvironment{proof}[1][\proofname] {\par\pushQED{\qed}\normalfont\topsep6\p@\@plus6\p@\relax\trivlist\item[\hskip\labelsep\itshape\sffamily#1\@addpunct{.}]\ignorespaces}{\popQED\endtrivlist\@endpefalse}
\makeatother

% Configuración general de mdframe, los estilos de los teoremas, etc
\mdfsetup{skipabove=12pt,skipbelow=12pt,innertopmargin=12pt,innerbottommargin=4pt}

% Creamos los 'marcos' de los estilos

\mdfdefinestyle{nth-frame}{
	linewidth=2pt, %
	linecolor= 500, % 
	%linecolor=black,
	topline=false, %
	bottomline=false, %
	rightline=false,%
	leftmargin=0pt, %
	innerleftmargin=1em, % 
	rightmargin=0pt, %
	innerrightmargin=0pt, % % 
	%splittopskip=\topskip, %
	%splitbottomskip=\topskip, %
}% 

\mdfdefinestyle{nprop-frame}{
	linewidth=2pt, %
	linecolor= 300, %
	%linecolor= gray, 
	topline=false, %
	bottomline=false, %
	rightline=false,%
	leftmargin=0pt, %
	innerleftmargin=1em, %
	innerrightmargin=0em, 
	rightmargin=0pt, %
	%splittopskip=\topskip, %
	%splitbottomskip=\topskip, %
}%       

\mdfdefinestyle{ndef-frame}{
	linewidth=2pt, %
	linecolor= 300, % 
	%linecolor= gray!50,
	backgroundcolor= 50,
	%backgroundcolor= gray!5,
	topline=false, %
	bottomline=false, %
	rightline=false,%
	leftmargin=0pt, %
	innerleftmargin=1em, %
	innerrightmargin=1em, 
	rightmargin=0pt, % 
	innertopmargin=1.5em,%
	innerbottommargin=1em, % 
	splittopskip=\topskip, %
	%splitbottomskip=\topskip, %
}% 

\mdfdefinestyle{ejemplo-frame}{
	linewidth=0pt, %
	linecolor= 300, % 
	%backgroundcolor= 50,
	leftline=false, %
	rightline=false, %
	leftmargin=0pt, %
	innerleftmargin=1.5em, %
	innerrightmargin=1.5em, 
	rightmargin=0pt, % 
	innertopmargin=0em,%
	innerbottommargin=0em, % 
	splittopskip=\topskip, %
	%splitbottomskip=\topskip, %
}%                

% Asignamos los marcos a los estilos

\surroundwithmdframed[style=nth-frame]{nth}
\surroundwithmdframed[style=nprop-frame]{nprop}
\surroundwithmdframed[style=nprop-frame]{ncor}
\surroundwithmdframed[style=ndef-frame]{ndef}
\surroundwithmdframed[style=ejemplo-frame]{ejemplo}
                 
% Asignamos los estilos definidos anteriormente a los entornos correspondientes
% Teoremas, proposiciones y corolarios.
\theoremstyle{theorem-style}
\newtheorem{nth}{Teorema}[section]
\newtheorem{nprop}{Proposición}[section]
\newtheorem{ncor}{Corolario}[section]
\newtheorem{lema}{Lema}[section]
% Definiciones.
\theoremstyle{definition-style}
\newtheorem{ndef}{Definición}[section]

% Notas.
\theoremstyle{remark-style}
\newtheorem*{nota}{Nota}

% Ejemplos.
\theoremstyle{example-style}
\newtheorem{ejemplo}{Ejemplo}[section]



% Listas ordenadas con números romanos (i), (ii), etc.
\newenvironment{nlist}
{\begin{enumerate}
    \renewcommand\labelenumi{(\emph{\roman{enumi})}}}
  {\end{enumerate}}

% División por casos con llave a la derecha.
\newenvironment{rcases}
{\left.\begin{aligned}}
    {\end{aligned}\right\rbrace}


% ---------------------------------------------------------------------------
% PÁGINA DE TÍTULO
% ---------------------------------------------------------------------------

% Título del documento.
\newcommand{\subject}{Álgebra: ejercicio 8}

% Autor del documento.
\newcommand{\docauthor}{Miguel Anguita Ruiz}

% Título
\title{
  \normalfont \normalsize 
  \textsc{Universidad de Granada} \\ [25pt]    % Texto por encima.
  \horrule{0.5pt} \\[0.4cm] % Línea horizontal fina.
  \huge \sffamily\subject\\ % Título.
  \horrule{2pt} \\[0.5cm] % Línea horizontal gruesa.
}

% Autor.
\author{\Large\sffamily{\docauthor}}

% Fecha.
\date{\vspace{-1.5em} \normalsize \sffamily Curso 2017/18}



% ---------------------------------------------------------------------------
% COMIENZO DEL DOCUMENTO
% ---------------------------------------------------------------------------

\begin{document}

\maketitle  % Título.
\vfill
\begin{center}
  %\includegraphics{by-nc-sa.pdf}  % Licencia.
\end{center}
\newpage
\tableofcontents    % Índice
\newpage


% --------------------------------------------------------------------------------
% Introducción.
% --------------------------------------------------------------------------------

\section{Enunciado}
Dada la aplicación $f: \R^4 \to \R^3$, definida por: 
$$f(x,y,z,t)=(7x-7t,x-4z,9x-4y+7z+7t)$$
	
i) Halla sus ecuaciones matriciales y comprueba que es una a.l. \\
ii) Amplía su matriz con la identidad por debajo y halla su núcleo y su imagen, aplicando transformaciones elementales de columnas. \\
iii) ¿ Cuánto vale la dimensión de tu núcleo mas la dimensión de tu imagen ?. \\

\section{Solución}

Su matriz asociada por columnas, respecto a las bases canónicas, es:

$$\begin{pmatrix}
7 & 0 & 0 & -7\\
1 & 0 & -4 & 0\\
9 & -4 & 7 & 7\\
\end{pmatrix}$$

La aplicación lineal es equivalente al siguiente producto matricial y así definimos las ecuaciones matriciales:

$$\begin{pmatrix}
7 & 0 & 0 & -7\\
1 & 0 & -4 & 0\\
9 & -4 & 7 & 7\\
\end{pmatrix}
\begin{pmatrix}
x \\
y \\
z \\
t \\
\end{pmatrix} = 
\begin{pmatrix}
7x-7t \\
x-4z \\
9x-4y+7z+7t \\
\end{pmatrix}$$

Sabemos que $f$ es una aplicación lineal pues se comporta bien para la suma y el producto externo, además es evidente que es lineal pues todas sus incógnitas tienen grado uno.

A continuación, ampliamos por debajo con la identidad y realizaré transformaciones elementales a la matriz original y a la matriz identidad.

Multiplico la matriz original y la identidad por $E_23$ para intercambiar la segunda y la tercera columna, con el objetivo de no tener ceros en la diagonal principal.

$$\begin{pmatrix}
7 & 0 & 0 & -7\\
1 & 0 & -4 & 0\\
9 & -4 & 7 & 7\\
1 & 0 & 0 & 0\\
0 & 1 & 0 & 0\\
0 & 0 & 1 & 0\\
0 & 0 & 0 & 1\\
\end{pmatrix} \Longrightarrow
\begin{pmatrix}
7 & 0 & 0 & -7\\
1 & -4 & 0 & 0\\
9 & 7 & -4 & 7\\
1 & 0 & 0 & 0\\
0 & 0 & 1 & 0\\
0 & 1 & 0 & 0\\
0 & 0 & 0 & 1\\
\end{pmatrix}$$

Multiplico ahora por $E_{23}(\frac{7}{4})$ y por $E_{13}(\frac{9}{4})$ para hacer dos ceros a la izquierda del tercer pivote. \\

$$\begin{pmatrix}
7 & 0 & 0 & -7\\
1 & -4 & 0 & 0\\
9 & 7 & -4 & 7\\
1 & 0 & 0 & 0\\
0 & 0 & 1 & 0\\
0 & 1 & 0 & 0\\
0 & 0 & 0 & 1\\
\end{pmatrix} \Longrightarrow
\begin{pmatrix}
7 & 0 & 0 & -7\\
1 & -4 & 0 & 0\\
9 & 0 & -4 & 7\\
1 & 0 & 0 & 0\\
0 & \frac{7}{4} & 1 & 0\\
0 & 1 & 0 & 0\\
0 & 0 & 0 & 1\\
\end{pmatrix} \Longrightarrow
\begin{pmatrix}
7 & 0 & 0 & -7\\
1 & -4 & 0 & 0\\
0 & 0 & -4 & 7\\
1 & 0 & 0 & 0\\
\frac{9}{4} & \frac{7}{4} & 1 & 0\\
0 & 1 & 0 & 0\\
0 & 0 & 0 & 1\\
\end{pmatrix}$$

Multiplico ahora por $E_{12}(\frac{1}{4})$ para hacer un cero a la izquierda del segundo pivote. \\

$$\begin{pmatrix}
7 & 0 & 0 & -7\\
1 & -4 & 0 & 0\\
0 & 0 & -4 & 7\\
1 & 0 & 0 & 0\\
\frac{9}{4} & \frac{7}{4} & 1 & 0\\
0 & 1 & 0 & 0\\
0 & 0 & 0 & 1\\
\end{pmatrix} \Longrightarrow 
\begin{pmatrix}
7 & 0 & 0 & -7\\
0 & -4 & 0 & 0\\
0 & 0 & -4 & 7\\
1 & 0 & 0 & 0\\
\frac{43}{16} & \frac{7}{4} & 1 & 0\\
\frac{1}{4} & 1 & 0 & 0\\
0 & 0 & 0 & 1\\
\end{pmatrix}$$

Multiplico por $E_41(1)$ y por $E_43(\frac{7}{4})$ para hacer otros dos ceros en la cuarta columna. \\

$$\begin{pmatrix}
7 & 0 & 0 & -7\\
0 & -4 & 0 & 0\\
0 & 0 & -4 & 7\\
1 & 0 & 0 & 0\\
\frac{43}{16} & \frac{7}{4} & 1 & 0\\
\frac{1}{4} & 1 & 0 & 0\\
0 & 0 & 0 & 1\\
\end{pmatrix} \Longrightarrow
\begin{pmatrix}
7 & 0 & 0 & 0\\
0 & -4 & 0 & 0\\
0 & 0 & -4 & 0\\
1 & 0 & 0 & 1\\
\frac{43}{16} & \frac{7}{4} & 1 & \frac{71}{16}\\
\frac{1}{4} & 1 & 0 & \frac{1}{4}\\
0 & 0 & 0 & 1\\
\end{pmatrix}$$

Finalmente, multiplico por $E_1(\frac{1}{7})$, por $E_2(-\frac{1}{4})$ y por $E_3(-\frac{1}{4})$ para obtener unos en la diagonal principal.

$$\begin{pmatrix}
7 & 0 & 0 & 0\\
0 & -4 & 0 & 0\\
0 & 0 & -4 & 0\\
1 & 0 & 0 & 1\\
\frac{43}{16} & \frac{7}{4} & 1 & \frac{71}{16}\\
\frac{1}{4} & 1 & 0 & \frac{1}{4}\\
0 & 0 & 0 & 1\\
\end{pmatrix} \Longrightarrow
\begin{pmatrix}
1 & 0 & 0 & 0\\
0 & 1 & 0 & 0\\
0 & 0 & 1 & 0\\
\frac{1}{7} & 0 & 0 & 1\\
\frac{43}{112} & \frac{-7}{16} & -\frac{1}{4} & \frac{71}{16}\\
\frac{1}{28} & -\frac{1}{4} & 0 & \frac{1}{4}\\
0 & 0 & 0 & 1\\
\end{pmatrix}$$

Hemos terminado el algoritmo por columnas y la nueva matriz C, formada por las tres primeras filas, es una forma normal de Hermite por columnas.

Además, obtenemos una matriz de cambio Q tal que

$$C = AQ = 
\begin{pmatrix}
7 & 0 & 0 & -7\\
1 & 0 & -4 & 0\\
9 & -4 & 7 & 7\\
\end{pmatrix}
\begin{pmatrix}
\frac{1}{7} & 0 & 0 & 1\\
\frac{43}{112} & \frac{-7}{16} & -\frac{1}{4} & \frac{71}{16}\\
\frac{1}{28} & -\frac{1}{4} & 0 & \frac{1}{4}\\
0 & 0 & 0 & 1\\
\end{pmatrix} = 
\begin{pmatrix}
1 & 0 & 0 & 0\\
0 & 1 & 0 & 0\\
0 & 0 & 1 & 0\\
\end{pmatrix} \Longrightarrow 
\begin{pmatrix}
7 & 0 & 0 & -7\\
1 & 0 & -4 & 0\\
9 & -4 & 7 & 7\\
\end{pmatrix}
\begin{pmatrix}
1\\
\frac{71}{16}\\
\frac{1}{4}\\
1\\
\end{pmatrix} =
\begin{pmatrix}
0\\
0\\
0\\
\end{pmatrix}
$$

Observamos que las 4 columnas de la matriz Q son l.i. y forman una base de $\R^4$ (el espacio origen).

Por tanto si $u = \lambda_1u_1+\lambda_2u_2+\lambda_3u_3+\lambda_4u_4 \in \R^4$ se tiene:

$$f(u)=\lambda_1f(u_1)+\lambda_2f(u_2)+\lambda_3f(u_3)+\lambda_4f(u_4) = \lambda_1c_1+\lambda_2c_2+\lambda_3c_3$$

ya que hemos visto que $u_4 = (1,\frac{71}{16},\frac{1}{4},1)$ satisface $f(u_4)= A*u_4=0$.

Ahora, como las columnas  $c_1 = (1,0,0), c_2 = (0,1,0), c_3 = (0,0,1)$ de C son l.i se tiene que $f(u)=\lambda_1c_1+\lambda_2c_2+\lambda_3c_3 = 0 \Leftrightarrow \lambda_1=\lambda_2=0$ y por tanto que $u \in N(f) \Leftrightarrow u=\lambda_4u_4$.

O sea, hemos demostrado que una base de N(f) está formada por la columna $u_4=(1,\frac{71}{16},\frac{1}{4},1)$, luego N(f) tiene dimensión 1.

Finalmente, una base de la imagen está formada por las columnas $c_1, c_2, c_3$ de C ya que son un s.g. de la imagen y son l.i.. La dimensión de la imagen es 3.

$$B_{Im(f)}=\{(1,0,0),(0,1,0),(0,0,1)\}$$

En realidad, lo que hemos hecho es un cambio de base, $u'= Q*u$, en el espacio origen $\R^4$, de forma que la nueva base es una ampliación de una base del núcleo y por tanto las ecuaciones salen más sencillas.

$$f(u)=A*u=A*Q*u'=C*u'$$

Finalmente, observamos que $dim(N(f))+dim(Im(f))=1+3=4$, que es la dimensión del espacio origen $\R^4$.

% ---------------------------------------------------------------------------
% FIN DEL DOCUMENTO
% ---------------------------------------------------------------------------

\end{document}