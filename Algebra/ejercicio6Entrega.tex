%%%%%%%%%%%%%%%%%%%%%%%%%%%%%%%%%%%%%%%%%%%%%%%%%%%%%%%%%%%%%%%%%%%%%%%%% 
% 
% Apuntes de la asignatura Análisis Matemático II.
% Grado en Estadística.
% Universidad de Granada.
% Curso 2017/18.
% 
% 
% Colaboradores:
% Miguel Anguita Ruiz
% 
% Agradecimientos:
% Andrés Herrera (@andreshp) y Mario Román (@M42) por
% las plantillas base.
% 
% Sitio original:
% https://github.com/libreim/apuntesDGIIM/
% 
% Licencia:
% CC BY-NC-SA 4.0 (https://creativecommons.org/licenses/by-nc-sa/4.0/)
% 
%%%%%%%%%%%%%%%%%%%%%%%%%%%%%%%%%%%%%%%%%%%%%%%%%%%%%%%%%%%%%%%%%%%%%%%%% 


% ------------------------------------------------------------------------------
% ACKNOWLEDGMENTS
% ------------------------------------------------------------------------------

%%%%%%%%%%%%%%%%%%%%%%%%%%%%%%%%%%%%%%%%%%%%%%%%%%%%%%%%%%%%%%%%%%%%%%%% 
% Plantilla básica de Latex en Español.
% 
% Autor: Andrés Herrera Poyatos (https://github.com/andreshp) 
% 
% Es una plantilla básica para redactar documentos. Utiliza el paquete fancyhdr
% para darle un estilo moderno pero serio.
% 
% La plantilla se encuentra adaptada al español.
% 
%%%%%%%%%%%%%%%%%%%%%%%%%%%%%%%%%%%%%%%%%%%%%%%%%%%%%%%%%%%%%%%%%%%%%%%%% 

%%%%%%%%%%%%%%%%%%%%%%%%%%%%%%%%%%%%%%%%%%%%%%%%%%%%%%%%%%%%%%%%%%%%%%%%% 
% Plantilla de Trabajo
% Modificación de una plantilla de Latex de Frits Wenneker para adaptarla 
% al castellano y a las necesidades de escribir informática y matemáticas.
% 
% Editada por: Mario Román
% 
% License:
% CC BY-NC-SA 3.0 (http://creativecommons.org/licenses/by-nc-sa/3.0/)
%%%%%%%%%%%%%%%%%%%%%%%%%%%%%%%%%%%%%%%%%%%%%%%%%%%%%%%%%%%%%%%%%%%%%%%%% 

%%%%%%%%%%%%%%%%%%%%%%%%%%%%%%%%%%%%%%%%%%%%%%%%%%%%%%%%%%%%%%%%%%%%%%%%% 
% Short Sectioned Assignment
% LaTeX Template
% Version 1.0 (5/5/12)
% 
% This template has been downloaded from:
% http://www.LaTeXTemplates.com
% 
% Original author:
% Frits Wenneker (http://www.howtotex.com)
% 
% License:
% CC BY-NC-SA 3.0 (http://creativecommons.org/licenses/by-nc-sa/3.0/)
% 
%%%%%%%%%%%%%%%%%%%%%%%%%%%%%%%%%%%%%%%%%%%%%%%%%%%%%%%%%%%%%%%%%%%%%%%%% 


% Tipo de documento y opciones.
%\documentclass[11pt, a4paper, twoside]{article} % Usar para imprimir
\documentclass[11pt, a4paper]{article}

% ---------------------------------------------------------------------------
% PAQUETES
% ---------------------------------------------------------------------------

% Idioma y codificación para Español.
\usepackage[utf8]{inputenc}
\usepackage[spanish, es-tabla, es-lcroman, es-noquoting]{babel}
\selectlanguage{spanish} 
% \usepackage[T1]{fontenc}

% Fuente utilizada.
\usepackage{courier}    % Fuente Courier.
\usepackage{microtype}  % Mejora la letra final de cara al lector.

% Diseño de página.
\usepackage{fancyhdr}   % Utilizado para hacer cabeceras y pies de página.

\usepackage{titlesec} 	% Utilizado para hacer títulos propios.
\usepackage{lastpage}   % Referencia a la última página.
\usepackage{extramarks} % Marcas extras. Utilizado en pie de página y cabecera.
\usepackage[parfill]{parskip}    % Crea una nueva línea entre párrafos.
\usepackage{geometry}            % Geometría de las páginas.

% Símbolos y matemáticas.
\usepackage{amssymb, amsmath, amsthm, amsfonts, amscd}
\usepackage{upgreek}
\usepackage{mathrsfs}

\usepackage{mdframed}

% Gráficos

\usepackage{pgf,tikz}
\usepackage{tkz-euclide}
\usetkzobj{all}

% Otros.
\usepackage{enumitem}   % Listas mejoradas.
\usepackage{hyperref}
\usepackage{graphicx}   % Gráficos.
\usepackage[space]{grffile}  % Permitir espacios en rutas de gráficos
\usepackage{xcolor}     % Colores.8

% Fuentes personalizadas

\usepackage[scaled=.85]{newpxtext,newpxmath}
\usepackage[scaled=.85]{FiraSans}
\usepackage[T1]{fontenc}

% Código para ajustar las fuentes matemáticas al estilo del texto
% que le rodea

\DeclareMathVersion{sans}
\SetSymbolFont{operators}{sans}{OT1}{cmbr}{m}{n}
\SetSymbolFont{letters}{sans}{OML}{cmbr}{m}{it}
\SetSymbolFont{symbols}{sans}{OMS}{cmbrs}{m}{n}
\SetMathAlphabet{\mathit}{sans}{OT1}{cmbr}{m}{sl}
\SetMathAlphabet{\mathbf}{sans}{OT1}{cmbr}{bx}{n}
\SetMathAlphabet{\mathtt}{sans}{OT1}{cmtl}{m}{n}
\SetSymbolFont{largesymbols}{sans}{OMX}{iwona}{m}{n}

\DeclareMathVersion{boldsans}
\SetSymbolFont{operators}{boldsans}{OT1}{cmbr}{b}{n}
\SetSymbolFont{letters}{boldsans}{OML}{cmbrm}{b}{it}
\SetSymbolFont{symbols}{boldsans}{OMS}{cmbrs}{b}{n}
\SetMathAlphabet{\mathit}{boldsans}{OT1}{cmbr}{b}{sl}
\SetMathAlphabet{\mathbf}{boldsans}{OT1}{cmbr}{bx}{n}
\SetMathAlphabet{\mathtt}{boldsans}{OT1}{cmtl}{b}{n}
\SetSymbolFont{largesymbols}{boldsans}{OMX}{iwona}{bx}{n}

\newif\IfInSansMode
\let\oldsf\sffamily
\renewcommand*{\sffamily}{\oldsf\mathversion{sans}\InSansModetrue}
\let\oldmd\mdseries
\renewcommand*{\mdseries}{\oldmd\IfInSansMode\mathversion{sans}\fi\relax}
\let\oldbf\bfseries
\renewcommand*{\bfseries}{\oldbf\IfInSansMode\mathversion{boldsans}\else%
   \mathversion{bold}\fi\relax}
\let\oldnorm\normalfont
\renewcommand*{\normalfont}{\oldnorm\InSansModefalse\mathversion{normal}}
\let\oldrm\rmfamily
\renewcommand*{\rmfamily}{\oldrm\InSansModefalse\mathversion{normal}}

% Colores

\definecolor{50}{HTML}{E8F5E9}
\definecolor{300}{HTML}{81C784}
\definecolor{500}{HTML}{4CAF50}
\definecolor{700}{HTML}{388E3C}

% ---------------------------------------------------------------------------
% OPCIONES PERSONALIZADAS
% ---------------------------------------------------------------------------

% Formato de texto.
\linespread{1.3}            % Espaciado entre líneas.
\setlength\parindent{0pt}   % No indentar el texto por defecto.
\setlist{leftmargin=.5in}   % Indentación para las listas.

% Estilo de página.
\pagestyle{fancy}
\fancyhf{}
\geometry{left=3cm,right=3cm,top=3cm,bottom=3cm}   % Márgenes y cabecera.

% Estilo de las cabeceras

\titleformat{\section}
  {\Large\bfseries\sffamily}{\thesection}{1em}{}
\titleformat{\subsection}
  {\large\sffamily}{\thesubsection}{1em}{}
\titleformat{\subsubsection}
  {\sffamily}{\thesubsubsection}{1em}{}

% Estilo de enlaces
\hypersetup{
  % hidelinks = true,   % Oculta todos los enlaces.
  colorlinks = true,   % Muestra todos los enlaces, sin bordes alrededor.
  linkcolor=black,     % Color de enlaces genéricos
  citecolor={blue!50!black},   % Color de enlaces de referencias
  urlcolor={blue!80!black}     % Color de enlaces de URL
}

% Ruta donde buscar gráficos
\graphicspath{{../Recursos/Plantillas/} {Recursos/Plantillas/} {./img/} {Análisis Matemático II/img/}}

% Redefinir entorno de demostración (reducir espacio superior)
% \makeatletter
% \renewenvironment{proof}[1][\proofname] {\vspace{-15pt}\par\pushQED{\qed}\normalfont\topsep6\p@\@plus6\p@\relax\trivlist\item[\hskip\labelsep\it#1\@addpunct{.}]\ignorespaces}{\popQED\endtrivlist\@endpefalse}
% \makeatother

% Aumentar el tamaño del interlineado
\linespread{1.3}

% Permitir salto de página en ecuaciones
\allowdisplaybreaks


% ---------------------------------------------------------------------------
% COMANDOS PERSONALIZADOS
% ---------------------------------------------------------------------------

% Redefinir letra griega épsilon.
\let\epsilon\upvarepsilon

% Valor absoluto: \abs{}
\providecommand{\abs}[1]{\lvert#1\rvert}    

% Fracción grande: \ddfrac{}{}
\newcommand\ddfrac[2]{\frac{\displaystyle #1}{\displaystyle #2}}

% Texto en negrita en modo matemática: \bm{}
\newcommand{\bm}[1]{\boldsymbol{#1}}

% Línea horizontal.
\newcommand{\horrule}[1]{\rule{\linewidth}{#1}}

% Letras de conjuntos
\newcommand{\R}{\mathbb{R}} \newcommand{\N}{\mathbb{N}}

% Sucesiones
\newcommand{\xn}{\{x_n\}}
\newcommand{\fn}{\{f_n\}}

% Letra griega "chi" en línea con el texto
\DeclareRobustCommand{\rchi}{{\Large \mathpalette\irchi\relax}}
\newcommand{\irchi}[2]{\raisebox{0.4\depth}{$#1\chi$}} % inner command, used by \rchi 

% Letra 'omega'
\newcommand{\W}{\Omega}
\newcommand{\w}{\omega}


% ---------------------------------------------------------------------------
% CABECERA Y PIE DE PÁGINA
% ---------------------------------------------------------------------------

\renewcommand{\sectionmark}[1]{%
\markboth{#1}{}}

\renewcommand{\subsectionmark}[1]{%
\markright{#1}{}}

%\addtolength{\headheight}{4ex}
\renewcommand{\headrulewidth}{0pt}

\addtolength{\headwidth}{\marginparsep}
%\addtolength{\headwidth}{\marginparwidth}

\fancypagestyle{section}{%
  \fancyhead{}
  %\addtolength{\headheight}{-10ex}
  %\renewcommand{\headheight}{0pt}%
  %\setlength{\footskip}{-48pt}%
  \fancyfoot[LE,RO]{\Large\sffamily\thepage}
  \renewcommand{\headrulewidth}{0pt}%
  \renewcommand{\footrulewidth}{0pt}%
}

\let\originalsection\section
\RenewDocumentCommand{\section}{som}{%
  \IfBooleanTF{#1}
    {\originalsection*{#3}}
    {\IfNoValueTF{#2}
      {\originalsection{#3}}
      {\originalsection[#2]{#3}}%
    }%
  \thispagestyle{section}%
}

\fancyhead[LE,RO]{\rule[-4ex]{0pt}{2ex}\sffamily\textsl{\rightmark}}
\fancyhead[LO,RE]{\sffamily{\leftmark}}
\fancyfoot[LE,RO]{\Large\sffamily\thepage}



% ---------------------------------------------------------------------------
% ENTORNOS PARA MATEMÁTICAS
% ---------------------------------------------------------------------------

% Nuevo estilo para definiciones.
\newtheoremstyle{definition-style} % Nombre del estilo.
{}               % Espacio por encima.
{}               % Espacio por debajo.
{}                   % Fuente del cuerpo.
{}                   % Identación.
{\bf\sffamily}                % Fuente para la cabecera.
{.}                  % Puntuación tras la cabecera.
{.5em}               % Espacio tras la cabecera.
{\thmname{#1}\thmnumber{ #2}\thmnote{ (#3)}}     % Especificación de la cabecera (actual: nombre en negrita).

% Nuevo estilo para notas.
\newtheoremstyle{remark-style} 
{10pt}                
{10pt}                
{}                   
{}                   
{\itshape \sffamily}          
{.}                  
{.5em}               
{}                  

% Nuevo estilo para teoremas y proposiciones.
\newtheoremstyle{theorem-style}
{}                
{}                
{}           
{}                  
{\bfseries \sffamily}             
{.}                
{.5em}           
{\thmname{#1}\thmnumber{ #2}\thmnote{ (#3)}}

% Nuevo estilo para ejemplos.
\newtheoremstyle{example-style}
{10pt}                
{10pt}                
{}                  
{}                   
{\bf \sffamily}              
{}                 
{.5em}               
{\thmname{#1}\thmnumber{ #2.}\thmnote{ #3.}}  

% Nuevo estilo para la demostración
%\renewenvironment{proof}{{\itshape \sffamily Demostración \\}}{\hspace*{\fill}\qed}

\makeatletter
\renewenvironment{proof}[1][\proofname] {\par\pushQED{\qed}\normalfont\topsep6\p@\@plus6\p@\relax\trivlist\item[\hskip\labelsep\itshape\sffamily#1\@addpunct{.}]\ignorespaces}{\popQED\endtrivlist\@endpefalse}
\makeatother

% Configuración general de mdframe, los estilos de los teoremas, etc
\mdfsetup{skipabove=12pt,skipbelow=12pt,innertopmargin=12pt,innerbottommargin=4pt}

% Creamos los 'marcos' de los estilos

\mdfdefinestyle{nth-frame}{
	linewidth=2pt, %
	linecolor= 500, % 
	%linecolor=black,
	topline=false, %
	bottomline=false, %
	rightline=false,%
	leftmargin=0pt, %
	innerleftmargin=1em, % 
	rightmargin=0pt, %
	innerrightmargin=0pt, % % 
	%splittopskip=\topskip, %
	%splitbottomskip=\topskip, %
}% 

\mdfdefinestyle{nprop-frame}{
	linewidth=2pt, %
	linecolor= 300, %
	%linecolor= gray, 
	topline=false, %
	bottomline=false, %
	rightline=false,%
	leftmargin=0pt, %
	innerleftmargin=1em, %
	innerrightmargin=0em, 
	rightmargin=0pt, %
	%splittopskip=\topskip, %
	%splitbottomskip=\topskip, %
}%       

\mdfdefinestyle{ndef-frame}{
	linewidth=2pt, %
	linecolor= 300, % 
	%linecolor= gray!50,
	backgroundcolor= 50,
	%backgroundcolor= gray!5,
	topline=false, %
	bottomline=false, %
	rightline=false,%
	leftmargin=0pt, %
	innerleftmargin=1em, %
	innerrightmargin=1em, 
	rightmargin=0pt, % 
	innertopmargin=1.5em,%
	innerbottommargin=1em, % 
	splittopskip=\topskip, %
	%splitbottomskip=\topskip, %
}% 

\mdfdefinestyle{ejemplo-frame}{
	linewidth=0pt, %
	linecolor= 300, % 
	%backgroundcolor= 50,
	leftline=false, %
	rightline=false, %
	leftmargin=0pt, %
	innerleftmargin=1.5em, %
	innerrightmargin=1.5em, 
	rightmargin=0pt, % 
	innertopmargin=0em,%
	innerbottommargin=0em, % 
	splittopskip=\topskip, %
	%splitbottomskip=\topskip, %
}%                

% Asignamos los marcos a los estilos

\surroundwithmdframed[style=nth-frame]{nth}
\surroundwithmdframed[style=nprop-frame]{nprop}
\surroundwithmdframed[style=nprop-frame]{ncor}
\surroundwithmdframed[style=ndef-frame]{ndef}
\surroundwithmdframed[style=ejemplo-frame]{ejemplo}
                 
% Asignamos los estilos definidos anteriormente a los entornos correspondientes
% Teoremas, proposiciones y corolarios.
\theoremstyle{theorem-style}
\newtheorem{nth}{Teorema}[section]
\newtheorem{nprop}{Proposición}[section]
\newtheorem{ncor}{Corolario}[section]
\newtheorem{lema}{Lema}[section]
% Definiciones.
\theoremstyle{definition-style}
\newtheorem{ndef}{Definición}[section]

% Notas.
\theoremstyle{remark-style}
\newtheorem*{nota}{Nota}

% Ejemplos.
\theoremstyle{example-style}
\newtheorem{ejemplo}{Ejemplo}[section]



% Listas ordenadas con números romanos (i), (ii), etc.
\newenvironment{nlist}
{\begin{enumerate}
    \renewcommand\labelenumi{(\emph{\roman{enumi})}}}
  {\end{enumerate}}

% División por casos con llave a la derecha.
\newenvironment{rcases}
{\left.\begin{aligned}}
    {\end{aligned}\right\rbrace}


% ---------------------------------------------------------------------------
% PÁGINA DE TÍTULO
% ---------------------------------------------------------------------------

% Título del documento.
\newcommand{\subject}{Álgebra: ejercicio 6}

% Autor del documento.
\newcommand{\docauthor}{Miguel Anguita Ruiz}

% Título
\title{
  \normalfont \normalsize 
  \textsc{Universidad de Granada} \\ [25pt]    % Texto por encima.
  \horrule{0.5pt} \\[0.4cm] % Línea horizontal fina.
  \huge \sffamily\subject\\ % Título.
  \horrule{2pt} \\[0.5cm] % Línea horizontal gruesa.
}

% Autor.
\author{\Large\sffamily{\docauthor}}

% Fecha.
\date{\vspace{-1.5em} \normalsize \sffamily Curso 2017/18}



% ---------------------------------------------------------------------------
% COMIENZO DEL DOCUMENTO
% ---------------------------------------------------------------------------

\begin{document}

\maketitle  % Título.
\vfill
\begin{center}
  %\includegraphics{by-nc-sa.pdf}  % Licencia.
\end{center}
\newpage
\tableofcontents    % Índice
\newpage


% --------------------------------------------------------------------------------
% Introducción.
% --------------------------------------------------------------------------------

\section{Enunciado}
El DNI usado para el ejercicio es el siguiente: 77149477. \\

Dada la matriz 
$A = \begin{pmatrix}
7 & 7 & 1 & 4 \\
-9 & 4 & -7 & 7 \\
\end{pmatrix}$ 
	
i) Usa una base ampliada de su espacio de filas para encontrar una
factorización de A. \\
ii) Usa una base ampliada de su espacio de columnas para encontrar una factorización de A. \\
iii) Amplía el espacio nulo de la matriz hasta una base de $\R^4$. Usa esta base y la de su espacio de columnas para encontrar una factorización de A. \\

\section{Solución}

Para la matriz A, su espacio de filas está contenido en $\R^4$:

$$A = \begin{pmatrix}
7 & 7 & 1 & 4 \\
-9 & 4 & -7 & 7 \\
\end{pmatrix}$$ 

$F(A)= \{\lambda_1(7,7,1,4) + \lambda_2(-9,4,-7,7): \lambda_1,\lambda_2 \in \R \} \subset \R^4$ 

Como A tiene rango 2, sus dos filas son linealmente independientes y forman una base de F(A). Esta se puede ampliar hasta una base de $\R^4$ añadiendo dos vectores mas que sean l.i. con ellos. Por ejemplo, $(1,0,0,0)$ y $(0,1,0,0)$ sirven.

Así se obtiene una base de $\R^4$ o equivalentemente una matriz regular:

$$P = \begin{pmatrix}
7 & 7 & 1 & 4 \\
-9 & 4 & -7 & 7 \\
1 & 0 & 0 & 0 \\
0 & 1 & 0 & 0 \\
\end{pmatrix}$$

que tiene la propiedad de simplificar A en una $H_1$ que aquí es su FNHC.

$$A = \begin{pmatrix}
7 & 7 & 1 & 4 \\
-9 & 4 & -7 & 7 \\
\end{pmatrix} = \begin{pmatrix}
1 & 0 & 0 & 0 \\
0 & 1 & 0 & 0 \\
\end{pmatrix} \begin{pmatrix}
7 & 7 & 1 & 4 \\
-9 & 4 & -7 & 7 \\
1 & 0 & 0 & 0 \\
0 & 1 & 0 & 0 \\
\end{pmatrix} = H_1P$$

obteniendose una factorización sencilla $A=H_1P$. Las dos últimas filas de P pueden ser cualesquiera vectores pero se eligen que sean l.i. con los originales para que P tenga inversa y $H_1=AP^{-1}$. Hay infinitas P pero una única $H_1$ porque A es de rango pleno por filas.

Ampliemos ahora una base de columnas.

La misma matriz A anterior tiene 4 columnas:

$$A = \begin{pmatrix}
7 & 7 & 1 & 4 \\
-9 & 4 & -7 & 7 \\
\end{pmatrix}$$ 

tal que $v_1 = \begin{pmatrix} 7 \\ -9 \end{pmatrix}$, $v_2 = \begin{pmatrix} 7 \\ 4 \end{pmatrix}$, $v_3 = \begin{pmatrix} 1 \\ -7 \end{pmatrix}$, $v_4 = \begin{pmatrix} 4 \\ 7 \end{pmatrix}$ \\

pero si elegimos $v_1$ y $v_2$ sale una base de $\R^2$ que generan también su espacio de columnas.

$C(A)= \{\mu_1v_1 + \mu_2v_2 \in \R^2: \mu_1,\mu_2 \in \R \}$

Así se obtiene una base de $\R^2$ o equivalentemente una matriz regular:

$$Q = \begin{pmatrix}
 7 & 7 \\
 -9 & 4 \\
\end{pmatrix}$$ 

que tiene la propiedad de simplificar A en una $H_2$ que aquí es su FNHF.

$$A = \begin{pmatrix}
7 & 7 & 1 & 4 \\
-9 & 4 & -7 & 7 \\
\end{pmatrix} = \begin{pmatrix}
7 & 7 \\
-9 & 4 \\
\end{pmatrix} \begin{pmatrix}
1 & 0 & \frac{53}{91} & \frac{-33}{91} \\
0 & 1 & \frac{-40}{91} & \frac{85}{91} \\
\end{pmatrix} = QH_2$$

obteniéndose una factprización sencilla $A=QH_2 \Leftrightarrow H_2=Q^{-1}A$.

Pero dependiendo de la elección de la base Q pueden obtenerse distintas $H_2$. Si se eligen la primera y la cuarta como base de columnas se obtiene una factorización distinta donde:

$$A = \begin{pmatrix}
7 & 7 & 1 & 4 \\
-9 & 4 & -7 & 7 \\
\end{pmatrix} = \begin{pmatrix}
7 & 4 \\
-9 & 7 \\
\end{pmatrix} \begin{pmatrix}
1 & \frac{33}{85} & \frac{7}{17} & 0 \\
0 & \frac{91}{85} & \frac{-8}{17} & 1 \\
\end{pmatrix} = QH_3$$

La matriz $H_3$ no es una FNHC pero su primera y cuarta columnas son elementales.

Finalmente, ampliemos el espacio nulo.

La misma matriz A anterior define el siguiente s.l. homogéneo:

$$ \begin{pmatrix}
7 & 7 & 1 & 4 \\
-9 & 4 & -7 & 7 \\
\end{pmatrix}\begin{pmatrix}
x_1 \\
x_2 \\
x_3 \\
x_4 \\
\end{pmatrix}=
\begin{pmatrix}
0 \\
0 \\
\end{pmatrix} \Leftrightarrow $$

$$\begin{array}{c} 7x_1+7x_2+x_3+4x_4=0 \\ -9x_1+4x_2-7x_3+7x_4=0 \end{array} $$ 

Resolviéndolo, se obtiene una solución general:

$$(x_1,x_2,x_3,x_4) = (\frac{-53\lambda+33\mu}{91},\frac{40\lambda-85\mu}{91},\lambda,\mu) = \lambda(\frac{-53}{91},\frac{40}{91},1,0)+\mu(\frac{33}{91},\frac{-85}{91},0,1)$$
 donde $(\frac{-53}{91},\frac{40}{91},1,0),(\frac{33}{91},\frac{-85}{91},0,1) \in \R^4$ son 2 vectores l.i. que generan a todas las soluciones y forman una base del espacio nulo N(A).

Esta base del espacio nulo se puede ampliar hasta una base de $\R^4$ añadiendo dos vectores que sean l.i. con ellos. Por ejemplo, nos sirven $e_1=(1,0,0,0),(0,1,0,0)$ ya que la matriz por columnas

$$P_1 = \begin{pmatrix}
1 & 0 & \frac{-53}{91} & \frac{33}{91} \\
0 & 1 & \frac{40}{91} & \frac{-85}{91} \\
0 & 0 & 1 & 0 \\
0 & 0 & 0 & 1 \\
\end{pmatrix}$$

tiene determinante distinto de 0 (vale 1). Así, multiplicando a derecha por $P_1$ se obtiene una matriz $A_1$ con 2 columnas de ceros:

$$A_1 = \begin{pmatrix}
7 & 7 & 0 & 0 \\
-9 & 4 & 0 & 0 \\
\end{pmatrix} = \begin{pmatrix}
7 & 7 & 1 & 4 \\
-9 & 4 & -7 & 7 \\
\end{pmatrix} \begin{pmatrix}
1 & 0 & \frac{-53}{91} & \frac{33}{91} \\
0 & 1 & \frac{40}{91} & \frac{-85}{91} \\
0 & 0 & 1 & 0 \\
0 & 0 & 0 & 1 \\
\end{pmatrix} = AP_1$$

Como esta matriz $A_1$ se puede factorizar usando su base de columnas:

$$A_1 = \begin{pmatrix}
7 & 7 & 0 & 0 \\
-9 & 4 & 0 & 0 \\
\end{pmatrix} = \begin{pmatrix}
7 & 7 \\
-9 & 4 \\
\end{pmatrix} \begin{pmatrix}
1 & 0 & 0 & 0 \\
0 & 1 & 0 & 0 \\
\end{pmatrix} = QH$$

De las dos últimas igualdades matriciales se puede despejar A:

$$AP_1 = A_1 = QH \Leftrightarrow A = QHP_1^{-1}$$

O sea, hemos obtenido la factorización:

$$ \begin{pmatrix}
7 & 7 & 1 & 4 \\
-9 & 4 & -7 & 7 \\
\end{pmatrix} = \begin{pmatrix}
7 & 7 \\
-9 & 4 \\
\end{pmatrix} \begin{pmatrix}
1 & 0 & 0 & 0 \\
0 & 1 & 0 & 0 \\
\end{pmatrix} \begin{pmatrix}
1 & 0 & \frac{-53}{91} & \frac{33}{91} \\
0 & 1 & \frac{40}{91} & \frac{-85}{91} \\
0 & 0 & 1 & 0 \\
0 & 0 & 0 & 1 \\
\end{pmatrix}^{-1}$$

donde la matriz central H es la forma de Hermite por filas y columnas de A, la matriz izquierda Q es una base de columnas y la matriz derecha $P_1^{-1}$ es la inversa de una base ampliada de N(A) escrita por columnas.











% ---------------------------------------------------------------------------
% FIN DEL DOCUMENTO
% ---------------------------------------------------------------------------

\end{document}