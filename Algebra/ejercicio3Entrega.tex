%%%%%%%%%%%%%%%%%%%%%%%%%%%%%%%%%%%%%%%%%%%%%%%%%%%%%%%%%%%%%%%%%%%%%%%%% 
% 
% Apuntes de la asignatura Análisis Matemático II.
% Grado en Estadística.
% Universidad de Granada.
% Curso 2017/18.
% 
% 
% Colaboradores:
% Miguel Anguita Ruiz
% 
% Agradecimientos:
% Andrés Herrera (@andreshp) y Mario Román (@M42) por
% las plantillas base.
% 
% Sitio original:
% https://github.com/libreim/apuntesDGIIM/
% 
% Licencia:
% CC BY-NC-SA 4.0 (https://creativecommons.org/licenses/by-nc-sa/4.0/)
% 
%%%%%%%%%%%%%%%%%%%%%%%%%%%%%%%%%%%%%%%%%%%%%%%%%%%%%%%%%%%%%%%%%%%%%%%%% 


% ------------------------------------------------------------------------------
% ACKNOWLEDGMENTS
% ------------------------------------------------------------------------------

%%%%%%%%%%%%%%%%%%%%%%%%%%%%%%%%%%%%%%%%%%%%%%%%%%%%%%%%%%%%%%%%%%%%%%%% 
% Plantilla básica de Latex en Español.
% 
% Autor: Andrés Herrera Poyatos (https://github.com/andreshp) 
% 
% Es una plantilla básica para redactar documentos. Utiliza el paquete fancyhdr
% para darle un estilo moderno pero serio.
% 
% La plantilla se encuentra adaptada al español.
% 
%%%%%%%%%%%%%%%%%%%%%%%%%%%%%%%%%%%%%%%%%%%%%%%%%%%%%%%%%%%%%%%%%%%%%%%%% 

%%%%%%%%%%%%%%%%%%%%%%%%%%%%%%%%%%%%%%%%%%%%%%%%%%%%%%%%%%%%%%%%%%%%%%%%% 
% Plantilla de Trabajo
% Modificación de una plantilla de Latex de Frits Wenneker para adaptarla 
% al castellano y a las necesidades de escribir informática y matemáticas.
% 
% Editada por: Mario Román
% 
% License:
% CC BY-NC-SA 3.0 (http://creativecommons.org/licenses/by-nc-sa/3.0/)
%%%%%%%%%%%%%%%%%%%%%%%%%%%%%%%%%%%%%%%%%%%%%%%%%%%%%%%%%%%%%%%%%%%%%%%%% 

%%%%%%%%%%%%%%%%%%%%%%%%%%%%%%%%%%%%%%%%%%%%%%%%%%%%%%%%%%%%%%%%%%%%%%%%% 
% Short Sectioned Assignment
% LaTeX Template
% Version 1.0 (5/5/12)
% 
% This template has been downloaded from:
% http://www.LaTeXTemplates.com
% 
% Original author:
% Frits Wenneker (http://www.howtotex.com)
% 
% License:
% CC BY-NC-SA 3.0 (http://creativecommons.org/licenses/by-nc-sa/3.0/)
% 
%%%%%%%%%%%%%%%%%%%%%%%%%%%%%%%%%%%%%%%%%%%%%%%%%%%%%%%%%%%%%%%%%%%%%%%%% 


% Tipo de documento y opciones.
%\documentclass[11pt, a4paper, twoside]{article} % Usar para imprimir
\documentclass[11pt, a4paper]{article}

% ---------------------------------------------------------------------------
% PAQUETES
% ---------------------------------------------------------------------------

% Idioma y codificación para Español.
\usepackage[utf8]{inputenc}
\usepackage[spanish, es-tabla, es-lcroman, es-noquoting]{babel}
\selectlanguage{spanish} 
% \usepackage[T1]{fontenc}

% Fuente utilizada.
\usepackage{courier}    % Fuente Courier.
\usepackage{microtype}  % Mejora la letra final de cara al lector.

% Diseño de página.
\usepackage{fancyhdr}   % Utilizado para hacer cabeceras y pies de página.

\usepackage{titlesec} 	% Utilizado para hacer títulos propios.
\usepackage{lastpage}   % Referencia a la última página.
\usepackage{extramarks} % Marcas extras. Utilizado en pie de página y cabecera.
\usepackage[parfill]{parskip}    % Crea una nueva línea entre párrafos.
\usepackage{geometry}            % Geometría de las páginas.

% Símbolos y matemáticas.
\usepackage{amssymb, amsmath, amsthm, amsfonts, amscd}
\usepackage{upgreek}
\usepackage{mathrsfs}

\usepackage{mdframed}

% Gráficos

\usepackage{pgf,tikz}
\usepackage{tkz-euclide}
\usetkzobj{all}

% Otros.
\usepackage{enumitem}   % Listas mejoradas.
\usepackage{hyperref}
\usepackage{graphicx}   % Gráficos.
\usepackage[space]{grffile}  % Permitir espacios en rutas de gráficos
\usepackage{xcolor}     % Colores.8

% Fuentes personalizadas

\usepackage[scaled=.85]{newpxtext,newpxmath}
\usepackage[scaled=.85]{FiraSans}
\usepackage[T1]{fontenc}

% Código para ajustar las fuentes matemáticas al estilo del texto
% que le rodea

\DeclareMathVersion{sans}
\SetSymbolFont{operators}{sans}{OT1}{cmbr}{m}{n}
\SetSymbolFont{letters}{sans}{OML}{cmbr}{m}{it}
\SetSymbolFont{symbols}{sans}{OMS}{cmbrs}{m}{n}
\SetMathAlphabet{\mathit}{sans}{OT1}{cmbr}{m}{sl}
\SetMathAlphabet{\mathbf}{sans}{OT1}{cmbr}{bx}{n}
\SetMathAlphabet{\mathtt}{sans}{OT1}{cmtl}{m}{n}
\SetSymbolFont{largesymbols}{sans}{OMX}{iwona}{m}{n}

\DeclareMathVersion{boldsans}
\SetSymbolFont{operators}{boldsans}{OT1}{cmbr}{b}{n}
\SetSymbolFont{letters}{boldsans}{OML}{cmbrm}{b}{it}
\SetSymbolFont{symbols}{boldsans}{OMS}{cmbrs}{b}{n}
\SetMathAlphabet{\mathit}{boldsans}{OT1}{cmbr}{b}{sl}
\SetMathAlphabet{\mathbf}{boldsans}{OT1}{cmbr}{bx}{n}
\SetMathAlphabet{\mathtt}{boldsans}{OT1}{cmtl}{b}{n}
\SetSymbolFont{largesymbols}{boldsans}{OMX}{iwona}{bx}{n}

\newif\IfInSansMode
\let\oldsf\sffamily
\renewcommand*{\sffamily}{\oldsf\mathversion{sans}\InSansModetrue}
\let\oldmd\mdseries
\renewcommand*{\mdseries}{\oldmd\IfInSansMode\mathversion{sans}\fi\relax}
\let\oldbf\bfseries
\renewcommand*{\bfseries}{\oldbf\IfInSansMode\mathversion{boldsans}\else%
   \mathversion{bold}\fi\relax}
\let\oldnorm\normalfont
\renewcommand*{\normalfont}{\oldnorm\InSansModefalse\mathversion{normal}}
\let\oldrm\rmfamily
\renewcommand*{\rmfamily}{\oldrm\InSansModefalse\mathversion{normal}}

% Colores

\definecolor{50}{HTML}{E8F5E9}
\definecolor{300}{HTML}{81C784}
\definecolor{500}{HTML}{4CAF50}
\definecolor{700}{HTML}{388E3C}

% ---------------------------------------------------------------------------
% OPCIONES PERSONALIZADAS
% ---------------------------------------------------------------------------

% Formato de texto.
\linespread{1.3}            % Espaciado entre líneas.
\setlength\parindent{0pt}   % No indentar el texto por defecto.
\setlist{leftmargin=.5in}   % Indentación para las listas.

% Estilo de página.
\pagestyle{fancy}
\fancyhf{}
\geometry{left=3cm,right=3cm,top=3cm,bottom=3cm}   % Márgenes y cabecera.

% Estilo de las cabeceras

\titleformat{\section}
  {\Large\bfseries\sffamily}{\thesection}{1em}{}
\titleformat{\subsection}
  {\large\sffamily}{\thesubsection}{1em}{}
\titleformat{\subsubsection}
  {\sffamily}{\thesubsubsection}{1em}{}

% Estilo de enlaces
\hypersetup{
  % hidelinks = true,   % Oculta todos los enlaces.
  colorlinks = true,   % Muestra todos los enlaces, sin bordes alrededor.
  linkcolor=black,     % Color de enlaces genéricos
  citecolor={blue!50!black},   % Color de enlaces de referencias
  urlcolor={blue!80!black}     % Color de enlaces de URL
}

% Ruta donde buscar gráficos
\graphicspath{{../Recursos/Plantillas/} {Recursos/Plantillas/} {./img/} {Análisis Matemático II/img/}}

% Redefinir entorno de demostración (reducir espacio superior)
% \makeatletter
% \renewenvironment{proof}[1][\proofname] {\vspace{-15pt}\par\pushQED{\qed}\normalfont\topsep6\p@\@plus6\p@\relax\trivlist\item[\hskip\labelsep\it#1\@addpunct{.}]\ignorespaces}{\popQED\endtrivlist\@endpefalse}
% \makeatother

% Aumentar el tamaño del interlineado
\linespread{1.3}

% Permitir salto de página en ecuaciones
\allowdisplaybreaks


% ---------------------------------------------------------------------------
% COMANDOS PERSONALIZADOS
% ---------------------------------------------------------------------------

% Redefinir letra griega épsilon.
\let\epsilon\upvarepsilon

% Valor absoluto: \abs{}
\providecommand{\abs}[1]{\lvert#1\rvert}    

% Fracción grande: \ddfrac{}{}
\newcommand\ddfrac[2]{\frac{\displaystyle #1}{\displaystyle #2}}

% Texto en negrita en modo matemática: \bm{}
\newcommand{\bm}[1]{\boldsymbol{#1}}

% Línea horizontal.
\newcommand{\horrule}[1]{\rule{\linewidth}{#1}}

% Letras de conjuntos
\newcommand{\R}{\mathbb{R}} \newcommand{\N}{\mathbb{N}}

% Sucesiones
\newcommand{\xn}{\{x_n\}}
\newcommand{\fn}{\{f_n\}}

% Letra griega "chi" en línea con el texto
\DeclareRobustCommand{\rchi}{{\Large \mathpalette\irchi\relax}}
\newcommand{\irchi}[2]{\raisebox{0.4\depth}{$#1\chi$}} % inner command, used by \rchi 

% Letra 'omega'
\newcommand{\W}{\Omega}
\newcommand{\w}{\omega}


% ---------------------------------------------------------------------------
% CABECERA Y PIE DE PÁGINA
% ---------------------------------------------------------------------------

\renewcommand{\sectionmark}[1]{%
\markboth{#1}{}}

\renewcommand{\subsectionmark}[1]{%
\markright{#1}{}}

%\addtolength{\headheight}{4ex}
\renewcommand{\headrulewidth}{0pt}

\addtolength{\headwidth}{\marginparsep}
%\addtolength{\headwidth}{\marginparwidth}

\fancypagestyle{section}{%
  \fancyhead{}
  %\addtolength{\headheight}{-10ex}
  %\renewcommand{\headheight}{0pt}%
  %\setlength{\footskip}{-48pt}%
  \fancyfoot[LE,RO]{\Large\sffamily\thepage}
  \renewcommand{\headrulewidth}{0pt}%
  \renewcommand{\footrulewidth}{0pt}%
}

\let\originalsection\section
\RenewDocumentCommand{\section}{som}{%
  \IfBooleanTF{#1}
    {\originalsection*{#3}}
    {\IfNoValueTF{#2}
      {\originalsection{#3}}
      {\originalsection[#2]{#3}}%
    }%
  \thispagestyle{section}%
}

\fancyhead[LE,RO]{\rule[-4ex]{0pt}{2ex}\sffamily\textsl{\rightmark}}
\fancyhead[LO,RE]{\sffamily{\leftmark}}
\fancyfoot[LE,RO]{\Large\sffamily\thepage}



% ---------------------------------------------------------------------------
% ENTORNOS PARA MATEMÁTICAS
% ---------------------------------------------------------------------------

% Nuevo estilo para definiciones.
\newtheoremstyle{definition-style} % Nombre del estilo.
{}               % Espacio por encima.
{}               % Espacio por debajo.
{}                   % Fuente del cuerpo.
{}                   % Identación.
{\bf\sffamily}                % Fuente para la cabecera.
{.}                  % Puntuación tras la cabecera.
{.5em}               % Espacio tras la cabecera.
{\thmname{#1}\thmnumber{ #2}\thmnote{ (#3)}}     % Especificación de la cabecera (actual: nombre en negrita).

% Nuevo estilo para notas.
\newtheoremstyle{remark-style} 
{10pt}                
{10pt}                
{}                   
{}                   
{\itshape \sffamily}          
{.}                  
{.5em}               
{}                  

% Nuevo estilo para teoremas y proposiciones.
\newtheoremstyle{theorem-style}
{}                
{}                
{}           
{}                  
{\bfseries \sffamily}             
{.}                
{.5em}           
{\thmname{#1}\thmnumber{ #2}\thmnote{ (#3)}}

% Nuevo estilo para ejemplos.
\newtheoremstyle{example-style}
{10pt}                
{10pt}                
{}                  
{}                   
{\bf \sffamily}              
{}                 
{.5em}               
{\thmname{#1}\thmnumber{ #2.}\thmnote{ #3.}}  

% Nuevo estilo para la demostración
%\renewenvironment{proof}{{\itshape \sffamily Demostración \\}}{\hspace*{\fill}\qed}

\makeatletter
\renewenvironment{proof}[1][\proofname] {\par\pushQED{\qed}\normalfont\topsep6\p@\@plus6\p@\relax\trivlist\item[\hskip\labelsep\itshape\sffamily#1\@addpunct{.}]\ignorespaces}{\popQED\endtrivlist\@endpefalse}
\makeatother

% Configuración general de mdframe, los estilos de los teoremas, etc
\mdfsetup{skipabove=12pt,skipbelow=12pt,innertopmargin=12pt,innerbottommargin=4pt}

% Creamos los 'marcos' de los estilos

\mdfdefinestyle{nth-frame}{
	linewidth=2pt, %
	linecolor= 500, % 
	%linecolor=black,
	topline=false, %
	bottomline=false, %
	rightline=false,%
	leftmargin=0pt, %
	innerleftmargin=1em, % 
	rightmargin=0pt, %
	innerrightmargin=0pt, % % 
	%splittopskip=\topskip, %
	%splitbottomskip=\topskip, %
}% 

\mdfdefinestyle{nprop-frame}{
	linewidth=2pt, %
	linecolor= 300, %
	%linecolor= gray, 
	topline=false, %
	bottomline=false, %
	rightline=false,%
	leftmargin=0pt, %
	innerleftmargin=1em, %
	innerrightmargin=0em, 
	rightmargin=0pt, %
	%splittopskip=\topskip, %
	%splitbottomskip=\topskip, %
}%       

\mdfdefinestyle{ndef-frame}{
	linewidth=2pt, %
	linecolor= 300, % 
	%linecolor= gray!50,
	backgroundcolor= 50,
	%backgroundcolor= gray!5,
	topline=false, %
	bottomline=false, %
	rightline=false,%
	leftmargin=0pt, %
	innerleftmargin=1em, %
	innerrightmargin=1em, 
	rightmargin=0pt, % 
	innertopmargin=1.5em,%
	innerbottommargin=1em, % 
	splittopskip=\topskip, %
	%splitbottomskip=\topskip, %
}% 

\mdfdefinestyle{ejemplo-frame}{
	linewidth=0pt, %
	linecolor= 300, % 
	%backgroundcolor= 50,
	leftline=false, %
	rightline=false, %
	leftmargin=0pt, %
	innerleftmargin=1.5em, %
	innerrightmargin=1.5em, 
	rightmargin=0pt, % 
	innertopmargin=0em,%
	innerbottommargin=0em, % 
	splittopskip=\topskip, %
	%splitbottomskip=\topskip, %
}%                

% Asignamos los marcos a los estilos

\surroundwithmdframed[style=nth-frame]{nth}
\surroundwithmdframed[style=nprop-frame]{nprop}
\surroundwithmdframed[style=nprop-frame]{ncor}
\surroundwithmdframed[style=ndef-frame]{ndef}
\surroundwithmdframed[style=ejemplo-frame]{ejemplo}
                 
% Asignamos los estilos definidos anteriormente a los entornos correspondientes
% Teoremas, proposiciones y corolarios.
\theoremstyle{theorem-style}
\newtheorem{nth}{Teorema}[section]
\newtheorem{nprop}{Proposición}[section]
\newtheorem{ncor}{Corolario}[section]
\newtheorem{lema}{Lema}[section]
% Definiciones.
\theoremstyle{definition-style}
\newtheorem{ndef}{Definición}[section]

% Notas.
\theoremstyle{remark-style}
\newtheorem*{nota}{Nota}

% Ejemplos.
\theoremstyle{example-style}
\newtheorem{ejemplo}{Ejemplo}[section]



% Listas ordenadas con números romanos (i), (ii), etc.
\newenvironment{nlist}
{\begin{enumerate}
    \renewcommand\labelenumi{(\emph{\roman{enumi})}}}
  {\end{enumerate}}

% División por casos con llave a la derecha.
\newenvironment{rcases}
{\left.\begin{aligned}}
    {\end{aligned}\right\rbrace}


% ---------------------------------------------------------------------------
% PÁGINA DE TÍTULO
% ---------------------------------------------------------------------------

% Título del documento.
\newcommand{\subject}{Álgebra: ejercicio 3}

% Autor del documento.
\newcommand{\docauthor}{Miguel Anguita Ruiz}

% Título
\title{
  \normalfont \normalsize 
  \textsc{Universidad de Granada} \\ [25pt]    % Texto por encima.
  \horrule{0.5pt} \\[0.4cm] % Línea horizontal fina.
  \huge \sffamily\subject\\ % Título.
  \horrule{2pt} \\[0.5cm] % Línea horizontal gruesa.
}

% Autor.
\author{\Large\sffamily{\docauthor}}

% Fecha.
\date{\vspace{-1.5em} \normalsize \sffamily Curso 2017/18}



% ---------------------------------------------------------------------------
% COMIENZO DEL DOCUMENTO
% ---------------------------------------------------------------------------

\begin{document}

\maketitle  % Título.
\vfill
\begin{center}
  %\includegraphics{by-nc-sa.pdf}  % Licencia.
\end{center}
\newpage
\tableofcontents    % Índice
\newpage


% --------------------------------------------------------------------------------
% Introducción.
% --------------------------------------------------------------------------------

\section{Enunciado}
El DNI usado para el ejercicio es el siguiente: 77149477. \\

Dada la matriz $A = 
	\begin{pmatrix}
	7 & 7 & 9 & 4 \\
	1 & 4 & 7 & 7 \\
	6 & 8 & 8 & 5 \\
	\end{pmatrix}$
	
a) Halla sus FNHF y FNHC. ¿Sale el mismo rango por filas que por columnas? \\
b) Halla las 2 grammianas de $A$. Sin calcular determinantes, ¿pueden ser ambas matrices regulares? \\
c) Razona que $A$ no puede tener inversas laterales por los dos lados. ¿Tiene inversa lateral tu matriz $A$? En caso afirmativo, ¿por qué lado tiene inversa?

\section{Solución}

A continuación realizaré transformaciones elementales para obtener las formas de Hermite por filas y por columnas. Empezaré el proceso por filas:

Multiplico A por $E_{21}(-\frac{1}{7})$ y por $E_{31}(-\frac{6}{7})$ para hacer dos ceros por debajo del primer pivote.

$\begin{pmatrix}
1 & 0 & 0 \\
-\frac{1}{7} & 1 & 0 \\
0 & 0 & 1 \\
\end{pmatrix}
\begin{pmatrix}
7 & 7 & 9 & 4 \\
1 & 4 & 7 & 7 \\
6 & 8 & 8 & 5 \\
\end{pmatrix} = 
\begin{pmatrix}
7 & 7 & 9 & 4 \\
0 & 3 & \frac{40}{7} & \frac{45}{7} \\
6 & 8 & 8 & 5 \\
\end{pmatrix} \Longrightarrow
\begin{pmatrix}
1 & 0 & 0 \\
0 & 1 & 0 \\
-\frac{6}{7} & 0 & 1 \\
\end{pmatrix}
\begin{pmatrix}
7 & 7 & 9 & 4 \\
0 & 3 & \frac{40}{7} & \frac{45}{7} \\
6 & 8 & 8 & 5 \\
\end{pmatrix} = 
\begin{pmatrix}
7 & 7 & 9 & 4 \\
0 & 3 & \frac{40}{7} & \frac{45}{7} \\
0 & 2 & \frac{2}{7} & \frac{11}{7} \\
\end{pmatrix}$ \\

Multiplico ahora por $E_{32}(-\frac{2}{3})$ y por $E_{23}(\frac{60}{37})$ para hacer un cero por debajo del segundo pivote y un cero por encima del tercer pivote. \\

$\begin{pmatrix}
1 & 0 & 0 \\
0 & 1 & 0 \\
0 & -\frac{2}{3} & 1 \\
\end{pmatrix} 
\begin{pmatrix}
7 & 7 & 9 & 4 \\
0 & 3 & \frac{40}{7} & \frac{45}{7} \\
0 & 2 & \frac{2}{7} & \frac{11}{7} \\
\end{pmatrix} =
\begin{pmatrix}
7 & 7 & 9 & 4 \\
0 & 3 & \frac{40}{7} & \frac{45}{7} \\
0 & 0 & -\frac{74}{21} & -\frac{57}{21} \\
\end{pmatrix} \Longrightarrow
\begin{pmatrix}
1 & 0 & 0 \\
0 & 1 & \frac{60}{37} \\
0 & 0 & 1 \\
\end{pmatrix} 
\begin{pmatrix}
7 & 7 & 9 & 4 \\
0 & 3 & \frac{40}{7} & \frac{45}{7} \\
0 & 0 & -\frac{74}{21} & -\frac{57}{21} \\
\end{pmatrix} = 
\begin{pmatrix}
7 & 7 & 9 & 4 \\
0 & 3 & 0 & \frac{75}{37} \\
0 & 0 & -\frac{74}{21} & -\frac{57}{21} \\
\end{pmatrix}$ \\

Multiplico ahora por $E_{13}(\frac{189}{74})$ y por $E_{12}(-\frac{7}{3})$ para hacer otro cero por encima del tercer pivote y un último cero por encima del segundo pivote. \\

$\begin{pmatrix}
1 & 0 & \frac{189}{74} \\
0 & 1 & 0 \\
0 & 0 & 1 \\
\end{pmatrix}  
\begin{pmatrix}
7 & 7 & 9 & 4 \\
0 & 3 & 0 & \frac{75}{37} \\
0 & 0 & -\frac{74}{21} & -\frac{57}{21} \\
\end{pmatrix} = 
\begin{pmatrix}
7 & 7 & 0 & -\frac{217}{74} \\
0 & 3 & 0 & \frac{75}{37} \\
0 & 0 & -\frac{74}{21} & -\frac{57}{21} \\
\end{pmatrix} \Longrightarrow
\begin{pmatrix}
1 & -\frac{7}{3} & 0 \\
0 & 1 & 0 \\
0 & 0 & 1 \\
\end{pmatrix}   
\begin{pmatrix}
7 & 7 & 0 & -\frac{217}{74} \\
0 & 3 & 0 & \frac{75}{37} \\
0 & 0 & -\frac{74}{21} & -\frac{57}{21} \\
\end{pmatrix} = 
\begin{pmatrix}
7 & 0 & 0 & -\frac{567}{74} \\
0 & 3 & 0 & \frac{75}{37} \\
0 & 0 & -\frac{74}{21} & -\frac{57}{21} \\
\end{pmatrix}$ \\

Finalmente, multiplico por $E_1(\frac{1}{7})$, por $E_2(\frac{1}{3})$ y por $E_3(-\frac{21}{74})$ para obtener una diagonal principal con unos. \\

$\begin{pmatrix}
\frac{1}{7} & 0 & 0 \\
0 & \frac{1}{3} & 0 \\
0 & 0 & -\frac{21}{74} \\
\end{pmatrix}    
\begin{pmatrix}
7 & 0 & 0 & -\frac{567}{74} \\
0 & 3 & 0 & \frac{75}{37} \\
0 & 0 & -\frac{74}{21} & -\frac{57}{21} \\
\end{pmatrix} = 
\begin{pmatrix}
1 & 0 & 0 & -\frac{81}{74} \\
0 & 1 & 0 & \frac{25}{37} \\
0 & 0 & 1 & \frac{57}{74} \\
\end{pmatrix}$ \\

Esta última matriz es la FNHF de $A$ (forma normal de Hermite por filas) y puede observarse que el rango de $A$ es 3, que es el número de filas no nulas de su FNHF. 

A continuación, procedo a hallar la forma normal de Hermite por columnas de $A$.

Multiplico A por $E_{21}(-1)$ y por $E_{31}(-\frac{9}{7})$ para hacer dos ceros a la derecha del primer pivote.

$
\begin{pmatrix}
7 & 7 & 9 & 4 \\
1 & 4 & 7 & 7 \\
6 & 8 & 8 & 5 \\
\end{pmatrix}
\begin{pmatrix}
1 & -1 & 0 & 0 \\
0 & 1 & 0 & 0 \\
0 & 0 & 1 & 0 \\
0 & 0 & 0 & 1 \\
\end{pmatrix}
 = 
\begin{pmatrix}
7 & 0 & 9 & 4 \\
1 & 3 & 7 & 7 \\
6 & 2 & 8 & 5 \\
\end{pmatrix} \Longrightarrow
\begin{pmatrix}
7 & 0 & 9 & 4 \\
1 & 3 & 7 & 7 \\
6 & 2 & 8 & 5 \\
\end{pmatrix}
\begin{pmatrix}
1 & 0 & -\frac{9}{7} & 0 \\
0 & 1 & 0 & 0 \\
0 & 0 & 1 & 0 \\
0 & 0 & 0 & 1 \\
\end{pmatrix}
 = 
\begin{pmatrix}
7 & 0 & 0 & 4 \\
1 & 3 & \frac{40}{7} & 7 \\
6 & 2 & \frac{2}{7} & 5 \\
\end{pmatrix}$ \\

Multiplico ahora por $E_{32}(-\frac{40}{21})$ y por $E_{23}(\frac{42}{74})$ para hacer un cero a la derecha del segundo pivote y un cero a la izquierda del tercer pivote. \\

$\begin{pmatrix}
7 & 0 & 0 & 4 \\
1 & 3 & \frac{40}{7} & 7 \\
6 & 2 & \frac{2}{7} & 5 \\
\end{pmatrix}
\begin{pmatrix}
1 & 0 & 0 & 0 \\
0 & 1 & -\frac{40}{21} & 0 \\
0 & 0 & 1 & 0 \\
0 & 0 & 0 & 1 \\
\end{pmatrix} 
 =
\begin{pmatrix}
7 & 0 & 0 & 4 \\
1 & 3 & 0 & 7 \\
6 & 2 & -\frac{74}{21} & 5 \\
\end{pmatrix} \Longrightarrow
\begin{pmatrix}
7 & 0 & 0 & 4 \\
1 & 3 & 0 & 7 \\
6 & 2 & -\frac{74}{21} & 5 \\
\end{pmatrix}
\begin{pmatrix}
1 & 0 & 0 & 0 \\
0 & 1 & 0 & 0 \\
0 & \frac{42}{74} & 1 & 0 \\
0 & 0 & 0 & 1 \\
\end{pmatrix} 
 = 
\begin{pmatrix}
7 & 0 & 0 & 4 \\
1 & 3 & 0 & 7 \\
6 & 0 & -\frac{74}{21} & 5 \\
\end{pmatrix}$ \\

Multiplico ahora por $E_{13}(\frac{63}{37})$ y por $E_{12}(-\frac{1}{3})$ para hacer otro cero a la izquierda del tercer pivote y un último cero a la izquierda del segundo pivote. \\

$\begin{pmatrix}
7 & 0 & 0 & 4 \\
1 & 3 & 0 & 7 \\
6 & 0 & -\frac{74}{21} & 5 \\
\end{pmatrix}
\begin{pmatrix}
1 & 0 & 0 & 0 \\
0 & 1 & 0 & 0 \\
\frac{63}{37} & 0 & 1 & 0 \\
0 & 0 & 0 & 1 \\
\end{pmatrix}  
 = 
\begin{pmatrix}
7 & 0 & 0 & 4 \\
1 & 3 & 0 & 7 \\
0 & 0 & -\frac{74}{21} & 5 \\
\end{pmatrix} \Longrightarrow
\begin{pmatrix}
7 & 0 & 0 & 4 \\
1 & 3 & 0 & 7 \\
0 & 0 & -\frac{74}{21} & 5 \\
\end{pmatrix}
\begin{pmatrix}
1 & 0 & 0 & 0 \\
-\frac{1}{3} & 1 & 0 & 0 \\
0 & 0 & 1 & 0 \\
0 & 0 & 0 & 1 \\
\end{pmatrix}   
 = 
\begin{pmatrix}
7 & 0 & 0 & 4 \\
0 & 3 & 0 & 7 \\
0 & 0 & -\frac{74}{21} & 5 \\
\end{pmatrix}$ \\

Multiplico ahora por $E_{41}(-\frac{4}{7})$, por $E_{42}(-\frac{7}{3})$ y por $E_{43}(\frac{105}{74})$ para hacer ceros en la última columna. \\

$\begin{pmatrix}
7 & 0 & 0 & 4 \\
0 & 3 & 0 & 7 \\
0 & 0 & -\frac{74}{21} & 5 \\
\end{pmatrix}
\begin{pmatrix}
1 & 0 & 0 & -\frac{4}{7} \\
0 & 1 & 0 & -\frac{7}{3} \\
0 & 0 & 1 & \frac{105}{74} \\
0 & 0 & 0 & 1 \\
\end{pmatrix}   
 = 
\begin{pmatrix}
7 & 0 & 0 & 0 \\
0 & 3 & 0 & 0 \\
0 & 0 & -\frac{74}{21} & 0 \\
\end{pmatrix}$ \\

Finalmente, multiplico por $E_1(\frac{1}{7})$, por $E_2(\frac{1}{3})$ y por $E_3(-\frac{21}{74})$ para obtener una diagonal principal con unos. \\

$\begin{pmatrix}
7 & 0 & 0 & 0 \\
0 & 3 & 0 & 0 \\
0 & 0 & -\frac{74}{21} & 0 \\
\end{pmatrix}
\begin{pmatrix}
\frac{1}{7} & 0 & 0 & 0 \\
0 & \frac{1}{3} & 0 & 0 \\
0 & 0 & -\frac{21}{74} & 0 \\
0 & 0 & 0 & 1 \\
\end{pmatrix}    
 = 
\begin{pmatrix}
1 & 0 & 0 & 0 \\
0 & 1 & 0 & 0 \\
0 & 0 & 1 & 0 \\
\end{pmatrix}$ \\

Esta última matriz es la FNHC de $A$ (forma normal de Hermite por columnas) y puede observarse que el rango de $A$ es 3, que es el número de columnas no nulas de su FNHC.

Como se puede comprobar, sale el mismo rango por filas que por columnas.

A continuación, hallaré las grammianas de $A$.

$G_1 = A^tA = \begin{pmatrix}
7 & 1 & 6 \\
7 & 4 & 8 \\
9 & 7 & 8 \\
4 & 7 & 5 \\
\end{pmatrix}
\begin{pmatrix}
7 & 7 & 9 & 4 \\
1 & 4 & 7 & 7 \\
6 & 8 & 8 & 5 \\
\end{pmatrix} = 
\begin{pmatrix}
86 & 101 & 118 & 65 \\
101 & 129 & 155 & 96 \\
118 & 155 & 194 & 125 \\
65 & 96 & 125 & 90 \\
\end{pmatrix}$ \\

$G_2 = AA^t = \begin{pmatrix}
7 & 7 & 9 & 4 \\
1 & 4 & 7 & 7 \\
6 & 8 & 8 & 5 \\
\end{pmatrix}\begin{pmatrix}
7 & 1 & 6 \\
7 & 4 & 8 \\
9 & 7 & 8 \\
4 & 7 & 5 \\
\end{pmatrix} = 
\begin{pmatrix}
195 & 126 & 190 \\
126 & 115 & 129 \\
190 & 129 & 189 \\
\end{pmatrix}$

Sabemos que las grammianas de una matriz conservan el rango de la propia matriz, pues multiplicar por su traspuesta no lo varía. También sabemos que las grammianas son siempre matrices cuadradas. Podemos observar, además, que la matriz $A$ es rectangular, luego sus grammianas serán cuadradas de distintas dimensiones, luego ambas no pueden regulares a la vez pues tienen igual rango con distintas dimensiones. Como el rango de $A$ es 3 y el rango de $G_1 = 3 < min(4,4)$ entonces $G_1$ no es regular, mientras que $G_2$ sí lo es.

De la matriz $A$ se observa además que tiene el mismo número de filas que el máximo rango posible para una matriz de esas dimensiones (3), luego $A$ es de rango pleno por filas (luego no lo es de columnas). Ello implica que tiene inversa lateral por la derecha. No puede tener inversas laterales por ambos lados, porque, de ser así, $A$ sería regular, pero $A$ no puede ser regular ya que no es cuadrada.

A continuación, hallaré la inversa lateral por la derecha de $A$ resolviendo la siguiente ecuación: 

$$\begin{pmatrix}
7 & 7 & 9 & 4 \\
1 & 4 & 7 & 7 \\
6 & 8 & 8 & 5 \\
\end{pmatrix}
\begin{pmatrix}
x_1 & y_1 & z_1 \\
x_2 & y_2 & z_2 \\
x_3 & y_3 & z_3 \\
x_4 & y_4 & z_4 \\
\end{pmatrix} = 
\begin{pmatrix}
1 & 0 & 0 \\
0 & 1 & 0 \\
0 & 0 & 1 \\
\end{pmatrix}$$

Por lo tanto, tengo que resolver tres sistemas lineales.

$7x_1+7x_2+9x_3+4x_4=1 \\
x_1+4x_2+7x_3+7x_4=0 \\
6x_1+8x_2+8x_3+5x_4=0 $

$7y_1+7y_2+9y_3+4y_4=0 \\
y_1+4y_2+7y_3+7y_4=1 \\
6y_1+8y_2+8y_3+5y_4=0$

$7z_1+7z_2+9z_3+4z_4=0 \\
z_1+4z_2+7z_3+7z_4=0 \\
6z_1+8z_2+8z_3+5z_4=1$

Los tres sistesmas son compatibles indeterminados pues la matriz de los sistemas es la matriz $A$, que tiene rango igual a la ampliada pero menor que el número de incógnitas, luego los sistemas tienen infinitas soluciones y, por ello, existen infinitas inversas laterales por la derecha de $A$.

Habiendo realizado ya los cálculos, los resultados son los siguientes:

$(x_1,x_2,x_3,x_4) = (\frac{12}{37}+\frac{81}{74}\lambda,-\frac{17}{37}-\frac{25}{37}\lambda,\frac{8}{37}-\frac{57}{74}\lambda,\lambda) \\
(y_1,y_2,y_3,y_4) = (-\frac{8}{37}+\frac{81}{74}\mu,-\frac{1}{37}-\frac{25}{37}\mu,\frac{7}{37}-\frac{57}{74}\mu,\mu) \\
(z_1,z_2,z_3,z_4) = (-\frac{13}{74}+\frac{81}{74}\beta,\frac{20}{37}-\frac{25}{37}\beta,-\frac{21}{74}-\frac{57}{74}\beta,\beta)$

Dando valores arbitrarios a $\lambda, \mu, \beta$, podemos hallar una de las infinitas inversas laterales a la derecha. Sea B esa matriz y $\lambda = \mu = \beta = 0$, entonces:

$$B = \begin{pmatrix}
\frac{12}{37} & -\frac{8}{37} & -\frac{13}{74} \\
-\frac{17}{37} & -\frac{1}{37} & \frac{20}{37} \\
\frac{8}{37} & \frac{7}{37} & -\frac{21}{74} \\
0 & 0 & 0 \\
\end{pmatrix}$$



% ---------------------------------------------------------------------------
% FIN DEL DOCUMENTO
% ---------------------------------------------------------------------------

\end{document}